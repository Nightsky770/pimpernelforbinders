%!TeX root=../scarlettop.tex

\chapter{One O'Clock Precisely!}
\lettrine[lines=4]{S}{upper} had been extremely gay. All those present declared that never had Lady Blakeney been more adorable, nor that »demmed idiot« Sir Percy more amusing. His Royal Highness had laughed until the tears streamed down his cheeks at Blakeney's foolish yet funny repartees. His doggerel verse, »We seek him here, we seek him there,« etc., was sung to the tune of »Ho! Merry Britons!« and to the accompaniment of glasses knocked loudly against the table. Lord Grenville, moreover, had a most perfect cook—some wags asserted that he was a scion of the old French \textit{noblesse}, who, having lost his fortune, had come to seek it in the \textit{cuisine} of the Foreign Office.

Marguerite Blakeney was in her most brilliant mood, and surely not a soul in that crowded supper-room had even an inkling of the terrible struggle which was raging within her heart.

The clock was ticking so mercilessly on. It was long past midnight, and even the Prince of Wales was thinking of leaving the supper-table. Within the next half-hour the destinies of two brave men would be pitted against one another—the dearly-beloved brother and he, the unknown hero.

Marguerite had not even tried to see Chauvelin during this last hour; she knew that his keen, fox-like eyes would terrify her at once, and incline the balance of her decision towards Armand. Whilst she did not see him, there still lingered in her heart of hearts a vague, undefined hope that »something« would occur, something big, enormous, epoch-making, which would shift from her young, weak shoulders this terrible burden of responsibility, of having to choose between two such cruel alternatives.

But the minutes ticked on with that dull monotony which they invariably seem to assume when our very nerves ache with their incessant ticking.

After supper, dancing was resumed. His Royal Highness had left, and there was general talk of departing among the older guests; the young ones were indefatigable and had started on a new gavotte, which would fill the next quarter of an hour.

Marguerite did not feel equal to another dance; there is a limit to the most enduring self-control. Escorted by a Cabinet Minister, she had once more found her way to the tiny boudoir, still the most deserted among all the rooms. She knew that Chauvelin must be lying in wait for her somewhere, ready to seize the first possible opportunity for a \textit{tête-à-tête}. His eyes had met hers for a moment after the 'fore-supper minuet, and she knew that the keen diplomatist, with those searching pale eyes of his, had divined that her work was accomplished.

Fate had willed it so. Marguerite, torn by the most terrible conflict heart of woman can ever know, had resigned herself to its decrees. But Armand must be saved at any cost; he, first of all, for he was her brother, had been mother, father, friend to her ever since she, a tiny babe, had lost both her parents. To think of Armand dying a traitor's death on the guillotine was too horrible even to dwell upon—impossible, in fact. That could never be, never\textellipsis \allowbreak  As for the stranger, the hero\textellipsis \allowbreak  well! there, let Fate decide. Marguerite would redeem her brother's life at the hands of the relentless enemy, then let that cunning Scarlet Pimpernel extricate himself after that.

Perhaps—vaguely—Marguerite hoped that the daring plotter, who for so many months had baffled an army of spies, would still manage to evade Chauvelin and remain immune to the end.

She thought of all this, as she sat listening to the witty discourse of the Cabinet Minister, who, no doubt, felt that he had found in Lady Blakeney a most perfect listener. Suddenly she saw the keen, fox-like face of Chauvelin peeping through the curtained doorway.

»Lord Fancourt,« she said to the Minister, »will you do me a service?«

»I am entirely at your ladyship's service,« he replied gallantly.

»Will you see if my husband is still in the card-room? And if he is, will you tell him that I am very tired, and would be glad to go home soon.«

The commands of a beautiful woman are binding on all mankind, even on Cabinet Ministers. Lord Fancourt prepared to obey instantly.

»I do not like to leave your ladyship alone,« he said.

»Never fear. I shall be quite safe here—and, I think, undisturbed\textellipsis \allowbreak  but I am really tired. You know Sir Percy will drive back to Richmond. It is a long way, and we shall not—an we do not hurry—get home before daybreak.«

Lord Fancourt had perforce to go.

The moment he had disappeared, Chauvelin slipped into the room, and the next instant stood calm and impassive by her side.

»You have news for me?« he said.

An icy mantle seemed to have suddenly settled round Marguerite's shoulders; though her cheeks glowed with fire, she felt chilled and numbed. Oh, Armand! will you ever know the terrible sacrifice of pride, of dignity, of womanliness a devoted sister is making for your sake?

»Nothing of importance,« she said, staring mechanically before her, »but it might prove a clue. I contrived—no matter how—to detect Sir Andrew Ffoulkes in the very act of burning a paper at one of these candles, in this very room. That paper I succeeded in holding between my fingers for the space of two minutes, and to cast my eye on it for that of ten seconds.«

»Time enough to learn its contents?« asked Chauvelin, quietly.

She nodded. Then she continued in the same even, mechanical tone of voice\longdash


»In the corner of the paper there was the usual rough device of a small star-shaped flower. Above it I read two lines, everything else was scorched and blackened by the flame.«

»And what were these two lines?«

Her throat seemed suddenly to have contracted. For an instant she felt that she could not speak the words, which might send a brave man to his death.

»It is lucky that the whole paper was not burned,« added Chauve\-lin, with dry sarcasm, »for it might have fared ill with Armand St~Just. What were the two lines, citoyenne?«

»One was, »I start myself to-morrow,«« she said quietly; »the other—\allowbreak»If you wish to speak to me, I shall be in the supper-room at one o'clock precisely.««

Chauvelin looked up at the clock just above the mantelpiece.

»Then I have plenty of time,« he said placidly.

»What are you going to do?« she asked.

She was pale as a statue, her hands were icy cold, her head and heart throbbed with the awful strain upon her nerves. Oh, this was cruel! cruel! What had she done to have deserved all this? Her choice was made: had she done a vile action or one that was sublime? The recording angel, who writes in the book of gold, alone could give an answer.

»What are you going to do?« she repeated mechanically.

»Oh, nothing for the present. After that it will depend.«

»On what?«

»On whom I shall see in the supper-room at one o'clock precisely.«

»You will see the Scarlet Pimpernel, of course. But you do not know him.«

»No. But I shall presently.«

»Sir Andrew will have warned him.«

»I think not. When you parted from him after the minuet he stood and watched you, for a moment or two, with a look which gave me to understand that something had happened between you. It was only natural, was it not? that I should make a shrewd guess as to the nature of that »something.« I thereupon engaged the young gallant in a long and animated conversation—we discussed Herr Glück's singular success in London—until a lady claimed his arm for supper.«

»Since then?«

»I did not lose sight of him through supper. When we all came upstairs again, Lady Portarles buttonholed him and started on the subject of pretty Mlle. Suzanne de Tournay. I knew he would not move until Lady Portarles had exhausted the subject, which will not be for another quarter of an hour at least, and it is five minutes to one now.«

He was preparing to go, and went up to the doorway, where, drawing aside the curtain, he stood for a moment pointing out to Marguerite the distant figure of Sir Andrew Ffoulkes in close conversation with Lady Portarles.

»I think,« he said, with a triumphant smile, »that I may safely expect to find the person I seek in the dining-room, fair lady.«

»There may be more than one.«

»Whoever is there, as the clock strikes one, will be shadowed by one of my men; of these, one, or perhaps two, or even three, will leave for France to-morrow. \textit{One} of these will be the »Scarlet Pimpernel.««

»Yes?—And?«

»I also, fair lady, will leave for France to-morrow. The papers found at Dover upon the person of Sir Andrew Ffoulkes speak of the neighbourhood of Calais, of an inn which I know well, called »Le Chat Gris,« of a lonely place somewhere on the coast—the Père Blanchard's hut—which I must endeavour to find. All these places are given as the point where this meddlesome Englishman has bidden the traitor de Tournay and others to meet his emissaries. But it seems that he has decided not to send his emissaries, that »he will start himself to-morrow.« Now, one of those persons whom I shall see anon in the supper-room, will be journeying to Calais, and I shall follow that person, until I have tracked him to where those fugitive aristocrats await him; for that person, fair lady, will be the man whom I have sought for, for nearly a year, the man whose energy has outdone me, whose ingenuity has baffled me, whose audacity has set me wondering—yes! me!—who have seen a trick or two in my time—the mysterious and elusive Scarlet Pimpernel.«

»And Armand?« she pleaded.

»Have I ever broken my word? I promise you that the day the Scarlet Pimpernel and I start for France, I will send you that imprudent letter of his by special courier. More than that, I will pledge you the word of France, that the day I lay hands on that meddlesome Englishman, St~Just will be here in England, safe in the arms of his charming sister.«

And with a deep and elaborate bow and another look at the clock, Chauvelin glided out of the room.

It seemed to Marguerite that through all the noise, all the din of music, dancing, and laughter, she could hear his cat-like tread, gliding through the vast reception-rooms; that she could hear him go down the massive staircase, reach the dining-room and open the door. Fate \textit{had} decided, had made her speak, had made her do a vile and abominable thing, for the sake of the brother she loved. She lay back in her chair, passive and still, seeing the figure of her relentless enemy ever present before her aching eyes.

When Chauvelin reached the supper-room it was quite deserted. It had that woebegone, forsaken, tawdry appearance, which reminds one so much of a ball-dress, the morning after.

Half-empty glasses littered the table, unfolded napkins lay about, the chairs—turned towards one another in groups of twos and threes—seemed like the seats of ghosts, in close conversation with one another. There were sets of two chairs—very close to one another—in the far corners of the room, which spoke of recent whispered flirtations, over cold game-pie and champagne; there were sets of three and four chairs, that recalled pleasant, animated discussions over the latest scandals; there were chairs straight up in a row that still looked starchy, critical, acid, like antiquated dowagers; there were a few isolated, single chairs, close to the table, that spoke of gourmands intent on the most \textit{recherché} dishes, and others overturned on the floor, that spoke volumes on the subject of my Lord Grenville's cellars.

It was a ghostlike replica, in fact, of that fashionable gathering upstairs; a ghost that haunts every house where balls and good suppers are given; a picture drawn with white chalk on grey cardboard, dull and colourless, now that the bright silk dresses and gorgeously embroidered coats were no longer there to fill in the foreground, and now that the candles flickered sleepily in their sockets.

Chauvelin smiled benignly, and rubbing his long, thin hands together, he looked round the deserted supper-room, whence even the last flunkey had retired in order to join his friends in the hall below. All was silence in the dimly-lighted room, whilst the sound of the gavotte, the hum of distant talk and laughter, and the rumble of an occasional coach outside, only seemed to reach this palace of the Sleeping Beauty as the murmur of some flitting spooks far away.

It all looked so peaceful, so luxurious, and so still, that the keenest observer—a veritable prophet—could never have guessed that, at this present moment, that deserted supper-room was nothing but a trap laid for the capture of the most cunning and audacious plotter those stirring times had ever seen.

Chauvelin pondered and tried to peer into the immediate future. What would this man be like, whom he and the leaders of a whole revolution had sworn to bring to his death? Everything about him was weird and mysterious; his personality, which he had so cunningly concealed, the power he wielded over nineteen English gentlemen who seemed to obey his every command blindly and enthusiastically, the passionate love and submission he had roused in his little trained band, and, above all, his marvellous audacity, the boundless impudence which had caused him to beard his most implacable enemies, within the very walls of Paris.

No wonder that in France the \textit{sobriquet} of the mysterious Englishman roused in the people a superstitious shudder. Chauvelin himself as he gazed round the deserted room, where presently the weird hero would appear, felt a strange feeling of awe creeping all down his spine.

But his plans were well laid. He felt sure that the Scarlet Pimpernel had not been warned, and felt equally sure that Marguerite Blakeney had not played him false. If she had\textellipsis \allowbreak  a cruel look, that would have made her shudder, gleamed in Chauvelin's keen, pale eyes. If she had played him a trick, Armand St~Just would suffer the extreme penalty.

But no, no! of course she had not played him false!

Fortunately the supper-room was deserted: this would make Chauve\-lin's task all the easier, when presently that unsuspecting enigma would enter it alone. No one was here now save Chauvelin himself.

Stay! as he surveyed with a satisfied smile the solitude of the room, the cunning agent of the French Government became aware of the peaceful, monotonous breathing of some one of my Lord Grenville's guests, who, no doubt, had supped both wisely and well, and was enjoying a quiet sleep, away from the din of the dancing above.

Chauvelin looked round once more, and there in the corner of a sofa, in the dark angle of the room, his mouth open, his eyes shut, the sweet sounds of peaceful slumbers proceeding from his nostrils, reclined the gorgeously-apparelled, long-limbed husband of the cleverest woman in Europe.

Chauvelin looked at him as he lay there, placid, unconscious, at peace with all the world and himself, after the best of suppers, and a smile, that was almost one of pity, softened for a moment the hard lines of the Frenchman's face and the sarcastic twinkle of his pale eyes.

Evidently the slumberer, deep in dreamless sleep, would not interfere with Chauvelin's trap for catching that cunning Scarlet Pimpernel. Again he rubbed his hands together, and, following the example of Sir Percy Blakeney, he, too, stretched himself out in the corner of another sofa, shut his eyes, opened his mouth, gave forth sounds of peaceful breathing, and\textellipsis waited!