%!TeX root=../scarlettop.tex

\chapter{In the Opera Box}
\lettrine[lines=4]{I}{t} was one of the gala nights at Covent Garden Theatre, the first of the autumn season in this memorable year of grace 1792. The house was packed, both in the smart orchestra boxes and the pit, as well as in the more plebeian balconies and galleries above. Glück's\footnote{German composer Christoph Willibald Glück, 1714-1787.} \textit{Orpheus}\footnote{\textit{Orphée et Eurydice}, first performed in French in 1774.} made a strong appeal to the more intellectual portions of the house, whilst the fashionable women, the gaily-dressed and brilliant throng, spoke to the eye of those who cared but little for this \enquote{latest importation from Germany.}

Selina Storace\footnote{Anna Selina Storace (1765-1817), known professionally as Nancy Storace, was a celebrated English soprano. Mozart wrote the role of Susanna in \textit{Le nozze di Figaro} especially for her.} had been duly applauded after her grand \textit{aria} by her numerous admirers; Benjamin Incledon, the acknowledged favourite of the ladies, had received special gracious recognition from the royal box; and now the curtain came down after the glorious finale to the second act, and the audience, which had hung spell-bound on the magic strains of the great maestro, seemed collectively to breathe a long sigh of satisfaction, previous to letting loose its hundreds of waggish and frivolous tongues.

In the smart orchestra boxes many well-known faces were to be seen. Mr Pitt, overweighted with cares of state, was finding brief relaxation in to-night's musical treat; the Prince of Wales, jovial, rotund, somewhat coarse and commonplace in appearance, moved about from box to box, spending brief quarters of an hour with those of his more intimate friends.

In Lord Grenville's box, too, a curious, interesting personality attracted everyone's attention; a thin, small figure with shrewd, sarcastic face and deep-set eyes, attentive to the music, keenly critical of the audience, dressed in immaculate black, with dark hair free from any powder. Lord Grenville---Foreign Secretary of State---paid him marked, though frigid deference.

Here and there, dotted about among distinctly English types of beauty, one or two foreign faces stood out in marked contrast: the haughty aristocratic cast of countenance of the many French royalist \textit{émigrés} who, persecuted by the relentless, revolutionary faction of their country, had found a peaceful refuge in England. On these faces sorrow and care were deeply writ; the women especially paid but little heed, either to the music or to the brilliant audience; no doubt their thoughts were far away with husband, brother, son maybe, still in peril, or lately succumbed to a cruel fate.

Among these the Comtesse de Tournay de Basserive, but lately arrived from France, was a most conspicuous figure: dressed in deep, heavy black silk, with only a white lace kerchief to relieve the aspect of mourning about her person, she sat beside Lady Portarles, who was vainly trying by witty sallies and somewhat broad jokes, to bring a smile to the Comtesse's sad mouth. Behind her sat little Suzanne and the Vicomte, both silent and somewhat shy among so many strangers. Suzanne's eyes seemed wistful; when she first entered the crowded house, she had looked eagerly all around, scanned every face, scrutinised every box. Evidently the one face she wished to see was not there, for she settled herself down quietly behind her mother, listened apathetically to the music, and took no further interest in the audience itself.

\enquote{Ah, Lord Grenville,} said Lady Portarles, as following a discreet knock, the clever, interesting head of the Secretary of State appeared in the doorway of the box, \enquote{you could not arrive more \textit{à propos}. Here is Madame la Comtesse de Tournay positively dying to hear the latest news from France.}

The distinguished diplomatist had come forward and was shaking hands with the ladies.

\enquote{Alas!} he said sadly, \enquote{it is of the very worst. The massacres continue; Paris literally reeks with blood; and the guillotine claims a hundred victims a day.}

Pale and tearful, the Comtesse was leaning back in her chair, listening horror-struck to this brief and graphic account of what went on in her own misguided country.

\enquote{Ah, Monsieur!} she said in broken English, \enquote{it is dreadful to hear all that---and my poor husband still in that awful country. It is terrible for me to be sitting here, in a theatre, all safe and in peace, whilst he is in such peril.}

\enquote{Lud, Madame!} said honest, bluff Lady Portarles, \enquote{your sitting in a convent won't make your husband safe, and you have your children to consider: they are too young to be dosed with anxiety and premature mourning.}

The Comtesse smiled through her tears at the vehemence of her friend. Lady Portarles, whose voice and manner would not have misfitted a jockey, had a heart of gold, and hid the most genuine sympathy and most gentle kindliness, beneath the somewhat coarse manners affected by some ladies at that time.

\enquote{Besides which, Madame,} added Lord Grenville, \enquote{did you not tell me yesterday that the League of the Scarlet Pimpernel had pledged their honour to bring M. le Comte safely across the Channel?}

\enquote{Ah, yes!} replied the Comtesse, \enquote{and that is my only hope. I saw Lord Hastings yesterday... he reassured me again.}

\enquote{Then I am sure you need have no fear. What the league have sworn, that they surely will accomplish. Ah!} added the old diplomatist with a sigh, \enquote{if I were but a few years younger...}

\enquote{La, man!} interrupted honest Lady Portarles, \enquote{you are still young enough to turn your back on that French scarecrow that sits enthroned in your box to-night.}

\enquote{I wish I could... but your ladyship must remember that in serving our country we must put prejudices aside. M. Chauvelin is the accredited agent of his Government...}

\enquote{Odd's fish, man!} she retorted, \enquote{you don't call those bloodthirsty ruffians over there a government, do you?}

\enquote{It has not been thought advisable as yet,} said the Minister, guardedly, \enquote{for England to break off diplomatic relations with France, and we cannot therefore refuse to receive with courtesy the agent she wishes to send to us.}

\enquote{Diplomatic relations be demmed, my lord! That sly little fox over there is nothing but a spy, I'll warrant, and you'll find---an I'm much mistaken, that he'll concern himself little with diplomacy, beyond trying to do mischief to royalist refugees---to our heroic Scarlet Pimpernel and to the members of that brave little league.}

\enquote{I am sure,} said the Comtesse, pursing up her thin lips, \enquote{that if this Chauvelin wishes to do us mischief, he will find a faithful ally in Lady Blakeney.}

\enquote{Bless the woman!} ejaculated Lady Portarles, \enquote{did ever anyone see such perversity? My Lord Grenville, you have the gift of the gab, will you please explain to Madame la Comtesse that she is acting like a fool. In your position here in England, Madame,} she added, turning a wrathful and resolute face towards the Comtesse, \enquote{you cannot afford to put on the hoity-toity airs you French aristocrats are so fond of. Lady Blakeney may or may not be in sympathy with those ruffians in France; she may or may not have had anything to do with the arrest and condemnation of St~Cyr, or whatever the man's name is, but she is the leader of fashion in this country; Sir Percy Blakeney has more money than any half-dozen other men put together, he is hand and glove with royalty, and your trying to snub Lady Blakeney will not harm her, but will make you look a fool. Isn't that so, my lord?}

But what Lord Grenville thought of this matter, or to what reflections this homely tirade of Lady Portarles led the Comtesse de Tournay, remained unspoken, for the curtain had just risen on the third act of \textit{Orpheus}, and admonishments to silence came from every part of the house.

Lord Grenville took a hasty farewell of the ladies and slipped back into his box, where M. Chauvelin had sat all through this \textit{entr'acte}, with his eternal snuff-box in his hand, and with his keen pale eyes intently fixed upon a box opposite to him, where, with much frou-frou of silken skirts, much laughter and general stir of curiosity amongst the audience, Marguerite Blakeney had just entered, accompanied by her husband, and looking divinely pretty beneath the wealth of her golden, reddish curls, slightly besprinkled with powder, and tied back at the nape of her graceful neck with a gigantic black bow. Always dressed in the very latest vagary of fashion, Marguerite alone among the ladies that night had discarded the cross-over fichu and broad-lapelled over-dress, which had been in fashion for the last two or three years. She wore the short-waisted classical-shaped gown, which so soon was to become the approved mode in every country in Europe. It suited her graceful, regal figure to perfection, composed as it was of shimmering stuff which seemed a mass of rich gold embroidery.

As she entered, she leant for a moment out of the box, taking stock of all those present whom she knew. Many bowed to her as she did so, and from the royal box there came also a quick and gracious salute.

Chauvelin watched her intently all through the commencement of the third act, as she sat enthralled with the music, her exquisite little hand toying with a small jewelled fan, her regal head, her throat, arms and neck covered with magnificent diamonds and rare gems, the gift of the adoring husband who sprawled leisurely by her side.

Marguerite was passionately fond of music. \textit{Orpheus} charmed her to-night. The very joy of living was writ plainly upon the sweet young face, it sparkled out of the merry blue eyes and lit up the smile that lurked around the lips. She was after all but five-and-twenty, in the heyday of youth, the darling of a brilliant throng, adored, \textit{fêted}, petted, cherished. Two days ago the \textit{Day Dream} had returned from Calais, bringing her news that her idolised brother had safely landed, that he thought of her, and would be prudent for her sake.

What wonder for the moment, and listening to Glück's impassioned strains, that she forgot her disillusionments, forgot her vanished love-dreams, forgot even the lazy, good-humoured nonentity who had made up for his lack of spiritual attainments by lavishing worldly advantages upon her.

He had stayed beside her in the box just as long as convention demanded, making way for His Royal Highness, and for the host of admirers who in a continued procession came to pay homage to the queen of fashion. Sir Percy had strolled away, to talk to more congenial friends probably. Marguerite did not even wonder whither he had gone---she cared so little; she had had a little court round her, composed of the \textit{jeunesse dorée} of London, and had just dismissed them all, wishing to be alone with Glück for a brief while.

A discreet knock at the door roused her from her enjoyment.

\enquote{Come in,} she said with some impatience, without turning to look at the intruder.

Chauvelin, waiting for his opportunity, noted that she was alone, and now, without pausing for that impatient \enquote{Come in,} he quietly slipped into the box, and the next moment was standing behind Marguerite's chair.

\enquote{A word with you, citoyenne,} he said quietly.

Marguerite turned quickly, in alarm, which was not altogether feigned.

\enquote{Lud, man! you frightened me,} she said with a forced little laugh, \enquote{your presence is entirely inopportune. I want to listen to Glück, and have no mind for talking.}

\enquote{But this is my only opportunity,} he said, as quietly, and without waiting for permission, he drew a chair close behind her---so close that he could whisper in her ear, without disturbing the audience, and without being seen, in the dark background of the box. \enquote{This is my only opportunity,} he repeated, as she vouchsafed him no reply, \enquote{Lady Blakeney is always so surrounded, so \textit{fêted} by her court, that a mere old friend has but very little chance.}

\enquote{Faith, man!} she said impatiently, \enquote{you must seek for another opportunity then. I am going to Lord Grenville's ball to-night after the opera. So are you, probably. I'll give you five minutes then...}

\enquote{Three minutes in the privacy of this box are quite sufficient for me,} he rejoined placidly, \enquote{and I think that you would be wise to listen to me, Citoyenne St~Just.}

Marguerite instinctively shivered. Chauvelin had not raised his voice above a whisper; he was now quietly taking a pinch of snuff, yet there was something in his attitude, something in those pale, foxy eyes, which seemed to freeze the blood in her veins, as would the sight of some deadly hitherto unguessed peril.

\enquote{Is that a threat, citoyen?} she asked at last.

\enquote{Nay, fair lady,} he said gallantly, \enquote{only an arrow shot into the air.}

He paused a moment, like a cat which sees a mouse running heedlessly by, ready to spring, yet waiting with that feline sense of enjoyment of mischief about to be done. Then he said quietly---

\enquote{Your brother, St~Just, is in peril.}

Not a muscle moved in the beautiful face before him. He could only see it in profile, for Marguerite seemed to be watching the stage intently, but Chauvelin was a keen observer; he noticed the sudden rigidity of the eyes, the hardening of the mouth, the sharp, almost paralysed tension of the beautiful, graceful figure.

\enquote{Lud, then,} she said, with affected merriment, \enquote{since `tis one of your imaginary plots, you'd best go back to your own seat and leave me to enjoy the music.}

And with her hand she began to beat time nervously against the cushion of the box. Selina Storace was singing the \enquote{Che farò} to an audience that hung spellbound upon the prima donna's lips. Chauvelin did not move from his seat; he quietly watched that tiny nervous hand, the only indication that his shaft had indeed struck home.

\enquote{Well?} she said suddenly and irrelevantly, and with the same feigned unconcern.

\enquote{Well, citoyenne?} he rejoined placidly.

\enquote{About my brother?}

\enquote{I have news of him for you which, I think, will interest you, but first let me explain... May I?}

The question was unnecessary. He felt, though Marguerite still held her head steadily averted from him, that her every nerve was strained to hear what he had to say.

\enquote{The other day, citoyenne,} he said, \enquote{I asked for your help... France needed it, and I thought I could rely on you, but you gave me your answer... Since then the exigencies of my own affairs and your own social duties have kept us apart... although many things have happened...}

\enquote{To the point, I pray you, citoyen,} she said lightly; \enquote{the music is entrancing, and the audience will get impatient of your talk.}

\enquote{One moment, citoyenne. The day on which I had the honour of meeting you at Dover, and less than an hour after I had your final answer, I obtained possession of some papers, which revealed another of those subtle schemes for the escape of a batch of French aristocrats---that traitor de Tournay amongst others---all organised by that arch-meddler, the Scarlet Pimpernel. Some of the threads, too, of this mysterious organisation have fallen into my hands, but not all, and I want you---nay! you \textit{must} help me to gather them together.}

Marguerite seemed to have listened to him with marked impatience; she now shrugged her shoulders and said gaily---

\enquote{Bah! man. Have I not already told you that I care nought about your schemes or about the Scarlet Pimpernel. And had you not spoken about my brother...}

\enquote{A little patience, I entreat, citoyenne,} he continued imperturbably. \enquote{Two gentlemen, Lord Antony Dewhurst and Sir Andrew Ffoulkes were at \enquote{The Fisherman's Rest} at Dover that same night.}

\enquote{I know. I saw them there.}

\enquote{They were already known to my spies as members of that accursed league. It was Sir Andrew Ffoulkes who escorted the Comtesse de Tournay and her children across the Channel. When the two young men were alone, my spies forced their way into the coffee-room of the inn, gagged and pinioned the two gallants, seized their papers, and brought them to me.}

In a moment she had guessed the danger. Papers?... Had Armand been imprudent?... The very thought struck her with nameless terror. Still she would not let this man see that she feared; she laughed gaily and lightly.

\enquote{Faith! and your impudence passes belief,} she said merrily. \enquote{Robbery and violence!---in England!---in a crowd\-ed inn! Your men might have been caught in the act!}

\enquote{What if they had? They are children of France, and have been trained by your humble servant. Had they been caught they would have gone to jail, or even to the gallows, without a word of protest or indiscretion; at any rate it was well worth the risk. A crowded inn is safer for these little operations than you think, and my men have experience.}

\enquote{Well? And those papers?} she asked carelessly.

\enquote{Unfortunately, though they have given me cognisance of certain names... certain movements... enough, I think, to thwart their projected \textit{coup} for the moment, it would only be for the moment, and still leaves me in ignorance of the identity of the Scarlet Pimpernel.}

\enquote{La! my friend,} she said, with the same assumed flippancy of manner, \enquote{then you are where you were before, aren't you? and you can let me enjoy the last strophe of the \textit{aria}. Faith!} she added, ostentatiously smothering an imaginary yawn, \enquote{had you not spoken about my brother...}

\enquote{I am coming to him now, citoyenne. Among the papers there was a letter to Sir Andrew Ffoulkes, written by your brother, St~Just.}

\enquote{Well? And?}

\enquote{That letter shows him to be not only in sympathy with the enemies of France, but actually a helper, if not a member, of the League of the Scarlet Pimpernel.}

The blow had been struck at last. All along, Marguerite had been expecting it; she would not show fear, she was determined to seem unconcerned, flippant even. She wished, when the shock came, to be prepared for it, to have all her wits about her---those wits which had been nicknamed the keenest in Europe. Even now she did not flinch. She knew that Chauvelin had spoken the truth; the man was too earnest, too blindly devoted to the misguided cause he had at heart, too proud of his countrymen, of those makers of revolutions, to stoop to low, purposeless falsehoods.

That letter of Armand's---foolish, imprudent Armand---was in Chauvelin's hands. Marguerite knew that as if she had seen the letter with her own eyes; and Chauvelin would hold that letter for purposes of his own, until it suited him to destroy it or to make use of it against Armand. All that she knew, and yet she continued to laugh more gaily, more loudly than she had done before.

\enquote{La, man!} she said, speaking over her shoulder and looking him full and squarely in the face, \enquote{did I not say it was some imaginary plot... Armand in league with that enigmatic Scarlet Pimpernel!... Armand busy helping those French aristocrats whom he despises!... Faith, the tale does infinite credit to your imagination!}

\enquote{Let me make my point clear, citoyenne,} said Chauvelin, with the same unruffled calm, \enquote{I must assure you that St~Just is compromised beyond the slightest hope of pardon.}

Inside the orchestra box all was silent for a moment or two. Marguerite sat, straight upright, rigid and inert, trying to think, trying to face the situation, to realise what had best be done.

In the house Storace had finished the \textit{aria}, and was even now bowing in her classic garb, but in approved eighteenth-century fashion, to the enthusiastic audience, who cheered her to the echo.

\enquote{Chauvelin,} said Marguerite Blakeney at last, quietly, and without that touch of bravado which had characterised her attitude all along, \enquote{Chauvelin, my friend, shall we try to understand one another. It seems that my wits have become rusty by contact with this damp climate. Now, tell me, you are very anxious to discover the identity of the Scarlet Pimpernel, isn't that so?}

\enquote{France's most bitter enemy, citoyenne... all the more dangerous, as he works in the dark.}

\enquote{All the more noble, you mean... Well!---and you would now force me to do some spying work for you in exchange for my brother Armand's safety?---Is that it?}

\enquote{Fie! two very ugly words, fair lady,} protested Chauvelin, urbanely. \enquote{There can be no question of force, and the service which I would ask of you, in the name of France, could never be called by the shocking name of spying.}

\enquote{At any rate, that is what it is called over here,} she said drily. \enquote{That is your intention, is it not?}

\enquote{My intention is, that you yourself win a free pardon for Armand St~Just by doing me a small service.}

\enquote{What is it?}

\enquote{Only watch for me to-night, Citoyenne St~Just,} he said eagerly. \enquote{Listen: among the papers which were found about the person of Sir Andrew Ffoulkes there was a tiny note. See!} he added, taking a tiny scrap of paper from his pocket-book and handing it to her.

It was the same scrap of paper which, four days ago, the two young men had been in the act of reading, at the very moment when they were attacked by Chauvelin's minions. Marguerite took it mechanically and stooped to read it. There were only two lines, written in a distorted, evidently disguised, handwriting; she read them half aloud---

\blockquote{Remember we must not meet more often than is strictly necessary. You have all instructions for the 2\textsuperscript{nd}. If you wish to speak to me again, I shall be at G.'s ball.}

\enquote{What does it mean?} she asked.

\enquote{Look again, citoyenne, and you will understand.}

\enquote{There is a device here in the corner, a small red flower...}

\enquote{Yes.}

\enquote{The Scarlet Pimpernel,} she said eagerly, \enquote{and G.’s ball means Grenville's ball... He will be at my Lord Grenville's ball to-night.}

\enquote{That is how I interpret the note, citoyenne,} concluded Chauvelin, blandly. \enquote{Lord Antony Dewhurst and Sir Andrew Ffoulkes, after they were pinioned and searched by my spies, were carried by my orders to a lonely house on the Dover Road, which I had rented for the purpose: there they remained close prisoners until this morning. But having found this tiny scrap of paper, my intention was that they should be in London, in time to attend my Lord Grenville's ball. You see, do you not? that they must have a great deal to say to their chief... and thus they will have an opportunity of speaking to him to-night, just as he directed them to do. Therefore, this morning, those two young gallants found every bar and bolt open in that lonely house on the Dover Road, their jailers disappeared, and two good horses standing ready saddled and tethered in the yard. I have not seen them yet, but I think we may safely conclude that they did not draw rein until they reached London. Now you see how simple it all is, citoyenne!}

\enquote{It does seem simple, doesn't it?} she said, with a final bitter attempt at flippancy, \enquote{when you want to kill a chicken... you take hold of it... then you wring its neck... it's only the chicken who does not find it quite so simple. Now you hold a knife at my throat, and a hostage for my obedience... You find it simple... I don't.}

\enquote{Nay, citoyenne, I offer you a chance of saving the brother you love from the consequences of his own folly.}

Marguerite's face softened, her eyes at last grew moist, as she murmured, half to herself:

\enquote{The only being in the world who has loved me truly and constantly... But what do you want me to do, Chauvelin?} she said, with a world of despair in her tear-choked voice. \enquote{In my present position, it is well-nigh impossible!}

\enquote{Nay, citoyenne,} he said drily and relentlessly, not heeding that despairing, childlike appeal, which might have melted a heart of stone, \enquote{as Lady Blakeney, no one suspects you, and with your help to-night I may---who knows?---succeed in finally establishing the identity of the Scarlet Pimpernel... You are going to the ball anon... Watch for me there, citoyenne, watch and listen... You can tell me if you hear a chance word or whisper... You can note everyone to whom Sir Andrew Ffoulkes or Lord Antony Dewhurst will speak. You are absolutely beyond suspicion now. The Scarlet Pimpernel will be at Lord Grenville's ball to-night. Find out who he is, and I will pledge the word of France that your brother shall be safe.}

Chauvelin was putting the knife to her throat. Marguerite felt herself entangled in one of those webs, from which she could hope for no escape. A precious hostage was being held for her obedience: for she knew that this man would never make an empty threat. No doubt Armand was already signalled to the Committee of Public Safety as one of the \enquote{suspect}; he would not be allowed to leave France again, and would be ruthlessly struck, if she refused to obey Chauvelin. For a moment---woman-like---she still hoped to temporise. She held out her hand to this man, whom she now feared and hated.

\enquote{If I promise to help you in this matter, Chauvelin,} she said pleasantly, \enquote{will you give me that letter of St~Just's?}

\enquote{If you render me useful assistance to-night, citoyenne,} he replied with a sarcastic smile, \enquote{I will give you that letter... to-morrow.}

\enquote{You do not trust me?}

\enquote{I trust you absolutely, dear lady, but St~Just's life is forfeit to his country... it rests with you to redeem it.}

\enquote{I may be powerless to help you,} she pleaded, \enquote{were I ever so willing.}

\enquote{That would be terrible indeed,} he said quietly, \enquote{for you... and for St~Just.}

Marguerite shuddered. She felt that from this man she could expect no mercy. All-powerful, he held the beloved life in the hollow of his hand. She knew him too well not to know that, if he failed in gaining his own ends, he would be pitiless.

She felt cold in spite of the oppressive air of the opera-house. The heart-appealing strains of the music seemed to reach her, as from a distant land. She drew her costly lace scarf up around her shoulders, and sat silently watching the brilliant scene, as if in a dream.

For a moment her thoughts wandered away from the loved one who was in danger, to that other man who also had a claim on her confidence and her affection. She felt lonely, frightened for Armand's sake; she longed to seek comfort and advice from someone who would know how to help and console. Sir Percy Blakeney had loved her once; he was her husband; why should she stand alone through this terrible ordeal? He had very little brains, it is true, but he had plenty of muscle: surely, if she provided the thought, and he the manly energy and pluck, together they could outwit the astute diplomatist, and save the hostage from his vengeful hands, without imperilling the life of the noble leader of that gallant little band of heroes. Sir Percy knew St~Just well---he seemed attached to him---she was sure that he could help.

Chauvelin was taking no further heed of her. He had said his cruel \enquote{Either---or---} and left her to decide. He, in his turn now, appeared to be absorbed in the soul-stirring melodies of \textit{Orpheus}, and was beating time to the music with his sharp, ferret-like head.

A discreet rap at the door roused Marguerite from her thoughts. It was Sir Percy Blakeney, tall, sleepy, good-humoured, and wearing that half-shy, half-inane smile, which just now seemed to irritate her every nerve.

\enquote{Er... your chair is outside... m'dear,} he said, with his most exasperating drawl, \enquote{I suppose you will want to go to that demmed ball... Excuse me---er---Monsieur Chauvelin---I had not observed you...}

He extended two slender, white fingers towards Chauvelin, who had risen when Sir Percy entered the box.

\enquote{Are you coming, m'dear?}

\enquote{Hush! Sh! Sh!} came in angry remonstrance from different parts of the house.

\enquote{Demmed impudence,} commented Sir Percy with a good-natured smile.

Marguerite sighed impatiently. Her last hope seemed suddenly to have vanished away. She wrapped her cloak round her and without looking at her husband:

\enquote{I am ready to go,} she said, taking his arm. At the door of the box she turned and looked straight at Chauvelin, who, with his \textit{chapeau-bras} under his arm, and a curious smile round his thin lips, was preparing to follow the strangely ill-assorted couple.

\enquote{It is only \textit{au revoir}, Chauvelin,} she said pleasantly, \enquote{we shall meet at my Lord Grenville's ball, anon.}

And in her eyes the astute Frenchman read, no doubt, something which caused him profound satisfaction, for, with a sarcastic smile, he took a delicate pinch of snuff, then, having dusted his dainty lace jabot, he rubbed his thin, bony hands contentedly together.