%!TeX root=../scarlettop.tex

\chapter{The Death-Trap}
\lettrine[lines=4]{T}{he} next quarter of an hour went by swiftly and noiselessly. In the room downstairs, Brogard had for a while busied himself with clearing the table, and re-arranging it for another guest.

It was because she watched these preparations that Marguerite found the time slipping by more pleasantly. It was for Percy that this semblance of supper was being got ready. Evidently Brogard had a certain amount of respect for the tall Englishman, as he seemed to take some trouble in making the place look a trifle less uninviting than it had done before.

He even produced, from some hidden recess in the old dresser, what actually looked like a table-cloth; and when he spread it out, and saw it was full of holes, he shook his head dubiously for a while, then was at much pains so to spread it over the table as to hide most of its blemishes.

Then he got out a serviette, also old and ragged, but possessing some measure of cleanliness, and with this he carefully wiped the glasses, spoons and plates, which he put on the table.

Marguerite could not help smiling to herself as she watched all these preparations, which Brogard accomplished to an accompaniment of muttered oaths. Clearly the great height and bulk of the Englishman, or perhaps the weight of his fist, had overawed this free-born citizen of France, or he would never have been at such trouble for any \textit{sacrré aristo}.

When the table was set—such as it was—Brogard surveyed it with evident satisfaction. He then dusted one of the chairs with the corner of his blouse, gave a stir to the stock-pot, threw a fresh bundle of faggots on to the fire, and slouched out of the room.

Marguerite was left alone with her reflections. She had spread her travelling cloak over the straw, and was sitting fairly comfortably, as the straw was fresh, and the evil odours from below came up to her only in a modified form.

But, momentarily, she was almost happy; happy because, when she peeped through the tattered curtains, she could see a rickety chair, a torn table-cloth, a glass, a plate and a spoon; that was all. But those mute and ugly things seemed to say to her that they were waiting for Percy; that soon, very soon, he would be here, that the squalid room being still empty, they would be alone together.

That thought was so heavenly, that Marguerite closed her eyes in order to shut out everything but that. In a few minutes she would be alone with him; she would run down the ladder, and let him see her; then he would take her in his arms, and she would let him see that, after that, she would gladly die for him, and with him, for earth could hold no greater happiness than that.

And then what would happen? She could not even remotely conjecture. She knew, of course, that Sir Andrew was right, that Percy would do everything he had set out to accomplish; that she—now she was here—could do nothing, beyond warning him to be cautious, since Chauvelin himself was on his track. After having cautioned him, she would perforce have to see him go off upon his terrible and daring mission; she could not even with a word or look, attempt to keep him back. She would have to obey, whatever he told her to do, even perhaps have to efface herself, and wait, in indescribable agony, whilst he, perhaps, went to his death.

But even that seemed less terrible to bear than the thought that he should never know how much she loved him—that at any rate would be spared her; the squalid room itself, which seemed to be waiting for him, told her that he would be here soon.

Suddenly her over-sensitive ears caught the sound of distant footsteps drawing near; her heart gave a wild leap of joy! Was it Percy at last? No! the step did not seem quite as long, nor quite as firm as his; she also thought that she could hear two distinct sets of footsteps. Yes! that was it! two men were coming this way. Two strangers perhaps, to get a drink, or\textellipsis \allowbreak  But she had not time to conjecture, for presently there was a peremptory call at the door, and the next moment it was violently thrown open from the outside, whilst a rough, commanding voice shouted,\longdash


»Hey! Citoyen Brogard! Holá!«

Marguerite could not see the newcomers, but, through a hole in one of the curtains, she could observe one portion of the room below.

She heard Brogard's shuffling footsteps, as he came out of the inner room, muttering his usual string of oaths. On seeing the strangers, however, he paused in the middle of the room, well within range of Marguerite's vision, looked at them, with even more withering contempt than he had bestowed upon his former guests, and muttered, »Sacrrrée soutane!«

Marguerite's heart seemed all at once to stop beating; her eyes, large and dilated, had fastened on one of the newcomers, who, at this point, had taken a quick step forward towards Brogard. He was dressed in the soutane, broad-brimmed hat and buckled shoes habitual to the French \textit{curé}, but as he stood opposite the innkeeper, he threw open his soutane for a moment, displaying the tricolour scarf of officialism, which sight immediately had the effect of transforming Brogard's attitude of contempt, into one of cringing obsequiousness.

It was the sight of this French \textit{curé}, which seemed to freeze the very blood in Marguerite's veins. She could not see his face, which was shaded by his broad-brimmed hat, but she recognised the thin, bony hands, the slight stoop, the whole gait of the man! It was Chauvelin!

The horror of the situation struck her as with a physical blow; the awful disappointment, the dread of what was to come, made her very senses reel, and she needed almost superhuman effort, not to fall senseless beneath it all.

»A plate of soup and a bottle of wine,« said Chauvelin imperiously to Brogard, »then clear out of here—understand? I want to be alone.«

Silently, and without any muttering this time, Brogard obeyed. Chauvelin sat down at the table, which had been prepared for the tall Englishman, and the innkeeper busied himself obsequiously round him, dishing up the soup and pouring out the wine. The man who had entered with Chauvelin and whom Marguerite could not see, stood waiting close by the door.

At a brusque sign from Chauvelin, Brogard had hurried back to the inner room, and the former now beckoned to the man who had accompanied him.

In him Marguerite at once recognised Desgas, Chauvelin's secretary and confidential factotum, whom she had often seen in Paris, in the days gone by. He crossed the room, and for a moment or two listened attentively at the Brogards' door.

»Not listening?« asked Chauvelin, curtly.

»No, citoyen.«

For a second Marguerite dreaded lest Chauvelin should order Desgas to search the place; what would happen if she were to be discovered, she hardly dared to imagine. Fortunately, however, Chauvelin seemed more impatient to talk to his secretary than afraid of spies, for he called Desgas quickly back to his side.

»The English schooner?« he asked.

»She was lost sight of at sundown, citoyen,« replied Desgas, »but was then making west, towards Cap Gris Nez.«

»Ah!—good!\longdash« muttered Chauvelin, »and now, about Captain Jutley?—what did he say?«

»He assured me that all the orders you sent him last week have been implicitly obeyed. All the roads which converge to this place have been patrolled night and day ever since: and the beach and cliffs have been most rigorously searched and guarded.«

»Does he know where this »Père Blanchard's hut« is?«

»No, citoyen, nobody seems to know of it by that name. There are any amount of fishermen's huts all along the coast, of course\textellipsis \allowbreak  but\textellipsis«

»That'll do. Now about to-night?« interrupted Chauvelin, impatiently.

»The roads and the beach are patrolled as usual, citoyen, and Captain Jutley awaits further orders.«

»Go back to him at once, then. Tell him to send reinforcements to the various patrols; and especially to those along the beach—you understand?«

Chauvelin spoke curtly and to the point, and every word he uttered struck at Marguerite's heart like the death-knell of her fondest hopes.

»The men,« he continued, »are to keep the sharpest possible look-out for any stranger who may be walking, riding, or driving, along the road or the beach, more especially for a tall stranger, whom I need not describe further, as probably he will be disguised; but he cannot very well conceal his height, except by stooping. You understand?«

»Perfectly, citoyen,« replied Desgas.

»As soon as any of the men have sighted a stranger, two of them are to keep him in view. The man who loses sight of the tall stranger, after he is once seen, will pay for his negligence with his life; but one man is to ride straight back here and report to me. Is that clear?«

»Absolutely clear, citoyen.«

»Very well, then. Go and see Jutley at once. See the reinforcements start off for the patrol duty, then ask the captain to let you have half-a-dozen more men and bring them here with you. You can be back in ten minutes. Go\longdash«

Desgas saluted and went to the door.

As Marguerite, sick with horror, listened to Chauvelin's directions to his underling, the whole of the plan for the capture of the Scarlet Pimpernel became appallingly clear to her. Chauvelin wished that the fugitives should be left in false security waiting in their hidden retreat until Percy joined them. Then the daring plotter was to be surrounded and caught red-handed, in the very act of aiding and abetting royalists, who were traitors to the republic. Thus, if his capture were noised abroad, even the British Government could not legally protest in his favour; having plotted with the enemies of the French Government, France had the right to put him to death.

Escape for him and them would be impossible. All the roads patrolled and watched, the trap well set, the net, wide at present, but drawing together tighter and tighter, until it closed upon the daring plotter, whose superhuman cunning even could not rescue him from its meshes now.

Desgas was about to go, but Chauvelin once more called him back. Marguerite vaguely wondered what further devilish plans he could have formed, in order to entrap one brave man, alone, against two-score of others. She looked at him as he turned to speak to Desgas; she could just see his face beneath the broad-brimmed \textit{curé}'s hat. There was at that moment so much deadly hatred, such fiendish malice in the thin face and pale, small eyes, that Marguerite's last hope died in her heart, for she felt that from this man she could expect no mercy.

»I had forgotten,« repeated Chauvelin, with a weird chuckle, as he rubbed his bony, talon-like hands one against the other, with a gesture of fiendish satisfaction. »The tall stranger may show fight. In any case no shooting, remember, except as a last resort. I want that tall stranger alive\textellipsis \allowbreak  if possible.«

He laughed, as Dante has told us that the devils laugh at sight of the torture of the damned. Marguerite had thought that by now she had lived through the whole gamut of horror and anguish that human heart could bear; yet now, when Desgas left the house, and she remained alone in this lonely, squalid room, with that fiend for company, she felt as if all that she had suffered was nothing compared with this. He continued to laugh and chuckle to himself for a while, rubbing his hands together in anticipation of his triumph.

His plans were well laid, and he might well triumph! Not a loophole was left, through which the bravest, the most cunning man might escape. Every road guarded, every corner watched, and in that lonely hut somewhere on the coast, a small band of fugitives waiting for their rescuer, and leading him to his death—nay! to worse than death. That fiend there, in a holy man's garb, was too much of a devil to allow a brave man to die the quick, sudden death of a soldier at the post of duty.

He, above all, longed to have the cunning enemy, who had so long baffled him, helpless in his power; he wished to gloat over him, to enjoy his downfall, to inflict upon him what moral and mental torture a deadly hatred alone can devise. The brave eagle, captured, and with noble wings clipped, was doomed to endure the gnawing of the rat. And she, his wife, who loved him, and who had brought him to this, could do nothing to help him.

Nothing, save to hope for death by his side, and for one brief moment in which to tell him that her love—whole, true and passionate—was entirely his.

Chauvelin was now sitting close to the table; he had taken off his hat, and Marguerite could just see the outline of his thin profile and pointed chin, as he bent over his meagre supper. He was evidently quite contented, and awaited events with perfect calm; he even seemed to enjoy Brogard's unsavoury fare. Marguerite wondered how so much hatred could lurk in one human being against another.

Suddenly, as she watched Chauvelin, a sound caught her ear, which turned her very heart to stone. And yet that sound was not calculated to inspire anyone with horror, for it was merely the cheerful sound of a gay, fresh voice singing lustily, »God save the King!«