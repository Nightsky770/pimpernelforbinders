%!TeX root=../scarlettop.tex

\chapter{Calais}

\lettrine[lines=4]{T}{he} weariest nights, the longest days, sooner or later must perforce come to an end. Marguerite had spent over fifteen hours in such acute mental torture as well-nigh drove her crazy. After a sleepless night, she rose early, wild with excitement, dying to start on her journey, terrified lest further obstacles lay in her way. She rose before anyone else in the house was astir, so frightened was she, lest she should miss the one golden opportunity of making a start.

When she came downstairs, she found Sir Andrew  Ffoulkes sitting in the coffee-room. He had been out half an hour earlier, and had gone to the Admiralty Pier, only to find that neither the French packet nor any privately chartered vessel could put out of Dover yet. The storm was then at its fullest, and the tide was on the turn. If the wind did not abate or change, they would perforce have to wait another ten or twelve hours until the next tide, before a start could be made. And the storm had not abated, the wind had not changed, and the tide was rapidly drawing out.

Marguerite felt the sickness of despair when she heard this melancholy news. Only the most firm resolution kept her from totally breaking down, and thus adding to the young man's anxiety, which evidently had become very keen.

Though he tried to disguise it, Marguerite could see that Sir Andrew was just as anxious as she was to reach his comrade and friend. This enforced inactivity was terrible to them both.

How they spent that wearisome day at Dover, Marguerite could never afterwards say. She was in terror of showing herself, lest Chauvelin's spies happened to be about, so she had a private sitting-room, and she and Sir Andrew sat there hour after hour, trying to take, at long intervals, some perfunctory meals, which little Sally would bring them, with nothing to do but to think, to conjecture, and only occasionally to hope.

The storm had abated just too late; the tide was by then too far out to allow a vessel to put off to sea. The wind had changed, and was settling down to a comfortable north-westerly breeze\allowbreak---\allowbreak a veritable godsend for a  speedy passage across to France.

And there those two waited, wondering if the hour would ever come when they could finally make a start. There had been one happy interval in this long weary day, and that was when Sir Andrew went down once again to the pier, and presently came back to tell Marguerite that he had chartered a quick schooner, whose skipper was ready to put to sea the moment the tide was favourable.

From that moment the hours seemed less wearisome; there was less hopelessness in the waiting; and at last, at five o'clock in the afternoon, Marguerite, closely veiled and followed by Sir Andrew Ffoulkes, who, in the guise of her lacquey, was carrying a number of impedimenta, found her way down to the pier.

Once on board, the keen, fresh sea-air revived her, the breeze was just strong enough to nicely swell the sails of the \textit{Foam Crest}, as she cut her way merrily towards the open.

The sunset was glorious after the storm, and Marguerite, as she watched the white cliffs of Dover gradually disappearing from view, felt more at peace and once more almost hopeful.

Sir Andrew was full of kind attentions, and she felt how lucky she had been to have him by her side in this, her great trouble.

Gradually the grey coast of France began to emerge from the fast-gathering evening mists. One or two lights could be seen flickering, and the spires of several churches to rise out of the surrounding haze.

Half an hour later Marguerite had landed upon French shore. She was back in that country where at this very moment men slaughtered their fellow-creatures by the hundreds, and sent innocent women and children in thousands to the block.

The very aspect of the country and its people, even in this remote sea-coast town, spoke of that seething revolution, three hundred miles away, in beautiful Paris, now rendered hideous by the constant flow of the blood of her noblest sons, by the wailing of the widows, and the cries of fatherless children.

The men all wore red caps\allowbreak---\allowbreak in various stages of  cleanliness\allowbreak---\allowbreak but all with the tricolour cockade pinned on the left-hand side. Marguerite noticed with a shudder that, instead of the laughing, merry countenance habitual to her own countrymen, their faces now invariably wore a look of sly distrust.

Every man nowadays was a spy upon his fellows: the most innocent word uttered in jest might at any time be brought up as a proof of aristocratic tendencies, or of treachery against the people. Even the women went about with a curious look of fear and of hate lurking in their brown eyes; and all watched Marguerite as she stepped on shore, followed by Sir Andrew, and murmured as she passed along: »\textit{Sacrés aristos!}« or else »\textit{Sacrés Anglais!}«

Otherwise their presence excited no further comment. Calais, even in those days, was in constant business communication with England, and English merchants were often to be seen on this coast. It was well known that in view of the heavy duties in England, a vast deal of French wines and brandies were smuggled across. This pleased the French \textit{bourgeois} immensely; he liked to see the English Government and the English king, both of whom he hated, cheated out of their revenues; and an English smuggler was always a welcome guest at the tumble-down taverns of Calais and Boulogne.

So, perhaps, as Sir Andrew gradually directed Marguerite through the tortuous streets of Calais, many of the population, who turned with an oath to look at the strangers clad in the English fashion, thought that they were bent on purchasing dutiable articles for their own fog-ridden country, and gave them no more than a passing thought.

Marguerite, however, wondered how her husband's tall, massive figure could have passed through Calais unobserved: she marvelled what disguise he assumed to do his noble work, without exciting too much attention.

Without exchanging more than a few words, Sir Andrew was leading her right across the town, to the other side from that where they had landed, and on the way towards Cap Gris Nez. The streets were narrow, tortuous, and mostly evil-smelling, with a mixture of stale fish and damp cellar odours. There had been heavy rain here during the storm last night, and sometimes Marguerite sank ankle-deep in the mud, for the roads were not lighted save by the occasional glimmer from a lamp inside a house.

But she did not heed any of these petty discomforts: »We may meet Blakeney at the »Chat Gris,«« Sir Andrew had said, when they landed, and she was walking as if on a carpet of rose-leaves, for she was going to meet him almost at once.

At last they reached their destination. Sir Andrew evidently knew the road, for he had walked unerringly in the dark, and had not asked his way from anyone. It was too dark then for Marguerite to notice the outside aspect of this house. The »Chat Gris,« as Sir Andrew had called it, was evidently a small wayside inn on the outskirts of Calais, and on the way to Gris Nez. It lay some little distance from the coast, for the sound of the sea seemed to come from afar.

Sir Andrew knocked at the door with the knob of his cane, and from within Marguerite heard a sort of grunt and the muttering of a number of oaths. Sir Andrew knocked again, this time more peremptorily: more oaths were heard, and then shuffling steps seemed to draw near the door. Presently this was thrown open, and Marguerite found herself on the threshold of the most dilapidated, most squalid room she had ever seen in all her life.

The paper, such as it was, was hanging from the walls in strips; there did not seem to be a single piece of furniture in the room that could, by the wildest stretch of imagination, be called »whole.« Most of the chairs had broken backs, others had no seats to them, one corner of the table was propped up with a bundle of faggots, there where the fourth leg had been broken.

In one corner of the room there was a huge hearth, over which hung a stock-pot, with a not altogether unpalatable odour of hot soup emanating therefrom. On one side of the room, high up in the wall, there was a species of loft, before which hung a tattered blue-and-white checked curtain. A rickety set of steps led up to this loft.

On the great bare walls, with their colourless paper, all stained with varied filth, there were chalked up at intervals in great bold characters, the words: »Liberté\allowbreak---\allowbreak Egalité\allowbreak---\allowbreak Fraternité.«

The whole of this sordid abode was dimly lighted by an evil-smelling oil-lamp, which hung from the rickety rafters of the ceiling. It all looked so horribly squalid, so dirty and uninviting, that Marguerite hardly dared to cross the threshold.

Sir Andrew, however, had stepped unhesitatingly forward.

»English travellers, citoyen!« he said boldly, and speaking in French.

The individual who had come to the door in response to Sir Andrew's knock, and who, presumably, was the owner of this squalid abode, was an elderly, heavily-built peasant, dressed in a dirty blue blouse, heavy sabots, from which wisps of straw protruded all round, shabby blue trousers, and the inevitable red cap with the tricolour cockade, that proclaimed his momentary political views. He carried a short wooden pipe, from which the odour of rank tobacco emanated. He looked with some suspicion and a great deal of contempt at the two travellers, muttered »\textit{Sacrrrés Anglais!}« and spat upon the ground to further show his independence of spirit, but, nevertheless, he stood aside to let them enter, no doubt well aware that these same \textit{sacrrrés Anglais} always had well-filled purses.

»Oh, lud!« said Marguerite, as she advanced into the room, holding her handkerchief to her dainty nose, »what a dreadful hole! Are you sure this is the place?«

»Aye! 'tis the place, sure enough,« replied the young man as, with his lace-edged, fashionable handkerchief, he dusted a chair for Marguerite to sit on; »but I vow I never saw a more villainous hole.«

»Faith!« she said, looking round with some curiosity and a great deal of horror at the dilapidated walls, the broken chairs, the rickety table, »it certainly does not look inviting.«

The landlord of the »Chat Gris«\allowbreak---\allowbreak by name, Brogard\allowbreak---\allowbreak had taken no further notice of his guests; he concluded that presently they would order supper, and in the meanwhile it was not for a free citizen to show deference, or even courtesy, to anyone, however smartly they might be dressed.

By the hearth sat a huddled-up figure clad, seemingly, mostly in rags: that figure was apparently a woman, although even that would have been hard to distinguish, except for the cap, which had once been white, and for what looked like the semblance of a petticoat. She was sitting mumbling to herself, and from time to time stirring the brew in her stock-pot.

»Hey, my friend!« said Sir Andrew at last, »we should like some supper... The citoyenne there,« he added, pointing to the huddled-up bundle of rags by the hearth, »is concocting some delicious soup, I'll warrant, and my mistress has not tasted food for several hours.«

It took Brogard some few moments to consider the question. A free citizen does not respond too readily to the wishes of those who happen to require something of him.

»\textit{Sacrrrés aristos!}« he murmured, and once more spat upon the ground.

Then he went very slowly up to a dresser which stood in a corner of the room; from this he took an old pewter soup-tureen and slowly, and without a word, he handed it to his better-half, who, in the same silence, began filling the tureen with the soup out of her stock-pot.

Marguerite had watched all these preparations with absolute horror; were it not for the earnestness of her purpose, she would incontinently have fled from this abode of dirt and evil smells.

»Faith! our host and hostess are not cheerful people,« said Sir Andrew, seeing the look of horror on Marguerite's face. »I would I could offer you a more hearty and more appetising meal... but I think you will find the soup eatable and the wine good; these people wallow in dirt, but live well as a rule.«

»Nay! I pray you, Sir Andrew,« she said gently, »be not anxious about me. My mind is scarce inclined to dwell on thoughts of supper.«

Brogard was slowly pursuing his gruesome preparations; he had placed a couple of spoons, also two glasses on the table, both of which Sir Andrew took the precaution of wiping carefully.

Brogard had also produced a bottle of wine and some bread, and Marguerite made an effort to draw her chair to the table and to make some pretence at eating. Sir Andrew, as befitting his \textit{rôle} of lacquey, stood behind her chair.

»Nay, Madame, I pray you,« he said, seeing that Marguerite seemed quite unable to eat, »I beg of you to try and swallow some food\allowbreak---\allowbreak remember you have need of all your strength.«

The soup certainly was not bad; it smelt and tasted good. Marguerite might have enjoyed it, but for the horrible surroundings. She broke the bread, however, and drank some of the wine.

»Nay, Sir Andrew,« she said, »I do not like to see you standing. You have need of food just as much as I have. This creature will only think that I am an eccentric Englishwoman eloping with her lacquey, if you'll sit down and partake of this semblance of supper beside me.«

Indeed, Brogard having placed what was strictly necessary upon the table, seemed not to trouble himself any further about his guests. The Mere Brogard had quietly shuffled out of the room, and the man stood and lounged about, smoking his evil-smelling pipe, sometimes under Marguerite's very nose, as any free-born citizen who was anybody's equal should do.

»Confound the brute!« said Sir Andrew, with native British wrath, as Brogard leant up against the table, smoking and looking down superciliously at these two \textit{sacrrrés Anglais}.

»In Heaven's name, man,« admonished Marguerite, hurriedly, seeing that Sir Andrew, with British-born instinct, was ominously clenching his fist, »remember that you are in France, and that in this year of grace this is the temper of the people.«

»I'd like to scrag the brute!« muttered Sir Andrew, savagely.

He had taken Marguerite's advice and sat next to her at table, and they were both making noble efforts to deceive one another, by pretending to eat and drink.

»I pray you,« said Marguerite, »keep the creature in a good temper, so that he may answer the questions we must put to him.«

»I'll do my best, but, begad! I'd sooner scrag him than question him. Hey! my friend,« he said pleasantly in French, and tapping Brogard lightly on the shoulder, »do you see many of our quality along these parts? Many English travellers, I mean?«

Brogard looked round at him, over his near shoulder, puffed away at his pipe for a moment or two as he was in no hurry, then muttered,\longdash


»Heu!\allowbreak---\allowbreak sometimes!«

»Ah!« said Sir Andrew, carelessly, »English travellers always know where they can get good wine, eh! my friend?\allowbreak---\allowbreak Now, tell me, my lady was desiring to know if by any chance you happen to have seen a great friend of hers, an English gentleman, who often comes to Calais on business; he is tall, and recently was on his way to Paris\allowbreak---\allowbreak my lady hoped to have met him in Calais.«

Marguerite tried not to look at Brogard, lest she should betray before him the burning anxiety with which she waited for his reply. But a free-born French citizen is never in any hurry to answer questions: Brogard took his time, then he said very slowly,\longdash


»Tall Englishman?\allowbreak---\allowbreak To-day!\allowbreak---\allowbreak Yes.«

»You have seen him?« asked Sir Andrew, carelessly.

»Yes, to-day,« muttered Brogard, sullenly. Then he quietly took Sir Andrew's hat from a chair close by, put it on his own head, tugged at his dirty blouse, and generally tried to express in pantomime that the individual in question wore very fine clothes. »\textit{Sacrré aristo!}« he muttered, »that tall Englishman!«

Marguerite could scarce repress a scream.

»It's Sir Percy right enough,« she murmured, »and not even in disguise!«

She smiled, in the midst of all her anxiety and through her gathering tears, at thought of »the ruling passion strong in death«; of Percy running into the wildest, maddest dangers, with the latest-cut coat upon his back, and the laces of his jabot unruffled.

»Oh! the foolhardiness of it!« she sighed. »Quick, Sir Andrew! ask the man when he went.«

»Ah, yes, my friend,« said Sir Andrew, addressing Brogard, with the same assumption of carelessness, »my lord always wears beautiful clothes; the tall Englishman you saw, was certainly my lady's friend. And he has gone, you say?«

»He went... yes... but he's coming back... here\allowbreak---\allowbreak he ordered supper...«

Sir Andrew put his hand with a quick gesture of warning upon Marguerite's arm; it came none too soon, for the next moment her wild, mad joy would have betrayed her. He was safe and well, was coming back here presently, she would see him in a few moments perhaps... Oh! the wildness of her joy seemed almost more than she could bear.

»Here!« she said to Brogard, who seemed suddenly to have been transformed in her eyes into some heaven-born messenger of bliss. »Here!\allowbreak---\allowbreak did you say the English gentleman was coming back here?«

The heaven-born messenger of bliss spat upon the floor, to express his contempt for all and sundry \textit{aristos}, who chose to haunt the »Chat Gris.«

»Heu!« he muttered, »he ordered supper\allowbreak---\allowbreak he will come back... \textit{Sacrré Anglais!}« he added, by way of protest against all this fuss for a mere Englishman.

»But where is he now?\allowbreak---\allowbreak Do you know?« she asked eagerly, placing her dainty white hand upon the dirty sleeve of his blue blouse.

»He went to get a horse and cart,« said Brogard, laconically, as, with a surly gesture, he shook off from his arm that pretty hand which princes had been proud to kiss.

»At what time did he go?«

But Brogard had evidently had enough of these questionings. He did not think that it was fitting for a citizen\allowbreak---\allowbreak who was the equal of anybody\allowbreak---\allowbreak to be thus catechised by these \textit{sacrrés aristos}, even though they were rich English ones. It was distinctly more fitting to his new-born dignity to be as rude as possible; it was a sure sign of servility to meekly reply to civil questions.

»I don't know,« he said surlily. »I have said enough, \textit{voyons, les aristos}!... He came to-day. He ordered supper. He went out.\allowbreak---\allowbreak He'll come back. \textit{Voilà!}«

And with this parting assertion of his rights as a citizen and a free man, to be as rude as he well pleased, Brogard shuffled out of the room, banging the door after him.