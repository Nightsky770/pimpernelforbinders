%!TeX root=../scarlettop.tex

\chapter{The Schooner}
\lettrine[lines=4]{M}{arguerite's} aching heart stood still. She felt, more than she heard, the men on the watch preparing for the fight. Her senses told her that each, with sword in hand, was crouching, ready for the spring.

The voice came nearer and nearer; in the vast immensity of these lonely cliffs, with the loud murmur of the sea below, it was impossible to say how near, or how far, nor yet from which direction came that cheerful singer, who sang to God to save his King, whilst he himself was in such deadly danger. Faint at first, the voice grew louder and louder; from time to time a small pebble detached itself apparently from beneath the firm tread of the singer, and went rolling down the rocky cliffs to the beach below.

Marguerite as she heard, felt that her very life was slipping away, as if when that voice drew nearer, when that singer became entrapped... She distinctly heard the click of Desgas' gun close to her...

No! no! no! no! Oh, God in heaven! this cannot be! let Armand's blood then be upon her own head! let her be branded as his murderer! let even he, whom she loved, despise and loathe her for this, but God! oh God! save him at any cost!

With a wild shriek, she sprang to her feet, and darted round the rock, against which she had been cowering; she saw the little red gleam through the chinks of the hut; she ran up to it and fell against its wooden walls, which she began to hammer with clenched fists in an almost maniacal frenzy, while she shouted,\longdash


»Armand! Armand! for God's sake fire! your leader is near! he is coming! he is betrayed! Armand! Armand! fire in Heaven's name!«

She was seized and thrown to the ground. She lay there moaning, bruised, not caring, but still half-sobbing, half-shrieking,\longdash


»Percy, my husband, for God's sake fly! Armand! Armand! why don't you fire?«

»One of you stop that woman screaming,« hissed Chauvelin, who hardly could refrain from striking her.

Something was thrown over her face; she could not breathe, and perforce she was silent.

The bold singer, too, had become silent, warned, no doubt, of his impending danger by Marguerite's frantic shrieks. The men had sprung to their feet, there was no need for further silence on their part; the very cliffs echoed the poor, heart-broken woman's screams.

Chauvelin, with a muttered oath, which boded no good to her, who had dared to upset his most cherished plans, had hastily shouted the word of command,\longdash


»Into it, my men, and let no one escape from that hut alive!«

The moon had once more emerged from between the clouds: the darkness on the cliffs had gone, giving place once more to brilliant, silvery light. Some of the soldiers had rushed to the rough, wooden door of the hut, whilst one of them kept guard over Marguerite.

The door was partially open; one of the soldiers pushed it further, but within all was darkness, the charcoal fire only lighting with a dim, red light the furthest corner of the hut. The soldiers paused automatically at the door, like machines waiting for further orders.

Chauvelin, who was prepared for a violent onslaught from within, and for a vigorous resistance from the four fugitives, under cover of the darkness, was for the moment paralyzed with astonishment when he saw the soldiers standing there at attention, like sentries on guard, whilst not a sound proceeded from the hut.

Filled with strange, anxious foreboding, he, too, went to the door of the hut, and peering into the gloom, he asked quickly,\longdash


»What is the meaning of this?«

»I think, citoyen, that there is no one there now,« replied one of the soldiers imperturbably.

»You have not let those four men go?« thundered Chauvelin, menacingly. »I ordered you to let no man escape alive!\allowbreak---\allowbreak Quick, after them all of you! Quick, in every direction!«

The men, obedient as machines, rushed down the rocky incline towards the beach, some going off to right and left, as fast as their feet could carry them.

»You and your men will pay with your lives for this blunder, citoyen sergeant,« said Chauvelin viciously to the sergeant who had been in charge of the men; »and you, too, citoyen,« he added, turning with a snarl to Desgas, »for disobeying my orders.«

»You ordered us to wait, citoyen, until the tall Englishman arrived and joined the four men in the hut. No one came,« said the sergeant sullenly.

»But I ordered you just now, when the woman screamed, to rush in and let no one escape.«

»But, citoyen, the four men who were there before had been gone some time, I think...«

»You think?\allowbreak---\allowbreak You?...« said Chauvelin, almost choking with fury, »and you let them go...«

»You ordered us to wait, citoyen,« protested the sergeant, »and to implicitly obey your commands on pain of death. We waited.«

»I heard the men creep out of the hut, not many minutes after we took cover, and long before the woman screamed,« he added, as Chauvelin seemed still quite speechless with rage.

»Hark!« said Desgas suddenly.

In the distance the sound of repeated firing was heard. Chauvelin tried to peer along the beach below, but as luck would have it, the fitful moon once more hid her light behind a bank of clouds, and he could see nothing.

»One of you go into the hut and strike a light,« he stammered at last.

Stolidly the sergeant obeyed: he went up to the charcoal fire and lit the small lantern he carried in his belt; it was evident that the hut was quite empty.

»Which way did they go?« asked Chauvelin.

»I could not tell, citoyen,« said the sergeant; »they went straight down the cliff first, then disappeared behind some boulders.«

»Hush! what was that?«

All three men listened attentively. In the far, very far distance, could be heard faintly echoing and already dying away, the quick, sharp splash of half a dozen oars. Chauvelin took out his handkerchief and wiped the perspiration from his forehead.

»The schooner's boat!« was all he gasped.

Evidently Armand St~Just and his three companions had managed to creep along the side of the cliffs, whilst the men, like true soldiers of the well-drilled Republican army, had with blind obedience, and in fear of their lives, implicitly obeyed Chauvelin's orders\allowbreak---\allowbreak to wait for the tall Englishman, who was the important capture.

They had no doubt reached one of the creeks which jut far out to sea on this coast at intervals; behind this, the boat of the \textit{Day Dream} must have been on the look-out for them, and they were by now safely on board the British schooner.

As if to confirm this last supposition, the dull boom of a gun was heard from out at sea.

»The schooner, citoyen,« said Desgas, quietly; »she's off.«

It needed all Chauvelin's nerve and presence of mind not to give way to a useless and undignified access of rage. There was no doubt now, that once again, that accursed British head had completely outwitted him. How he had contrived to reach the hut, without being seen by one of the thirty soldiers who guarded the spot, was more than Chauvelin could conceive. That he had done so before the thirty men had arrived on the cliff was, of course, fairly clear, but how he had come over in Reuben Goldstein's cart, all the way from Calais, without being sighted by the various patrols on duty was impossible of explanation. It really seemed as if some potent Fate watched over that daring Scarlet Pimpernel, and his astute enemy almost felt a superstitious shudder pass through him, as he looked round at the towering cliffs, and the loneliness of this outlying coast.

But surely this was reality! and the year of grace 1792: there were no fairies and hobgoblins about. Chauvelin and his thirty men had all heard with their own ears that accursed voice singing »God save the King,« fully twenty minutes \textit{after} they had all taken cover around the hut; by that time the four fugitives must have reached the creek, and got into the boat, and the nearest creek was more than a mile from the hut.

Where had that daring singer got to? Unless Satan himself had lent him wings, he could not have covered that mile on a rocky cliff in the space of two minutes; and only two minutes had elapsed between his song and the sound of the boat's oars away at sea. He must have remained behind, and was even now hiding somewhere about the cliffs; the patrols were still about, he would still be sighted, no doubt. Chauvelin felt hopeful once again.

One or two of the men, who had run after the fugitives, were now slowly working their way up the cliff: one of them reached Chauvelin's side, at the very moment that this hope arose in the astute diplomatist's heart.

»We were too late, citoyen,« the soldier said, »we reached the beach just before the moon was hidden by that bank of clouds. The boat had undoubtedly been on the look-out behind that first creek, a mile off, but she had shoved off some time ago, when we got to the beach, and was already some way out to sea. We fired after her, but of course, it was no good. She was making straight and quickly for the schooner. We saw her very clearly in the moonlight.«

»Yes,« said Chauvelin, with eager impatience, »she had shoved off some time ago, you said, and the nearest creek is a mile further on.«

»Yes, citoyen! I ran all the way, straight to the beach, though I guessed the boat would have waited somewhere near the creek, as the tide would reach there earliest. The boat must have shoved off some minutes before the woman began to scream.«

Some minutes before the woman began to scream! Then Chauvelin's hopes had not deceived him. The Scarlet Pimpernel may have contrived to send the fugitives on ahead by the boat, but he himself had not had time to reach it; he was still on shore, and all the roads were well patrolled. At any rate, all was not yet lost, and would not be, whilst that impudent Britisher was still on French soil.

»Bring the light in here!« he commanded eagerly, as he once more entered the hut.

The sergeant brought his lantern, and together the two men explored the little place: with a rapid glance Chauvelin noted its contents: the cauldron placed close under an aperture in the wall, and containing the last few dying embers of burned charcoal, a couple of stools, overturned as if in the haste of sudden departure, then the fisherman's tools and his nets lying in one corner, and beside them, something small and white.

»Pick that up,« said Chauvelin to the sergeant, pointing to this white scrap, »and bring it to me.«

It was a crumpled piece of paper, evidently forgotten there by the fugitives, in their hurry to get away. The sergeant, much awed by the citoyen's obvious rage and impatience, picked the paper up and handed it respectfully to Chauvelin.

»Read it, sergeant,« said the latter curtly.

»It is almost illegible, citoyen... a fearful scrawl...«

»I ordered you to read it,« repeated Chauvelin, viciously.

The sergeant, by the light of his lantern, began deciphering the few hastily scrawled words.

»I cannot quite reach you, without risking your lives and endangering the success of your rescue. When you receive this, wait two minutes, then creep out of the hut one by one, turn to your left sharply, and creep cautiously down the cliff; keep to the left all the time, till you reach the first rock, which you see jutting far out to sea\allowbreak---\allowbreak behind it in the creek the boat is on the look-out for you\allowbreak---\allowbreak give a long, sharp whistle\allowbreak---\allowbreak she will come up\allowbreak---\allowbreak get into her\allowbreak---\allowbreak my men will row you to the schooner, and thence to England and safety\allowbreak---\allowbreak once on board the \textit{Day Dream} send the boat back for me, tell my men that I shall be at the creek, which is in a direct line opposite the »Chat Gris« near Calais. They know it. I shall be there as soon as possible\allowbreak---\allowbreak they must wait for me at a safe distance out at sea, till they hear the usual signal. Do not delay\allowbreak---\allowbreak and obey these instructions implicitly.«

»Then there is the signature, citoyen,« added the sergeant, as he handed the paper back to Chauvelin.

But the latter had not waited an instant. One phrase of the momentous scrawl had caught his ear. »I shall be at the creek which is in a direct line opposite the »Chat Gris« near Calais«: that phrase might yet mean victory for him.

»Which of you knows this coast well?« he shouted to his men who now one by one had all returned from their fruitless run, and were all assembled once more round the hut.

»I do, citoyen,« said one of them, »I was born in Calais, and know every stone of these cliffs.«

»There is a creek in a direct line from the »Chat Gris«?«

»There is, citoyen. I know it well.«

»The Englishman is hoping to reach that creek. He does \textit{not} know every stone of these cliffs, he may go there by the longest way round, and in any case he will proceed cautiously for fear of the patrols. At any rate, there is a chance to get him yet. A thousand francs to each man who gets to that creek before that long-legged Englishman.«

»I know a short cut across the cliffs,« said the soldier, and with an enthusiastic shout, he rushed forward, followed closely by his comrades.

Within a few minutes their running footsteps had died away in the distance. Chauvelin listened to them for a moment; the promise of the reward was lending spurs to the soldiers of the Republic. The gleam of hate and anticipated triumph was once more apparent on his face.

Close to him Desgas still stood mute and impassive, waiting for further orders, whilst two soldiers were kneeling beside the prostrate form of Marguerite. Chauvelin gave his secretary a vicious look. His well-laid plan had failed, its sequel was problematical; there was still a great chance now that the Scarlet Pimpernel might yet escape, and Chauvelin, with that unreasoning fury, which sometimes assails a strong nature, was longing to vent his rage on somebody.

The soldiers were holding Marguerite pinioned to the ground, though she, poor soul, was not making the faintest struggle. Overwrought nature had at last peremptorily asserted herself, and she lay there in a dead swoon: her eyes circled by deep purple lines, that told of long, sleepless nights, her hair matted and damp round her forehead, her lips parted in a sharp curve that spoke of physical pain.

The cleverest woman in Europe, the elegant and fashionable Lady Blakeney, who had dazzled London society with her beauty, her wit and her extravagances, presented a very pathetic picture of tired-out, suffering womanhood, which would have appealed to any, but the hard, vengeful heart of her baffled enemy.

»It is no use mounting guard over a woman who is half dead,« he said spitefully to the soldiers, »when you have allowed five men who were very much alive to escape.«

Obediently the soldiers rose to their feet.

»You'd better try and find that footpath again for me, and that broken-down cart we left on the road.«

Then suddenly a bright idea seemed to strike him.

»Ah! by-the-bye! where is the Jew?«

»Close by here, citoyen,« said Desgas; »I gagged him and tied his legs together as you commanded.«

From the immediate vicinity, a plaintive moan reached Chauvelin's ears. He followed his secretary, who led the way to the other side of the hut, where, fallen into an absolute heap of dejection, with his legs tightly pinioned together and his mouth gagged, lay the unfortunate descendant of Israel.

His face in the silvery light of the moon looked positively ghastly with terror: his eyes were wide open and almost glassy, and his whole body was trembling, as if with ague, while a piteous wail escaped his bloodless lips. The rope which had originally been wound round his shoulders and arms had evidently given way, for it lay in a tangle about his body, but he seemed quite unconscious of this, for he had not made the slightest attempt to move from the place where Desgas had originally put him: like a terrified chicken which looks upon a line of white chalk, drawn on a table, as on a string which paralyzes its movements.

»Bring the cowardly brute here,« commanded Chauvelin.

He certainly felt exceedingly vicious, and since he had no reasonable grounds for venting his ill-humour on the soldiers who had but too punctually obeyed his orders, he felt that the son of the despised race would prove an excellent butt. With true French contempt of the Jew, which has survived the lapse of centuries even to this day, he would not go too near him, but said with biting sarcasm, as the wretched old man was brought in full light of the moon by the two soldiers,\longdash


»I suppose now, that being a Jew, you have a good memory for bargains?«

»Answer!« he again commanded, as the Jew with trembling lips seemed too frightened to speak.

»Yes, your Honour,« stammered the poor wretch.

»You remember, then, the one you and I made together in Calais, when you undertook to overtake Reuben Goldstein, his nag and my friend the tall stranger? Eh?«

»B... b... but... your Honour...«

»There is no »but.« I said, do you remember?«

»Y... y... y... yes... your Honour!«

»What was the bargain?«

There was dead silence. The unfortunate man looked round at the great cliffs, the moon above, the stolid faces of the soldiers, and even at the poor, prostrate, inanimate woman close by, but said nothing.

»Will you speak?« thundered Chauvelin, menacingly.

He did try, poor wretch, but, obviously, he could not. There was no doubt, however, that he knew what to expect from the stern man before him.

»Your Honour...« he ventured imploringly.

»Since your terror seems to have paralyzed your tongue,« said Chauvelin, sarcastically, »I must needs refresh your memory. It was agreed between us, that if we overtook my friend the tall stranger, before he reached this place, you were to have ten pieces of gold.«

A low moan escaped from the Jew's trembling lips.

»But,« added Chauvelin, with slow emphasis, »if you deceived me in your promise, you were to have a sound beating, one that would teach you not to tell lies.«

»I did not, your Honour; I swear it by Abraham...«

»And by all the other patriarchs, I know. Unfortunately, they are still in Hades, I believe, according to your creed, and cannot help you much in your present trouble. Now, you did not fulfil your share of the bargain, but I am ready to fulfil mine. Here,« he added, turning to the soldiers, »the buckle-end of your two belts to this confounded Jew.«

As the soldiers obediently unbuckled their heavy leather belts, the Jew set up a howl that surely would have been enough to bring all the patriarchs out of Hades and elsewhere, to defend their descendant from the brutality of this French official.

»I think I can rely on you, citoyen soldiers,« laughed Chauvelin, maliciously, »to give this old liar the best and soundest beating he has ever experienced. But don't kill him,« he added drily.

»We will obey, citoyen,« replied the soldiers as imperturbably as ever.

He did not wait to see his orders carried out: he knew that he could trust these soldiers\allowbreak---\allowbreak who were still smarting under his rebuke\allowbreak---\allowbreak not to mince matters, when given a free hand to belabour a third party.

»When that lumbering coward has had his punishment,« he said to Desgas, »the men can guide us as far as the cart, and one of them can drive us in it back to Calais. The Jew and the woman can look after each other,« he added roughly, »until we can send somebody for them in the morning. They can't run away very far, in their present condition, and we cannot be troubled with them just now.«

Chauvelin had not given up all hope. His men, he knew, were spurred on by the hope of the reward. That enigmatic and audacious Scarlet Pimpernel, alone and with thirty men at his heels, could not reasonably be expected to escape a second time.

But he felt less sure now: the Englishman's audacity had baffled him once, whilst the wooden-headed stupidity of the soldiers, and the interference of a woman had turned his hand, which held all the trumps, into a losing one. If Marguerite had not taken up his time, if the soldiers had had a grain of intelligence, if... it was a long »if,« and Chauvelin stood for a moment quite still, and enrolled thirty odd people in one long, overwhelming anathema. Nature, poetic, silent, balmy, the bright moon, the calm, silvery sea spoke of beauty and of rest, and Chauvelin cursed nature, cursed man and woman, and, above all, he cursed all long-legged, meddlesome British enigmas with one gigantic curse.

The howls of the Jew behind him, undergoing his punishment, sent a balm through his heart, overburdened as it was with revengeful malice. He smiled. It eased his mind to think that some human being at least was, like himself, not altogether at peace with mankind.

He turned and took a last look at the lonely bit of coast, where stood the wooden hut, now bathed in moonlight, the scene of the greatest discomfiture ever experienced by a leading member of the Committee of Public Safety.

Against a rock, on a hard bed of stone, lay the unconscious figure of Marguerite Blakeney, while some few paces further on, the unfortunate Jew was receiving on his broad back the blows of two stout leather belts, wielded by the stolid arms of two sturdy soldiers of the Republic. The howls of Benjamin Rosenbaum were fit to make the dead rise from their graves. They must have wakened all the gulls from sleep, and made them look down with great interest at the doings of the lords of the creation.

»That will do,« commanded Chauvelin, as the Jew's moans became more feeble, and the poor wretch seemed to have fainted away, »we don't want to kill him.«

Obediently the soldiers buckled on their belts, one of them viciously kicking the Jew to one side.

»Leave him there,« said Chauvelin, »and lead the way now quickly to the cart. I'll follow.«

He walked up to where Marguerite lay, and looked down into her face. She had evidently recovered consciousness, and was making feeble efforts to raise herself. Her large, blue eyes were looking at the moonlit scene round her with a scared and terrified look; they rested with a mixture of horror and pity on the Jew, whose luckless fate and wild howls had been the first signs that struck her, with her returning senses; then she caught sight of Chauvelin, in his neat, dark clothes, which seemed hardly crumpled after the stirring events of the last few hours. He was smiling sarcastically, and his pale eyes peered down at her with a look of intense malice.

With mock gallantry, he stooped and raised her icy-cold hand to his lips, which sent a thrill of indescribable loathing through Marguerite's weary frame.

»I much regret, fair lady,« he said in his most suave tones, »that circumstances, over which I have no control, compel me to leave you here for the moment. But I go away, secure in the knowledge that I do not leave you unprotected. Our friend Benjamin here, though a trifle the worse for wear at the present moment, will prove a gallant defender of your fair person, I have no doubt. At dawn I will send an escort for you; until then, I feel sure that you will find him devoted, though perhaps a trifle slow.«

Marguerite only had the strength to turn her head away. Her heart was broken with cruel anguish. One awful thought had returned to her mind, together with gathering consciousness: »What had become of Percy?\allowbreak---\allowbreak What of Armand?«

She knew nothing of what had happened after she heard the cheerful song, »God save the King,« which she believed to be the signal of death.

»I, myself,« concluded Chauvelin, »must now very reluctantly leave you. \textit{Au revoir}, fair lady. We meet, I hope, soon in London. Shall I see you at the Prince of Wales' garden party?\allowbreak---\allowbreak No?\allowbreak---\allowbreak Ah, well, \textit{au revoir}!\allowbreak---\allowbreak Remember me, I pray, to Sir Percy Blakeney.«

And, with a last ironical smile and bow, he once more kissed her hand, and disappeared down the footpath in the wake of the soldiers, and followed by the imperturbable Desgas.