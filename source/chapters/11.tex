%!TeX root=../scarlettop.tex

\chapter{Lord Grenville's Ball}
\lettrine[lines=4]{T}{he} historic ball given by the then Secretary of State for Foreign Affairs—Lord Grenville—was the most brilliant function of the year. Though the autumn season had only just begun, everybody who was anybody had contrived to be in London in time to be present there, and to shine at this ball, to the best of his or her respective ability.

His Royal Highness the Prince of Wales had promised to be present. He was coming on presently from the opera. Lord Grenville himself had listened to the two first acts of \textit{Orpheus}, before preparing to receive his guests. At ten o'clock—an unusually late hour in those days—the grand rooms of the Foreign Office, exquisitely decorated with exotic palms and flowers, were filled to overflowing. One room had been set apart for dancing, and the dainty strains of the minuet made a soft accompaniment to the gay chatter, the merry laughter of the numerous and brilliant company.

In a smaller chamber, facing the top of the fine stairway, the distinguished host stood ready to receive his guests. Distinguished men, beautiful women, notabilities from every European country had already filed past him, had exchanged the elaborate bows and curtsies with him, which the extravagant fashion of the time demanded, and then, laughing and talking, had dispersed in the ball, reception, and card rooms beyond.

Not far from Lord Grenville's elbow, leaning against one of the console tables, Chauvelin, in his irreproachable black costume, was taking a quiet survey of the brilliant throng. He noted that Sir Percy and Lady Blakeney had not yet arrived, and his keen, pale eyes glanced quickly towards the door every time a newcomer appeared.

He stood somewhat isolated: the envoy of the Revolutionary Government of France was not likely to be very popular in England, at a time when the news of the awful September massacres, and of the Reign of Terror and Anarchy, had just begun to filtrate across the Channel.

In his official capacity he had been received courteously by his English colleagues: Mr Pitt had shaken him by the hand; Lord Grenville had entertained him more than once; but the more intimate circles of London society ignored him altogether; the women openly turned their backs upon him; the men who held no official position refused to shake his hand.

But Chauvelin was not the man to trouble himself about these social amenities, which he called mere incidents in his diplomatic career. He was blindly enthusiastic for the revolutionary cause, he despised all social inequalities, and he had a burning love for his own country: these three sentiments made him supremely indifferent to the snubs he received in this fog-ridden, loyalist, old-fashioned England.

But, above all, Chauvelin had a purpose at heart. He firmly believed that the French aristocrat was the most bitter enemy of France; he would have wished to see every one of them annihilated: he was one of those who, during this awful Reign of Terror, had been the first to utter the historic and ferocious desire »that aristocrats might have but one head between them, so that it might be cut off with a single stroke of the guillotine.« And thus he looked upon every French aristocrat, who had succeeded in escaping from France, as so much prey of which the guillotine had been unwarrantably cheated. There is no doubt that those royalist \textit{émigrés}, once they had managed to cross the frontier, did their very best to stir up foreign indignation against France. Plots without end were hatched in England, in Belgium, in Holland, to try and induce some great power to send troops into revolutionary Paris, to free King Louis, and to summarily hang the bloodthirsty leaders of that monster republic.

Small wonder, therefore, that the romantic and mysterious personality of the Scarlet Pimpernel was a source of bitter hatred to Chauvelin. He and the few young jackanapes under his command, well furnished with money, armed with boundless daring, and acute cunning, had succeeded in rescuing hundreds of aristocrats from France. Nine-tenths of the \textit{émigrés}, who were \textit{fêted} at the English court, owed their safety to that man and to his league.

Chauvelin had sworn to his colleagues in Paris that he would discover the identity of that meddlesome Englishman, entice him over to France, and then\textellipsis \allowbreak  Chauvelin drew a deep breath of satisfaction at the very thought of seeing that enigmatic head falling under the knife of the guillotine, as easily as that of any other man.

Suddenly there was a great stir on the handsome staircase, all conversation stopped for a moment as the major-domo's voice outside announced,\longdash


»His Royal Highness the Prince of Wales and suite, Sir Percy Blakeney, Lady Blakeney.«

Lord Grenville went quickly to the door to receive his exalted guest.

The Prince of Wales, dressed in a magnificent court suit of salmon-coloured velvet richly embroidered with gold, entered with Marguerite Blakeney on his arm; and on his left Sir Percy, in gorgeous shimmering cream satin, cut in the extravagant »Incroyable« style, his fair hair free from powder, priceless lace at his neck and wrists, and the flat \textit{chapeau-bras} under his arm.

After the few conventional words of deferential greeting, Lord Grenville said to his royal guest,\longdash


»Will your Highness permit me to introduce M. Chauvelin, the accredited agent of the French Government?«

Chauvelin, immediately the Prince entered, had stepped forward, expecting this introduction. He bowed very low, whilst the Prince returned his salute with a curt nod of the head.

»Monsieur,« said His Royal Highness coldly, »we will try to forget the government that sent you, and look upon you merely as our guest—a private gentleman from France. As such you are welcome, Monsieur.«

»Monseigneur,« rejoined Chauvelin, bowing once again. »Madame,« he added, bowing ceremoniously before Marguerite.

»Ah! my little Chauvelin!« she said with unconcerned gaiety, and extending her tiny hand to him. »Monsieur and I are old friends, your Royal Highness.«

»Ah, then,« said the Prince, this time very graciously, »you are doubly welcome, Monsieur.«

»There is someone else I would crave permission to present to your Royal Highness,« here interposed Lord Grenville.

»Ah! who is it?« asked the Prince.

»Madame la Comtesse de Tournay de Basserive and her family, who have but recently come from France.«

»By all means!—They are among the lucky ones then!«

Lord Grenville turned in search of the Comtesse, who sat at the further end of the room.

»Lud love me!« whispered His Royal Highness to Marguerite, as soon as he had caught sight of the rigid figure of the old lady; »Lud love me! she looks very virtuous and very melancholy.«

»Faith, your Royal Highness,« she rejoined with a smile, »virtue is like precious odours, most fragrant when it is crushed.«

»Virtue, alas!« sighed the Prince, »is mostly unbecoming to your charming sex, Madame.«

»Madame la Comtesse de Tournay de Basserive,« said Lord Grenville, introducing the lady.

»This is a pleasure, Madame; my royal father, as you know, is ever glad to welcome those of your compatriots whom France has driven from her shores.«

»Your Royal Highness is ever gracious,« replied the Comtesse with becoming dignity. Then, indicating her daughter, who stood timidly by her side: »My daughter Suzanne, Monseigneur,« she said.

»Ah! charming!—charming!« said the Prince, »and now allow me, Comtesse, to introduce to you, Lady Blakeney, who honours us with her friendship. You and she will have much to say to one another, I vow. Every compatriot of Lady Blakeney's is doubly welcome for her sake\textellipsis \allowbreak  her friends are our friends\textellipsis \allowbreak  her enemies, the enemies of England.«

Marguerite's blue eyes had twinkled with merriment at this gracious speech from her exalted friend. The Comtesse de Tournay, who lately had so flagrantly insulted her, was here receiving a public lesson, at which Marguerite could not help but rejoice. But the Comtesse, for whom respect of royalty amounted almost to a religion, was too well-schooled in courtly etiquette to show the slightest sign of embarrassment, as the two ladies curtsied ceremoniously to one another.

»His Royal Highness is ever gracious, Madame,« said Marguerite, demurely, and with a wealth of mischief in her twinkling blue eyes, »but here there is no need for his kind mediation\textellipsis \allowbreak  Your amiable reception of me at our last meeting still dwells pleasantly in my memory.«

»We poor exiles, Madame,« rejoined the Comtesse, frigidly, »show our gratitude to England by devotion to the wishes of Monseigneur.«

»Madame!« said Marguerite, with another ceremonious curtsey.

»Madame,« responded the Comtesse with equal dignity.

The Prince in the meanwhile was saying a few gracious words to the young Vicomte.

»I am happy to know you, Monsieur le Vicomte,« he said. »I knew your father well when he was ambassador in London.«

»Ah, Monseigneur!« replied the Vicomte, »I was a leetle boy then\textellipsis \allowbreak  and now I owe the honour of this meeting to our protector, the Scarlet Pimpernel.«

»Hush!« said the Prince, earnestly and quickly, as he indicated Chauvelin, who had stood a little on one side throughout the whole of this little scene, watching Marguerite and the Comtesse with an amused, sarcastic little smile around his thin lips.

»Nay, Monseigneur,« he said now, as if in direct response to the Prince's challenge, »pray do not check this gentleman's display of gratitude; the name of that interesting red flower is well known to me—and to France.«

The Prince looked at him keenly for a moment or two.

»Faith, then, Monsieur,« he said, »perhaps you know more about our national hero than we do ourselves\textellipsis \allowbreak  perchance you know who he is\textellipsis \allowbreak  See!« he added, turning to the groups round the room, »the ladies hang upon your lips\textellipsis \allowbreak  you would render yourself popular among the fair sex if you were to gratify their curiosity.«

»Ah, Monseigneur,« said Chauvelin, significantly, »rumour has it in France that your Highness could—an you would—give the truest account of that enigmatical wayside flower.«

He looked quickly and keenly at Marguerite as he spoke; but she betrayed no emotion, and her eyes met his quite fearlessly.

»Nay, man,« replied the Prince, »my lips are sealed! and the members of the league jealously guard the secret of their chief\textellipsis \allowbreak  so his fair adorers have to be content with worshipping a shadow. Here in England, Monsieur,« he added, with wonderful charm and dignity, »we but name the Scarlet Pimpernel, and every fair cheek is suffused with a blush of enthusiasm. None have seen him save his faithful lieutenants. We know not if he be tall or short, fair or dark, handsome or ill-formed; but we know that he is the bravest gentleman in all the world, and we all feel a little proud, Monsieur, when we remember that he is an Englishman.«

»Ah, Monsieur Chauvelin,« added Marguerite, looking almost with defiance across at the placid, sphinx-like face of the Frenchman, »His Royal Highness should add that we ladies think of him as of a hero of old\textellipsis \allowbreak  we worship him\textellipsis \allowbreak  we wear his badge\textellipsis \allowbreak  we tremble for him when he is in danger, and exult with him in the hour of his victory.«

Chauvelin did no more than bow placidly both to the Prince and to Marguerite; he felt that both speeches were intended—each in their way—to convey contempt or defiance. The pleasure-loving, idle Prince he despised; the beautiful woman, who in her golden hair wore a spray of small red flowers composed of rubies and diamonds—her he held in the hollow of his hand: he could afford to remain silent and to await events.

A long, jovial, inane laugh broke the sudden silence which had fallen over everyone.

»And we poor husbands,« came in slow, affected accents from gorgeous Sir Percy, »we have to stand by\textellipsis \allowbreak  while they worship a demmed shadow.«

Everyone laughed—the Prince more loudly than anyone. The tension of subdued excitement was relieved, and the next moment everyone was laughing and chatting merrily as the gay crowd broke up and dispersed in the adjoining rooms.