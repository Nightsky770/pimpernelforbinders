%!TeX root=../scarlettop.tex

\chapter{The Eagle and the Fox}
\lettrine[lines=4]{M}{arguerite's} breath stopped short; she seemed to feel her very life standing still momentarily whilst she listened to that voice and to that song. In the singer she had recognised her husband. Chauvelin, too, had heard it, for he darted a quick glance towards the door, then hurriedly took up his broad-brimmed hat and clapped it over his head.

The voice drew nearer; for one brief second the wild desire seized Marguerite to rush down the steps and fly across the room, to stop that song at any cost, to beg the cheerful singer to fly\allowbreak---\allowbreak fly for his life, before it be too late. She checked the impulse just in time. Chauvelin would stop her before she reached the door, and, moreover, she had no idea if he had any soldiers posted within his call. Her impetuous act might prove the death-signal of the man she would have died to save.

\blockquote{
\enquote{Long to reign over us,\\
God save the King!}
}

sang the voice more lustily than ever. The next moment the door was thrown open and there was dead silence for a second or so.

Marguerite could not see the door: she held her breath, trying to imagine what was happening.

Percy Blakeney on entering had, of course, at once caught sight of the \textit{curé} at the table; his hesitation lasted less than five seconds, the next moment Marguerite saw his tall figure crossing the room, whilst he called in a loud, cheerful voice,\longdash


»Hello, there! no one about? Where's that fool Brogard?«

He wore the magnificent coat and riding-suit which he had on when Marguerite last saw him at Richmond, so many hours ago. As usual, his get-up was absolutely irreproachable, the fine Mechlin lace at his neck and wrists was immaculate in its gossamer daintiness, his hands looked slender and white, his fair hair was carefully brushed, and he carried his eye-glass with his usual affected gesture. In fact, at this moment, Sir Percy Blakeney, Bart., might have been on his way to a garden-party at the Prince of Wales', instead of deliberately, cold-bloodedly running his head in a trap, set for him by his deadliest enemy.

He stood for a moment in the middle of the room, whilst Marguerite, absolutely paralysed with horror, seemed unable even to breathe.

Every moment she expected that Chauvelin would give a signal, that the place would fill with soldiers, that she would rush down and help Percy to sell his life dearly. As he stood there, suavely unconscious, she very nearly screamed out to him,\longdash


»Fly, Percy!\allowbreak---\allowbreak 'tis your deadly enemy!\allowbreak---\allowbreak fly before it be too late!«

But she had not time even to do that, for the next moment Blakeney quietly walked to the table, and, jovially clapping the \textit{curé} on the back, said in his own drawly, affected way,\longdash


»Odd's fish!... er... M. Chauvelin... I vow I never thought of meeting you here.«

Chauvelin, who had been in the very act of conveying soup to his mouth, fairly choked. His thin face became absolutely purple, and a violent fit of coughing saved this cunning representative of France from betraying the most boundless surprise he had ever experienced. There was no doubt that this bold move on the part of the enemy had been wholly unexpected, as far as he was concerned: and the daring impudence of it completely nonplussed him for the moment.

Obviously he had not taken the precaution of having the inn surrounded with soldiers. Blakeney had evidently guessed that much, and no doubt his resourceful brain had already formed some plan by which he could turn this unexpected interview to account.

Marguerite up in the loft had not moved. She had made a solemn promise to Sir Andrew not to speak to her husband before strangers, and she had sufficient self-control not to throw herself unreasoningly and impulsively across his plans. To sit still and watch these two men together was a terrible trial of fortitude. Marguerite had heard Chauvelin give the orders for the patrolling of all the roads. She knew that if Percy now left the »Chat Gris«\allowbreak---\allowbreak in whichever direction he happened to go\allowbreak---\allowbreak he could not go far without being sighted by some of Captain Jutley's men on patrol. On the other hand, if he stayed, then Desgas would have time to come back with the half-dozen men Chauvelin had specially ordered.

The trap was closing in, and Marguerite could do nothing but watch and wonder. The two men looked such a strange contrast, and of the two it was Chauvelin who exhibited a slight touch of fear. Marguerite knew him well enough to guess what was passing in his mind. He had no fear for his own person, although he certainly was alone in a lonely inn with a man who was powerfully built, and who was daring and reckless beyond the bounds of probability. She knew that Chauvelin would willingly have braved perilous encounters for the sake of the cause he had at heart, but what he did fear was that this impudent Englishman would, by knocking him down, double his own chances of escape; his underlings might not succeed so well in capturing the Scarlet Pimpernel, when not directed by the cunning hand and the shrewd brain, which had deadly hate for an incentive.

Evidently, however, the representative of the French Government had nothing to fear for the moment, at the hands of his powerful adversary. Blakeney, with his most inane laugh and pleasant good-nature, was solemnly patting him on the back.

»I am so demmed sorry...« he was saying cheerfully, »so very sorry... I seem to have upset you... eating soup, too... nasty, awkward thing, soup... er... Begad!\allowbreak---\allowbreak a friend of mine died once... er... choked... just like you... with a spoonful of soup.«

And he smiled shyly, good-humouredly, down at Chauvelin.

»Odd's life!« he continued, as soon as the latter had somewhat recovered himself, »beastly hole this... ain't it now? La! you don't mind?« he added, apologetically, as he sat down on a chair close to the table and drew the soup tureen towards him. »That fool Brogard seems to be asleep or something.«

There was a second plate on the table, and he calmly helped himself to soup, then poured himself out a glass of wine.

For a moment Marguerite wondered what Chauvelin would do. His disguise was so good that perhaps he meant, on recovering himself, to deny his identity: but Chauvelin was too astute to make such an obviously false and childish move, and already he too had stretched out his hand and said pleasantly,\longdash


»I am indeed charmed to see you, Sir Percy. You must excuse me\allowbreak---\allowbreak h'm\allowbreak---\allowbreak I thought you the other side of the Channel. Sudden surprise almost took my breath away.«

»La!« said Sir Percy, with a good-humoured grin, »it did that quite, didn't it\allowbreak---\allowbreak er\allowbreak---\allowbreak M.\allowbreak---\allowbreak er\allowbreak---\allowbreak Chaubertin?«

»Pardon me\allowbreak---\allowbreak Chauvelin.«

»I beg pardon\allowbreak---\allowbreak a thousand times. Yes\allowbreak---\allowbreak Chauvelin of course... Er... I never could cotton to foreign names...«

He was calmly eating his soup, laughing with pleasant good-humour, as if he had come all the way to Calais for the express purpose of enjoying supper at this filthy inn, in the company of his arch-enemy.

For the moment Marguerite wondered why Percy did not knock the little Frenchman down then and there\allowbreak---\allowbreak and no doubt something of the sort must have darted through his mind, for every now and then his lazy eyes seemed to flash ominously, as they rested on the slight figure of Chauvelin, who had now quite recovered himself and was also calmly eating his soup.

But the keen brain, which had planned and carried through so many daring plots, was too far-seeing to take unnecessary risks. This place, after all, might be infested with spies; the innkeeper might be in Chauvelin's pay. One call on Chauvelin's part might bring twenty men about Blakeney's ears for aught he knew, and he might be caught and trapped before he could help or, at least, warn the fugitives. This he would not risk; he meant to help the others, to get \textit{them} safely away; for he had pledged his word to them, and his word he \textit{would} keep. And whilst he ate and chatted, he thought and planned, whilst, up in the loft, the poor, anxious woman racked her brain as to what she should do, and endured agonies of longing to rush down to him, yet not daring to move for fear of upsetting his plans.

»I didn't know,« Blakeney was saying jovially, »that you... er... were in holy orders.«

»I... er... hem...« stammered Chauvelin. The calm impudence of his antagonist had evidently thrown him off his usual balance.

»But, la! I should have known you anywhere,« continued Sir Percy, placidly, as he poured himself out another glass of wine, »although the wig and hat have changed you a bit.«

»Do you think so?«

»Lud! they alter a man so... but... begad! I hope you don't mind my having made the remark?... Demmed bad form making remarks... I hope you don't mind?«

»No, no, not at all\allowbreak---\allowbreak hem! I hope Lady Blakeney is well,« said Chauvelin, hurriedly changing the topic of conversation.

Blakeney, with much deliberation, finished his plate of soup, drank his glass of wine, and, momentarily, it seemed to Marguerite as if he glanced quickly all round the room.

»Quite well, thank you,« he said at last, drily. There was a pause, during which Marguerite could watch these two antagonists who, evidently in their minds, were measuring themselves against one another. She could see Percy almost full face where he sat at the table not ten yards from where she herself was crouching, puzzled, not knowing what to do, or what she should think. She had quite controlled her impulse by now of rushing down and disclosing herself to her husband. A man capable of acting a part, in the way he was doing at the present moment, did not need a woman's word to warn him to be cautious.

Marguerite indulged in the luxury, dear to every tender woman's heart, of looking at the man she loved. She looked through the tattered curtain, across at the handsome face of her husband, in whose lazy blue eyes, and behind whose inane smile, she could now so plainly see the strength, energy, and resourcefulness which had caused the Scarlet Pimpernel to be reverenced and trusted by his followers. »There are nineteen of us ready to lay down our lives for your husband, Lady Blakeney,« Sir Andrew had said to her; and as she looked at the forehead, low, but square and broad, the eyes, blue, yet deep-set and intense, the whole aspect of the man, of indomitable energy, hiding, behind a perfectly acted comedy, his almost superhuman strength of will and marvellous ingenuity, she understood the fascination which he exercised over his followers, for had he not also cast his spells over her heart and her imagination?

Chauvelin, who was trying to conceal his impatience beneath his usual urbane manner, took a quick look at his watch. Desgas should not be long: another two or three minutes, and this impudent Englishman would be secure in the keeping of half a dozen of Captain Jutley's most trusted men.

»You are on your way to Paris, Sir Percy?« he asked carelessly.

»Odd's life, no,« replied Blakeney, with a laugh. »Only as far as Lille\allowbreak---\allowbreak not Paris for me... beastly uncomfortable place Paris, just now... eh, Monsieur Chaubertin... beg pardon... Chauvelin!«

»Not for an English gentleman like yourself, Sir Percy,« rejoined Chauvelin, sarcastically, »who takes no interest in the conflict that is raging there.«

»a! you see it's no business of mine, and our demmed government is all on your side of the business. Old Pitt daren't say »Bo« to a goose. You are in a hurry, sir,« he added, as Chauvelin once again took out his watch; »an appointment, perhaps... I pray you take no heed of me... My time's my own.«

He rose from the table and dragged a chair to the hearth. Once more Marguerite was terribly tempted to go to him, for time was getting on; Desgas might be back at any moment with his men. Percy did not know that and... oh! how horrible it all was\allowbreak---\allowbreak and how helpless she felt.

»I am in no hurry,« continued Percy, pleasantly, »but, la! I don't want to spend any more time than I can help in this God-forsaken hole! But, begad! sir,« he added, as Chauvelin had surreptitiously looked at his watch for the third time, »that watch of yours won't go any faster for all the looking you give it. You are expecting a friend, maybe?«

»Aye\allowbreak---\allowbreak a friend!«

»Not a lady\allowbreak---\allowbreak I trust, Monsieur l'Abbé,« laughed Blakeney; »surely the holy Church does not allow?... eh?... what! But, I say, come by the fire... it's getting demmed cold.«

He kicked the fire with the heel of his boot, making the logs blaze in the old hearth. He seemed in no hurry to go, and apparently was quite unconscious of his immediate danger. He dragged another chair to the fire, and Chauvelin, whose impatience was by now quite beyond control, sat down beside the hearth, in such a way as to command a view of the door. Desgas had been gone nearly a quarter of an hour. It was quite plain to Marguerite's aching senses that as soon as he arrived, Chauvelin would abandon all his other plans with regard to the fugitives, and capture this impudent Scarlet Pimpernel at once.

»Hey, M. Chauvelin,« the latter was saying airily, »tell me, I pray you, is your friend pretty? Demmed smart these little French women sometimes\allowbreak---\allowbreak what? But I protest I need not ask,« he added, as he carelessly strode back towards the supper-table. »In matters of taste the Church has never been backward... Eh?«

But Chauvelin was not listening. His every faculty was now concentrated on that door through which presently Desgas would enter. Marguerite's thoughts, too, were centred there, for her ears had suddenly caught, through the stillness of the night, the sound of numerous and measured treads some distance away.

It was Desgas and his men. Another three minutes and they would be here! Another three minutes and the awful thing would have occurred: the brave eagle would have fallen in the ferret's trap! She would have moved now and screamed, but she dared not; for whilst she heard the soldiers approaching, she was looking at Percy and watching his every movement. He was standing by the table whereon the remnants of the supper, plates, glasses, spoons, salt and pepper-pots were scattered pell-mell. His back was turned to Chauvelin and he was still prattling along in his own affected and inane way, but from his pocket he had taken his snuff-box, and quickly and suddenly he emptied the contents of the pepper-pot into it.

Then he again turned with an inane laugh to Chauvelin,\longdash


»Eh? Did you speak, sir?«

Chauvelin had been too intent on listening to the sound of those approaching footsteps, to notice what his cunning adversary had been doing. He now pulled himself together, trying to look unconcerned in the very midst of his anticipated triumph.

»No,« he said presently, »that is\allowbreak---\allowbreak as you were saying, Sir Percy\allowbreak---\allowbreak ?«

»I was saying,« said Blakeney, going up to Chauvelin, by the fire, »that the Jew in Piccadilly has sold me better snuff this time than I have ever tasted. Will you honour me, Monsieur l'Abbé?«

He stood close to Chauvelin in his own careless, \textit{débonnaire} way, holding out his snuff-box to his arch-enemy.

Chauvelin, who, as he told Marguerite once, had seen a trick or two in his day, had never dreamed of this one. With one ear fixed on those fast-approaching footsteps, one eye turned to that door where Desgas and his men would presently appear, lulled into false security by the impudent Englishman's airy manner, he never even remotely guessed the trick which was being played upon him.

He took a pinch of snuff.

Only he, who has ever by accident sniffed vigorously a dose of pepper, can have the faintest conception of the hopeless condition in which such a sniff would reduce any human being.

Chauvelin felt as if his head would burst\allowbreak---\allowbreak sneeze after sneeze seemed nearly to choke him; he was blind, deaf, and dumb for the moment, and during that moment Blakeney quietly, without the slightest haste, took up his hat, took some money out of his pocket, which he left on the table, then calmly stalked out of the room!