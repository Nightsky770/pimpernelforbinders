%!TeX root=../scarlettop.tex

\chapter{The Jew}
\lettrine[lines=4]{I}{t} took Marguerite some time to collect her scattered senses; the whole of this last short episode had taken place in less than a minute, and Desgas and the soldiers were still about two hundred yards away from the \enquote{Chat Gris.}

When she realised what had happened, a curious mixture of joy and wonder filled her heart. It all was so neat, so ingenious. Chauvelin was still absolutely helpless, far more so than he could even have been under a blow from the fist, for now he could neither see, nor hear, nor speak, whilst his cunning adversary had quietly slipped through his fingers.

Blakeney was gone, obviously to try and join the fugitives at the Père Blanchard's hut. For the moment, true, Chauvelin was helpless; for the moment the daring Scarlet Pimpernel had not been caught by Desgas and his men. But all the roads and the beach were patrolled. Every place was watched, and every stranger kept in sight. How far could Percy go, thus arrayed in his gorgeous clothes, without being sighted and followed?

Now she blamed herself terribly for not having gone down to him sooner, and given him that word of warning and of love which, perhaps, after all, he needed. He could not know of the orders which Chauvelin had given for his capture, and even now, perhaps... But before all these horrible thoughts had taken concrete form in her brain, she heard the grounding of arms outside, close to the door, and Desgas’ voice shouting \enquote{Halt!} to his men.

Chauvelin had partially recovered; his sneezing had become less violent, and he had struggled to his feet. He managed to reach the door just as Desgas’ knock was heard on the outside.

Chauvelin threw open the door, and before his secretary could say a word, he had managed to stammer between two sneezes---

\enquote{The tall stranger---quick!---did any of you see him?}

\enquote{Where, citoyen?} asked Desgas, in surprise.

\enquote{Here, man! through that door! not five minutes ago.}

\enquote{We saw nothing, citoyen! The moon is not yet up, and...}

\enquote{And you are just five minutes too late, my friend,} said Chauvelin, with concentrated fury.

\enquote{Citoyen... I...}

\enquote{You did what I ordered you to do,} said Chauvelin, with impatience. \enquote{I know that, but you were a precious long time about it. Fortunately, there's not much harm done, or it had fared ill with you, Citoyen Desgas.}

Desgas turned a little pale. There was so much rage and hatred in his superior's whole attitude.

\enquote{The tall stranger, citoyen---} he stammered.

\enquote{Was here, in this room, five minutes ago, having supper at that table. Damn his impudence! For obvious reasons, I dared not tackle him alone. Brogard is too big a fool, and that cursed Englishman appears to have the strength of a bullock, and so he slipped away under your very nose.}

\enquote{He cannot go far without being sighted, citoyen.}

\enquote{Ah?}

\enquote{Captain Jutley sent forty men as reinforcements for the patrol duty: twenty went down to the beach. He again assured me that the watch has been constant all day, and that no stranger could possibly get to the beach, or reach a boat, without being sighted.}

\enquote{That's good.---Do the men know their work?}

\enquote{They have had very clear orders, citoyen: and I myself spoke to those who were about to start. They are to shadow---as secretly as possible---any stranger they may see, especially if he be tall, or stoop as if he would disguise his height.}

\enquote{In no case to detain such a person, of course,} said Chauvelin, eagerly. \enquote{That impudent Scarlet Pimpernel would slip through clumsy fingers. We must let him get to the Père Blanchard's hut now; there surround and capture him.}

\enquote{The men understand that, citoyen, and also that, as soon as a tall stranger has been sighted, he must be shadowed, whilst one man is to turn straight back and report to you.}

\enquote{That is right,} said Chauvelin, rubbing his hands, well pleased.

\enquote{I have further news for you, citoyen.}

\enquote{What is it?}

\enquote{A tall Englishman had a long conversation about three-quarters of an hour ago with a Jew, Reuben by name, who lives not ten paces from here.}

\enquote{Yes---and?} queried Chauvelin, impatiently.

\enquote{The conversation was all about a horse and cart, which the tall Englishman wished to hire, and which was to have been ready for him by eleven o'clock.}

\enquote{It is past that now. Where does that Reuben live?}

\enquote{A few minutes’ walk from this door.}

\enquote{Send one of the men to find out if the stranger has driven off in Reuben's cart.}

\enquote{Yes, citoyen.}

Desgas went to give the necessary orders to one of the men. Not a word of this conversation between him and Chauvelin had escaped Marguerite, and every word they had spoken seemed to strike at her heart, with terrible hopelessness and dark foreboding.

She had come all this way, and with such high hopes and firm determination to help her husband, and so far she had been able to do nothing, but to watch, with a heart breaking with anguish, the meshes of the deadly net closing round the daring Scarlet Pimpernel.

He could not now advance many steps, without spying eyes to track and denounce him. Her own helplessness struck her with the terrible sense of utter disappointment. The possibility of being of the slightest use to her husband had become almost \textit{nil}, and her only hope rested in being allowed to share his fate, whatever it might ultimately be.

For the moment, even her chance of ever seeing the man she loved again, had become a remote one. Still, she was determined to keep a close watch over his enemy, and a vague hope filled her heart, that whilst she kept Chauvelin in sight, Percy's fate might still be hanging in the balance.

Desgas had left Chauvelin moodily pacing up and down the room, whilst he himself waited outside for the return of the man whom he had sent in search of Reuben. Thus several minutes went by. Chauvelin was evidently devoured with impatience. Apparently he trusted no one: this last trick played upon him by the daring Scarlet Pimpernel had made him suddenly doubtful of success, unless he himself was there to watch, direct and superintend the capture of this impudent Englishman.

About five minutes later, Desgas returned, followed by an elderly Jew, in a dirty, threadbare gaberdine, worn greasy across the shoulders. His red hair, which he wore after the fashion of the Polish Jews, with the corkscrew curls each side of his face, was plentifully sprinkled with grey---a general coating of grime, about his cheeks and his chin, gave him a peculiarly dirty and loathsome appearance. He had the habitual stoop, those of his race affected in mock humility in past centuries, before the dawn of equality and freedom in matters of faith, and he walked behind Desgas with the peculiar shuffling gait which has remained the characteristic of the Jew trader in continental Europe to this day.

Chauvelin, who had all the Frenchman's prejudice against the despised race, motioned to the fellow to keep at a respectful distance. The group of the three men were standing just underneath the hanging oil-lamp, and Marguerite had a clear view of them all.

\enquote{Is this the man?} asked Chauvelin.

\enquote{No, citoyen,} replied Desgas, \enquote{Reuben could not be found, so presumably his cart has gone with the stranger; but this man here seems to know something, which he is willing to sell for a consideration.}

\enquote{Ah!} said Chauvelin, turning away with disgust from the loathsome specimen of humanity before him.

The Jew, with characteristic patience, stood humbly on one side, leaning on a thick knotted staff, his greasy, broad-brimmed hat casting a deep shadow over his grimy face, waiting for the noble Excellency to deign to put some questions to him.

\enquote{The citoyen tells me,} said Chauvelin peremptorily to him, \enquote{that you know something of my friend, the tall Englishman, whom I desire to meet... \textit{Morbleu}! keep your distance, man,} he added hurriedly, as the Jew took a quick and eager step forward.

\enquote{Yes, your Excellency,} replied the Jew, who spoke the language with that peculiar lisp which denotes Eastern origin, \enquote{I and Reuben Goldstein met a tall Englishman, on the road, close by here this evening.}

\enquote{Did you speak to him?}

\enquote{He spoke to us, your Excellency. He wanted to know if he could hire a horse and cart to go down along the St~Martin Road, to a place he wanted to reach to-night.}

\enquote{What did you say?}

\enquote{I did not say anything,} said the Jew in an injured tone, \enquote{Reuben Goldstein, that accursed traitor, that son of Belial...}

\enquote{Cut that short, man,} interrupted Chauvelin, roughly, \enquote{and go on with your story.}

\enquote{He took the words out of my mouth, your Excellency; when I was about to offer the wealthy Englishman my horse and cart, to take him wheresoever he chose, Reuben had already spoken, and offered his half-starved nag, and his broken-down cart.}

\enquote{And what did the Englishman do?}

\enquote{He listened to Reuben Goldstein, your Excellency, and put his hand in his pocket then and there, and took out a handful of gold, which he showed to that descendant of Beelzebub, telling him that all that would be his, if the horse and cart were ready for him by eleven o'clock.}

\enquote{And, of course, the horse and cart were ready?}

\enquote{Well! they were ready in a manner, so to speak, your Excellency. Reuben's nag was lame as usual; she refused to budge at first. It was only after a time and with plenty of kicks, that she at last could be made to move,} said the Jew with a malicious chuckle.

\enquote{Then they started?}

\enquote{Yes, they started about five minutes ago. I was disgusted with that stranger's folly. An Englishman too!---He ought to have known Reuben's nag was not fit to drive.}

\enquote{But if he had no choice?}

\enquote{No choice, your Excellency?} protested the Jew, in a rasping voice, \enquote{did I not repeat to him a dozen times, that my horse and cart would take him quicker, and more comfortably than Reuben's bag of bones. He would not listen. Reuben is such a liar, and has such insinuating ways. The stranger was deceived. If he was in a hurry, he would have had better value for his money by taking my cart.}

\enquote{You have a horse and cart too, then?} asked Chauvelin, peremptorily.

\enquote{Aye! that I have, your Excellency, and if your Excellency wants to drive...}

\enquote{Do you happen to know which way my friend went in Reuben Goldstein's cart?}

Thoughtfully the Jew rubbed his dirty chin. Marguerite's heart was beating well-nigh to bursting. She had heard the peremptory question; she looked anxiously at the Jew, but could not read his face beneath the shadow of his broad-brimmed hat. Vaguely she felt somehow as if he held Percy's fate in his long, dirty hands.

There was a long pause, whilst Chauvelin frowned impatiently at the stooping figure before him: at last the Jew slowly put his hand in his breast pocket, and drew out from its capacious depths a number of silver coins. He gazed at them thoughtfully, then remarked, in a quiet tone of voice,---

\enquote{This is what the tall stranger gave me, when he drove away with Reuben, for holding my tongue about him, and his doings.}

Chauvelin shrugged his shoulders impatiently.

\enquote{How much is there there?} he asked.

\enquote{Twenty francs, your Excellency,} replied the Jew, \enquote{and I have been an honest man all my life.}

Chauvelin without further comment took a few pieces of gold out of his own pocket, and leaving them in the palm of his hand, he allowed them to jingle as he held them out towards the Jew.

\enquote{How many gold pieces are there in the palm of my hand?} he asked quietly.

Evidently he had no desire to terrorise the man, but to conciliate him, for his own purposes, for his manner was pleasant and suave. No doubt he feared that threats of the guillotine, and various other persuasive methods of that type, might addle the old man's brains, and that he would be more likely to be useful through greed of gain, than through terror of death.

The eyes of the Jew shot a quick, keen glance at the gold in his interlocutor's hand.

\enquote{At least five, I should say, your Excellency,} he replied obsequiously.

\enquote{Enough, do you think, to loosen that honest tongue of yours?}

\enquote{What does your Excellency wish to know?}

\enquote{Whether your horse and cart can take me to where I can find my friend the tall stranger, who has driven off in Reuben Goldstein's cart?}

\enquote{My horse and cart can take your Honour there, where you please.}

\enquote{To a place called the Père Blanchard's hut?}

\enquote{Your Honour has guessed?} said the Jew in astonishment.

\enquote{You know the place?}

\enquote{I know it, your Honour.}

\enquote{Which road leads to it?}

\enquote{The St~Martin Road, your Honour, then a footpath from there to the cliffs.}

\enquote{You know the road?} repeated Chauvelin, roughly.

\enquote{Every stone, every blade of grass, your Honour,} replied the Jew quietly.

Chauvelin without another word threw the five pieces of gold one by one before the Jew, who knelt down, and on his hands and knees struggled to collect them. One rolled away, and he had some trouble to get it, for it had lodged underneath the dresser. Chauvelin quietly waited while the old man scrambled on the floor, to find the piece of gold.

When the Jew was again on his feet, Chauvelin said,---

\enquote{How soon can your horse and cart be ready?}

\enquote{They are ready now, your Honour.}

\enquote{Where?}

\enquote{Not ten mètres from this door. Will your Excellency deign to look?}

\enquote{I don't want to see it. How far can you drive me in it?}

\enquote{As far as the Père Blanchard's hut, your Honour, and further than Reuben's nag took your friend. I am sure that, not two leagues from here, we shall come across that wily Reuben, his nag, his cart and the tall stranger all in a heap in the middle of the road.}

\enquote{How far is the nearest village from here?}

\enquote{On the road which the Englishman took, Miquelon is the nearest village, not two leagues from here.}

\enquote{There he could get fresh conveyance, if he wanted to go further?}

\enquote{He could---if he ever got so far.}

\enquote{Can you?}

\enquote{Will your Excellency try?} said the Jew simply.

\enquote{That is my intention,} said Chauvelin very quietly, \enquote{but remember, if you have deceived me, I shall tell off two of my most stalwart soldiers to give you such a beating, that your breath will perhaps leave your ugly body for ever. But if we find my friend the tall Englishman, either on the road or at the Père Blanchard's hut, there will be ten more gold pieces for you. Do you accept the bargain?}

The Jew again thoughtfully rubbed his chin. He looked at the money in his hand, then at his stern interlocutor, and at Desgas, who had stood silently behind him all this while. After a moment's pause, he said deliberately,---

\enquote{I accept.}

\enquote{Go and wait outside then,} said Chauvelin, \enquote{and remember to stick to your bargain, or by Heaven, I will keep to mine.}

With a final, most abject and cringing bow, the old Jew shuffled out of the room. Chauvelin seemed pleased with his interview, for he rubbed his hands together, with that usual gesture of his, of malignant satisfaction.

\enquote{My coat and boots,} he said to Desgas at last.

Desgas went to the door, and apparently gave the necessary orders, for presently a soldier entered, carrying Chauvelin's coat, boots, and hat.

He took off his soutane, beneath which he was wearing close-fitting breeches and a cloth waistcoat, and began changing his attire.

\enquote{You, citoyen, in the meanwhile,} he said to Desgas, \enquote{go back to Captain Jutley as fast as you can, and tell him to let you have another dozen men, and bring them with you along the St~Martin Road, where I daresay you will soon overtake the Jew's cart with myself in it. There will be hot work presently, if I mistake not, in the Père Blanchard's hut. We shall corner our game there, I'll warrant, for this impudent Scarlet Pimpernel has had the audacity---or the stupidity, I hardly know which---to adhere to his original plans. He has gone to meet de Tournay, St~Just and the other traitors, which for the moment, I thought, perhaps, he did not intend to do. When we find them, there will be a band of desperate men at bay. Some of our men will, I presume, be put \textit{hors de combat}. These royalists are good swordsmen, and the Englishman is devilish cunning, and looks very powerful. Still, we shall be five against one at least. You can follow the cart closely with your men, all along the St~Martin Road, through Miquelon. The Englishman is ahead of us, and not likely to look behind him.}

Whilst he gave these curt and concise orders, he had completed his change of attire. The priest's costume had been laid aside, and he was once more dressed in his usual dark, tight-fitting clothes. At last he took up his hat.

\enquote{I shall have an interesting prisoner to deliver into your hands,} he said with a chuckle, as with unwonted familiarity he took Desgas’ arm, and led him towards the door. \enquote{We won't kill him outright, eh, friend Desgas? The Père Blanchard's hut is---an I mistake not---a lonely spot upon the beach, and our men will enjoy a bit of rough sport there with the wounded fox. Choose your men well, friend Desgas... of the sort who would enjoy that type of sport---eh? We must see that Scarlet Pimpernel wither a bit---what?---shrink and tremble, eh?... before we finally...}---he made an expressive gesture, whilst he laughed a low, evil laugh, which filled Marguerite's soul with sickening horror.

\enquote{Choose your men well, Citoyen Desgas,} he said once more, as he led his secretary finally out of the room.