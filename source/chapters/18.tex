%!TeX root=../scarlettop.tex

\chapter{The Mysterious Device}

\lettrine[lines=4]{T}{he} day was well advanced when Marguerite woke, refreshed by her long sleep. Louise had brought her some fresh milk and a dish of fruit, and she partook of this frugal breakfast with hearty appetite.

Thoughts crowded thick and fast in her mind as she munched her grapes; most of them went galloping away after the tall, erect figure of her husband, whom she had watched riding out of sight more than five hours ago.

In answer to her eager inquiries, Louise brought back the news that the groom had come home with Sultan, having left Sir Percy in London. The groom thought that his master was about to get on board his schooner, which was lying off just below London Bridge. Sir Percy had ridden thus far, had then met Briggs, the skipper of the \textit{Day Dream}, and had sent the groom back to Richmond with Sultan and the empty saddle.

This news puzzled Marguerite more than ever. Where could Sir Percy be going just now in the \textit{Day Dream}? On Armand's behalf, he had said. Well! Sir Percy had influential friends everywhere. Perhaps he was going to Greenwich, or... but Marguerite ceased to conjecture; all would be explained anon: he said that he would come back, and that he would remember.

A long, idle day lay before Marguerite. She was expecting the visit of her old school-fellow, little Suzanne de Tournay. With all the merry mischief at her command, she had tendered her request for Suzanne's company to the Comtesse in the presence of the Prince of Wales last night. His Royal Highness had loudly applauded the notion, and declared that he would give himself the pleasure of calling on the two ladies in the course of the afternoon. The Comtesse had not dared to refuse, and then and there was entrapped into a promise to send little Suzanne to spend a long and happy day at Richmond with her friend.

Marguerite expected her eagerly; she longed for a chat about old schooldays with the child; she felt that she would prefer Suzanne's company to that of anyone else, and together they would roam through the fine old garden and rich deer park, or stroll along the river.

But Suzanne had not come yet, and Marguerite being dressed, prepared to go downstairs. She looked quite a girl this morning in her simple muslin frock, with a broad blue sash round her slim waist, and the dainty cross-over fichu into which, at her bosom, she had fastened a few late crimson roses.

She crossed the landing outside her own suite of apartments, and stood still for a moment at the head of the fine oak staircase, which led to the lower floor. On her left were her husband's apartments, a suite of rooms which she practically never entered.

They consisted of bedroom, dressing and reception-room, and, at the extreme end of the landing, of a small study, which, when Sir Percy did not use it, was always kept locked. His own special and confidential valet, Frank, had charge of this room. No one was ever allowed to go inside. My lady had never cared to do so, and the other servants had, of course, not dared to break this hard-and-fast rule.

Marguerite had often, with that good-natured contempt which she had recently adopted towards her husband, chaffed him about this secrecy which surrounded his private study. Laughingly she had always declared that he strictly excluded all prying eyes from his sanctum for fear they should detect how very little »study« went on within its four walls: a comfortable arm-chair for Sir Percy's sweet slumbers was, no doubt, its most conspicuous piece of furniture.

Marguerite thought of all this on this bright October morning as she glanced along the corridor. Frank was evidently busy with his master's rooms, for most of the doors stood open, that of the study amongst the others.

A sudden, burning, childish curiosity seized her to have a peep at Sir Percy's sanctum. The restriction, of course, did not apply to her, and Frank would, of course, not dare to oppose her. Still, she hoped that the valet would be busy in one of the other rooms, that she might have that one quick peep in secret, and unmolested.

Gently, on tip-toe, she crossed the landing and, like Blue Beard's wife, trembling half with excitement and wonder, she paused a moment on the threshold, strangely perturbed and irresolute.

The door was ajar, and she could not see anything within. She pushed it open tentatively: there was no sound: Frank was evidently not there, and she walked boldly in.

At once she was struck by the severe simplicity of everything around her: the dark and heavy hangings, the massive oak furniture, the one or two maps on the wall, in no way recalled to her mind the lazy man about town, the lover of race-courses, the dandified leader of fashion, that was the outward representation of Sir Percy Blakeney.

There was no sign here, at any rate, of hurried departure. Every\-thing was in its place, not a scrap of paper littered the floor, not a cupboard or drawer was left open. The curtains were drawn aside, and through the open window the fresh morning air was streaming in.

Facing the window, and well into the centre of the room, stood a ponderous business-like desk, which looked as if it had seen much service. On the wall to the left of the desk, reaching almost from floor to ceiling, was a large full-length portrait of a woman, magnificently framed, exquisitely painted, and signed with the name of Boucher. It was Percy's mother.

Marguerite knew very little about her, except that she had died abroad, ailing in body as well as in mind, when Percy was still a lad. She must have been a very beautiful woman once, when Boucher painted her, and as Marguerite looked at the portrait, she could not but be struck by the extraordinary resemblance which must have existed between mother and son. There was the same low, square forehead, crowned with thick, fair hair, smooth and heavy; the same deep-set, somewhat lazy blue eyes beneath firmly marked, straight brows; and in those eyes there was the same intensity behind that apparent laziness, the same latent passion which used to light up Percy's face in the olden days before his marriage, and which Marguerite had again noted, last night at dawn, when she had come quite close to him, and had allowed a note of tenderness to creep into her voice.

Marguerite studied the portrait, for it interested her: after that she turned and looked again at the ponderous desk. It was covered with a mass of papers, all neatly tied and docketed, which looked like accounts and receipts arrayed with perfect method. It had never before struck Marguerite\allowbreak---\allowbreak nor had she, alas! found it worth while to inquire\allowbreak---\allowbreak as to how Sir Percy, whom all the world had credited with a total lack of brains, administered the vast fortune which his father had left him.

Since she had entered this neat, orderly room, she had been taken so much by surprise, that this obvious proof of her husband's strong business capacities did not cause her more than a passing thought of wonder. But it also strengthened her in the now certain knowledge that, with his worldly inanities, his foppish ways, and foolish talk, he was not only wearing a mask, but was playing a deliberate and studied part.

Marguerite wondered again. Why should he take all this trouble? Why should he\allowbreak---\allowbreak who was obviously a serious, earnest man\allowbreak---\allowbreak wish to appear before his fellow-men as an empty-headed nincompoop?

He may have wished to hide his love for a wife who held him in contempt... but surely such an object could have been gained at less sacrifice, and with far less trouble than constant incessant acting of an unnatural part.

She looked round her quite aimlessly now: she was horribly puzzled, and a nameless dread, before all this strange, unaccountable mystery, had begun to seize upon her. She felt cold and uncomfortable suddenly in this severe and dark room. There were no pictures on the wall, save the fine Boucher portrait, only a couple of maps, both of parts of France, one of the North coast and the other of the environs of Paris. What did Sir Percy want with those, she wondered.

Her head began to ache, she turned away from this strange Blue Beard's chamber, which she had entered, and which she did not understand. She did not wish Frank to find her here, and with a last look round, she once more turned to the door. As she did so, her foot knocked against a small object, which had apparently been lying close to the desk, on the carpet, and which now went rolling, right across the room.

She stooped to pick it up. It was a solid gold ring, with a flat shield, on which was engraved a small device.

Marguerite turned it over in her fingers, and then studied the engraving on the shield. It represented a small star-shaped flower, of a shape she had seen so distinctly twice before: once at the opera, and once at Lord Grenville's ball.