%!TeX root=../scarlettop.tex

\chapter{Trapped}
\lettrine[lines=4]{S}{he} did not know how long she was thus carried along, she had lost all notion of time and space, and for a few seconds tired nature, mercifully, deprived her of consciousness.

When she once more realised her state, she felt that she was placed with some degree of comfort upon a man's coat, with her back resting against a fragment of rock. The moon was hidden again behind some clouds, and the darkness seemed in comparison more intense. The sea was roaring some two hundred feet below her, and on looking all round she could no longer see any vestige of the tiny glimmer of red light.

That the end of the journey had been reached, she gathered from the fact that she heard rapid questions and answers spoken in a whisper quite close to her.

\enquote{There are four men in there, citoyen; they are sitting by the fire, and seem to be waiting quietly.}

\enquote{The hour?}

\enquote{Nearly two o'clock.}

\enquote{The tide?}

\enquote{Coming in quickly.}

\enquote{The schooner?}

\enquote{Obviously an English one, lying some three kilometres out. But we cannot see her boat.}

\enquote{Have the men taken cover?}

\enquote{Yes, citoyen.}

\enquote{They will not blunder?}

\enquote{They will not stir until the tall Englishman comes, then they will surround and overpower the five men.}

\enquote{Right. And the lady?}

\enquote{Still dazed, I fancy. She's close beside you, citoyen.}

\enquote{And the Jew?}

\enquote{He's gagged, and his legs strapped together. He cannot move or scream.}

\enquote{Good. Then have your gun ready, in case you want it. Get close to the hut and leave me to look after the lady.}

Desgas evidently obeyed, for Marguerite heard him creeping away along the stony cliff, then she felt that a pair of warm, thin, talon-like hands took hold of both her own, and held them in a grip of steel.

\enquote{Before that handkerchief is removed from your pretty mouth, fair lady,} whispered Chauvelin close to her ear, \enquote{I think it right to give you one small word of warning. What has procured me the honour of being followed across the Channel by so charming a companion, I cannot, of course, conceive, but, if I mistake not, the purpose of this flattering attention is not one that would commend itself to my vanity, and I think that I am right in surmising, moreover, that the first sound which your pretty lips would utter, as soon as the cruel gag is removed, would be one that would perhaps prove a warning to the cunning fox, which I have been at such pains to track to his lair.}

He paused a moment, while the steel-like grasp seemed to tighten round her wrist; then he resumed in the same hurried whisper:\longdash


\enquote{Inside that hut, if again I am not mistaken, your brother, Armand St~Just, waits with that traitor de Tournay, and two other men unknown to you, for the arrival of the mysterious rescuer, whose identity has for so long puzzled our Committee of Public Safety---the audacious Scarlet Pimpernel. No doubt if you scream, if there is a scuffle here, if shots are fired, it is more than likely that the same long legs that brought this scarlet enigma here, will as quickly take him to some place of safety. The purpose then, for which I have travelled all these miles, will remain unaccomplished. On the other hand it only rests with yourself that your brother---Armand---shall be free to go off with you to-night if you like, to England, or any other place of safety.}

Marguerite could not utter a sound, as the handkerchief was wound very tightly round her mouth, but Chauvelin was peering through the darkness very closely into her face; no doubt too her hand gave a responsive appeal to his last suggestion, for presently he continued:\longdash


\enquote{What I want you to do to ensure Armand's safety is a very simple thing, dear lady.}

\enquote{What is it?} Marguerite's hand seemed to convey to his, in response.

\enquote{To remain---on this spot, without uttering a sound, until I give you leave to speak. Ah! but I think you will obey,} he added, with that funny dry chuckle of his as Marguerite's whole figure seemed to stiffen, in defiance of this order, \enquote{for let me tell you that if you scream, nay! if you utter one sound, or attempt to move from here, my men---there are thirty of them about---will seize St~Just, de Tournay, and their two friends, and shoot them here---by my orders---before your eyes.}

Marguerite had listened to her implacable enemy's speech with ever-increasing terror. Numbed with physical pain, she yet had sufficient mental vitality in her to realise the full horror of this terrible \enquote{either---or} he was once more putting before her; an \enquote{either---or} ten thousand times more appalling and horrible, than the one he had suggested to her that fatal night at the ball.

This time it meant that she should keep still, and allow the husband she worshipped to walk unconsciously to his death, or that she should, by trying to give him a word of warning, which perhaps might even be unavailing, actually give the signal for her own brother's death, and that of three other unsuspecting men.

She could not see Chauvelin, but she could almost feel those keen, pale eyes of his fixed maliciously upon her helpless form, and his hurried, whispered words reached her ear, as the death-knell of her last faint, lingering hope.

\enquote{Nay, fair lady,} he added urbanely, \enquote{you can have no interest in anyone save in St~Just, and all you need do for his safety is to remain where you are, and to keep silent. My men have strict orders to spare him in every way. As for that enigmatic Scarlet Pimpernel, what is he to you? Believe me, no warning from you could possibly save him. And now dear lady, let me remove this unpleasant coercion, which has been placed before your pretty mouth. You see I wish you to be perfectly free, in the choice which you are about to make.}

Her thoughts in a whirl, her temples aching, her nerves paralyzed, her body numb with pain, Marguerite sat there, in the darkness which surrounded her as with a pall. From where she sat she could not see the sea, but she heard the incessant mournful murmur of the incoming tide, which spoke of her dead hopes, her lost love, the husband she had with her own hand betrayed, and sent to his death.

Chauvelin removed the handkerchief from her mouth. She certainly did not scream: at that moment, she had no strength to do anything but barely to hold herself upright, and to force herself to think.

Oh! think! think! think! of what she should do. The minutes flew on; in this awful stillness she could not tell how fast or how slowly; she heard nothing, she saw nothing: she did not feel the sweet-smelling autumn air, scented with the briny odour of the sea, she no longer heard the murmur of the waves, the occasional rattling of a pebble, as it rolled down some steep incline. More and more unreal did the whole situation seem. It was impossible that she, Marguerite Blakeney, the queen of London society, should actually be sitting here on this bit of lonely coast, in the middle of the night, side by side with a most bitter enemy: and oh! it was not possible that somewhere, not many hundred feet away perhaps, from where she stood, the being she had once despised, but who now, in every moment of this weird, dreamlike life, became more and more dear---it was not possible that \textit{he} was unconsciously, even now walking to his doom, whilst she did nothing to save him.

Why did she not with unearthly screams, that would re-echo from one end of the lonely beach to the other, send out a warning to him to desist, to retrace his steps, for death lurked here whilst he advanced? Once or twice the screams rose to her throat---as if by instinct: then, before her eyes there stood the awful alternative: her brother and those three men shot before her eyes, practically by her orders: she their murderer.

Oh! that fiend in human shape, next to her, knew human---female---nature well. He had played upon her feelings as a skilful musician plays upon an instrument. He had gauged her very thoughts to a nicety.

She could not give that signal---for she was weak, and she was a woman. How could she deliberately order Armand to be shot before her eyes, to have his dear blood upon her head, he dying perhaps with a curse on her, upon his lips. And little Suzanne's father, too! he, an old man; and the others!---oh! it was all too, too horrible.

Wait! wait! wait! how long? The early morning hours sped on, and yet it was not dawn: the sea continued its incessant mournful murmur, the autumnal breeze sighed gently in the night: the lonely beach was silent, even as the grave.

Suddenly from somewhere, not very far away, a cheerful, strong voice was heard singing \enquote{God save the King!}