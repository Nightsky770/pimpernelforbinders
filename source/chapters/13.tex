%!TeX root=../scarlettop.tex

\chapter[Either---Or?]{Either–Or?}
\lettrine[lines=4]{T}{he} few words which Marguerite Blakeney had managed to read on the half-scorched piece of paper, seemed literally to be the words of Fate. \enquote{Start myself to-morrow...} This she had read quite distinctly; then came a blur caused by the smoke of the candle, which obliterated the next few words; but, right at the bottom, there was another sentence, which was now standing clearly and distinctly, like letters of fire, before her mental vision. \enquote{If you wish to speak to me again, I shall be in the supper-room at one o'clock precisely.} The whole was signed with the hastily-scrawled little device---a tiny star-shaped flower, which had become so familiar to her.

One o'clock precisely! It was now close upon eleven, the last minuet was being danced, with Sir Andrew Ffoulkes and beautiful Lady Blakeney leading the couples, through its delicate and intricate figures.

Close upon eleven! the hands of the handsome Louis XV clock upon its ormolu bracket seemed to move along with maddening rapidity. Two hours more, and her fate and that of Armand would be sealed. In two hours she must make up her mind whether she will keep the knowledge so cunningly gained to herself, and leave her brother to his fate, or whether she will wilfully betray a brave man, whose life was devoted to his fellow-men, who was noble, generous, and above all, unsuspecting. It seemed a horrible thing to do. But then, there was Armand! Armand, too, was noble and brave, Armand, too, was unsuspecting. And Armand loved her, would have willingly trusted his life in her hands, and now, when she could save him from death, she hesitated. Oh! it was monstrous; her brother's kind, gentle face, so full of love for her, seemed to be looking reproachfully at her. \enquote{You might have saved me, Margot!} he seemed to say to her, \enquote{and you chose the life of a stranger, a man you do not know, whom you have never seen, and preferred that he should be safe, whilst you sent me to the guillotine!}

All these conflicting thoughts raged through Marguerite's brain, while, with a smile upon her lips, she glided through the graceful mazes of the minuet. She noted---with that acute sense of hers---that she had succeeded in completely allaying Sir Andrew's fears. Her self-control had been absolutely perfect---she was a finer actress at this moment, and throughout the whole of this minuet, than she had ever been upon the boards of the Comédie Française; but then, a beloved brother's life had not depended upon her histrionic powers.

She was too clever to overdo her part, and made no further allusions to the supposed \textit{billet doux}, which had caused Sir Andrew Ffoulkes such an agonising five minutes. She watched his anxiety melting away under her sunny smile, and soon perceived that, whatever doubt may have crossed his mind at the moment, she had, by the time the last bars of the minuet had been played, succeeded in completely dispelling it; he never realised in what a fever of excitement she was, what effort it cost her to keep up a constant ripple of \textit{banal} conversation.

When the minuet was over, she asked Sir Andrew to take her into the next room.

\enquote{I have promised to go down to supper with His Royal Highness,} she said, \enquote{but before we part, tell me... am I forgiven?}

\enquote{Forgiven?}

\enquote{Yes! Confess, I gave you a fright just now... But, remember, I am not an Englishwoman, and I do not look upon the exchanging of \textit{billet doux} as a crime, and I vow I'll not tell my little Suzanne. But now, tell me, shall I welcome you at my water-party on Wednesday?}

\enquote{I am not sure, Lady Blakeney,} he replied evasively. \enquote{I may have to leave London to-morrow.}

\enquote{I would not do that, if I were you,} she said earnestly; then seeing the anxious look once more reappearing in his eyes, she added gaily; \enquote{No one can throw a ball better than you can, Sir Andrew, we should so miss you on the bowling-green.}

He had led her across the room, to one beyond, where already His Royal Highness was waiting for the beautiful Lady Blakeney.

\enquote{Madame, supper awaits us,} said the Prince, offering his arm to Marguerite, \enquote{and I am full of hope. The goddess Fortune has frowned so persistently on me at hazard, that I look with confidence for the smiles of the goddess of Beauty.}

\enquote{Your Highness has been unfortunate at the card tables?} asked Marguerite, as she took the Prince's arm.

\enquote{Aye! most unfortunate. Blakeney, not content with being the richest among my father's subjects, has also the most outrageous luck. By the way, where is that inimitable wit? I vow, Madam, that this life would be but a dreary desert without your smiles and his sallies.}

