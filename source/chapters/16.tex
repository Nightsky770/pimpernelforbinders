%!TeX root=../scarlettop.tex

\chapter{Richmond}
\lettrine[lines=4]{A}{} few minutes later she was sitting, wrapped in costly furs, near Sir Percy Blakeney on the box-seat of his magnificent coach, and the four splendid bays had thundered down the quiet street.

The night was warm in spite of the gentle breeze which fanned Marguerite's burning cheeks. Soon London houses were left behind, and rattling over old Hammersmith  Bridge, Sir Percy was driving his bays rapidly towards Richmond.

The river wound in and out in its pretty delicate curves, looking like a silver serpent beneath the glittering rays of the moon. Long shadows from overhanging trees spread occasional deep palls right across the road. The bays were rushing along at breakneck speed, held but slightly back by Sir Percy's strong, unerring hands.

These nightly drives after balls and suppers in London were a source of perpetual delight to Marguerite, and she appreciated her husband's eccentricity keenly, which caused him to adopt this mode of taking her home every night, to their beautiful home by the river, instead of living in a stuffy London house. He loved driving his spirited horses along the lonely, moonlit roads, and she loved to sit on the box-seat, with the soft air of an English late summer's night fanning her face after the hot atmosphere of a ball or supper-party. The drive was not a long one---less than an hour, sometimes, when the bays were very fresh, and Sir Percy gave them full rein.

To-night he seemed to have a very devil in his fingers, and the coach seemed to fly along the road, beside the river. As usual, he did not speak to her, but stared straight in front of him, the ribbons seeming to lie quite loosely in his slender, white hands. Marguerite looked at him tentatively once or twice; she could see his handsome profile, and one lazy eye, with its straight fine brow and drooping heavy lid.

The face in the moonlight looked singularly earnest, and recalled to Marguerite's aching heart those happy days of courtship, before he had become the lazy nincompoop, the effete fop, whose life seemed spent in card and supper rooms.

But now, in the moonlight, she could not catch the expression of the lazy blue eyes; she could only see the outline of the firm chin, the corner of the strong mouth, the well-cut massive shape of the forehead; truly, nature had meant well by Sir Percy; his faults must all be laid at the door of that poor, half-crazy mother, and of the distracted heart-broken father, neither of whom had cared for the young life which was sprouting up between them, and which, perhaps, their very carelessness was already beginning to wreck.

Marguerite suddenly felt intense sympathy for her husband. The moral crisis she had just gone through made her feel indulgent towards the faults, the delinquencies, of others.

How thoroughly a human being can be buffeted and overmastered by Fate, had been borne in upon her with appalling force. Had anyone told her a week ago that she would stoop to spy upon her friends, that she would betray a brave and unsuspecting man into the hands of a relentless enemy, she would have laughed the idea to scorn.

Yet she had done these things; anon, perhaps the death of that brave man would be at her door, just as two years ago the Marquis de St~Cyr had perished through a thoughtless word of hers; but in that case she was morally innocent---she had meant no serious harm---fate merely had stepped in. But this time she had done a thing that obviously was base, had done it deliberately, for a motive which, perhaps, high moralists would not even appreciate.

And as she felt her husband's strong arm beside her, she also felt how much more he would dislike and despise her, if he knew of this night's work. Thus human beings judge of one another, superficially, casually, throwing contempt on one another, with but little reason, and no charity. She despised her husband for his inanities and vulgar, unintellectual occupations; and he, she felt, would despise her still worse, because she had not been strong enough to do right for right's sake, and to sacrifice her brother to the dictates of her conscience.

Buried in her thoughts, Marguerite had found this hour in the breezy summer night all too brief; and it was with a feeling of keen disappointment, that she suddenly realised that the bays had turned into the massive gates of her beautiful English home.

Sir Percy Blakeney's house on the river has become a historic one: palatial in its dimensions, it stands in the midst of exquisitely laid-out gardens, with a picturesque terrace and frontage to the river. Built in Tudor days, the old red brick of the walls looks eminently picturesque in the midst of a bower of green, the beautiful lawn, with its old sun-dial, adding the true note of harmony to its foreground. Great secular trees lent cool shadows to the grounds, and now, on this warm early autumn night, the leaves slightly turned to russets and gold, the old garden looked singularly poetic and peaceful in the moonlight.

With unerring precision, Sir Percy had brought the four bays to a standstill immediately in front of the fine Elizabethan entrance hall; in spite of the lateness of the hour, an army of grooms seemed to have emerged from the very ground, as the coach had thundered up, and were standing respectfully round.

Sir Percy jumped down quickly, then helped Marguerite to alight. She lingered outside for a moment, whilst he gave a few orders to one of his men. She skirted the house, and stepped on to the lawn, looking out dreamily into the silvery landscape. Nature seemed exquisitely at peace, in comparison with the tumultuous emotions she had gone through: she could faintly hear the ripple of the river and the occasional soft and ghostlike fall of a dead leaf from a tree.

All else was quiet round her. She had heard the horses prancing as they were being led away to their distant stables, the hurrying of servants’ feet as they had all gone within to rest: the house also was quite still. In two separate suites of apartments, just above the magnificent reception-rooms, lights were still burning; they were her rooms, and his, well divided from each other by the whole width of the house, as far apart as their own lives had become. Involuntarily she sighed---at that moment she could really not have told why.

She was suffering from unconquerable heartache. Deeply and achingly she was sorry for herself. Never had she felt so pitiably lonely, so bitterly in want of comfort and of sympathy. With another sigh she turned away from the river towards the house, vaguely wondering if, after such a night, she could ever find rest and sleep.

Suddenly, before she reached the terrace, she heard a firm step upon the crisp gravel, and the next moment her husband's figure emerged out of the shadow. He, too, had skirted the house, and was wandering along the lawn, towards the river. He still wore his heavy driving coat with the numerous lapels and collars he himself had set in fashion, but he had thrown it well back, burying his hands as was his wont, in the deep pockets of his satin breeches: the gorgeous white costume he had worn at Lord Grenville's ball, with its jabot of priceless lace, looked strangely ghostly against the dark background of the house.

He apparently did not notice her, for, after a few moments’ pause, he presently turned back towards the house, and walked straight up to the terrace.

\enquote{Sir Percy!}

He already had one foot on the lowest of the terrace steps, but at her voice he started, and paused, then looked searchingly into the shadows whence she had called to him.

She came forward quickly into the moonlight, and, as soon as he saw her, he said, with that air of consummate gallantry he always wore when speaking to her,\longdash


\enquote{At your service, Madame!}

But his foot was still on the step, and in his whole attitude there was a remote suggestion, distinctly visible to her, that he wished to go, and had no desire for a midnight interview.

\enquote{The air is deliciously cool,} she said, \enquote{the moonlight peaceful and poetic, and the garden inviting. Will you not stay in it awhile; the hour is not yet late, or is my company so distasteful to you, that you are in a hurry to rid yourself of it?}

\enquote{Nay, Madame,} he rejoined placidly, \enquote{but `tis on the other foot the shoe happens to be, and I'll warrant you'll find the midnight air more poetic without my company: no doubt the sooner I remove the obstruction the better your ladyship will like it.}

He turned once more to go.

\enquote{I protest you mistake me, Sir Percy,} she said hurriedly, and drawing a little closer to him; \enquote{the estrangement, which, alas! has arisen between us, was none of my making, remember.}

\enquote{Begad! you must pardon me there, Madame!} he protested coldly, \enquote{my memory was always of the shortest.}

He looked her straight in the eyes, with that lazy nonchalance which had become second nature to him. She returned his gaze for a moment, then her eyes softened, as she came up quite close to him, to the foot of the terrace steps.

\enquote{Of the shortest, Sir Percy? Faith! how it must have altered! Was it three years ago or four that you saw me for one hour in Paris, on your way to the East? When you came back two years later you had not forgotten me.}

She looked divinely pretty as she stood there in the moonlight, with the fur-cloak sliding off her beautiful shoulders, the gold embroidery on her dress shimmering around her, her childlike blue eyes turned up fully at him.

He stood for a moment, rigid and still, but for the clenching of his hand against the stone balustrade of the terrace.

\enquote{You desired my presence, Madame,} he said frigidly. \enquote{I take it that it was not with a view to indulging in tender reminiscences.}

His voice certainly was cold and uncompromising: his attitude before her, stiff and unbending. Womanly decorum would have suggested that Marguerite should return coldness for coldness, and should sweep past him without another word, only with a curt nod of the head: but womanly instinct suggested that she should remain---that keen instinct, which makes a beautiful woman conscious of her powers long to bring to her knees the one man who pays her no homage. She stretched out her hand to him.

\enquote{Nay, Sir Percy, why not? the present is not so glorious but that I should not wish to dwell a little in the past.}

He bent his tall figure, and taking hold of the extreme tip of the fingers which she still held out to him, he kissed them ceremoniously.

\enquote{I’ faith, Madame,} he said, \enquote{then you will pardon me, if my dull wits cannot accompany you there.}

Once again he attempted to go, once more her voice, sweet, childlike, almost tender, called him back.

\enquote{Sir Percy.}

\enquote{Your servant, Madame.}

\enquote{Is it possible that love can die?} she said with sudden, unreasoning vehemence. \enquote{Methought that the passion which you once felt for me would outlast the span of human life. Is there nothing left of that love, Percy... which might help you... to bridge over that sad estrangement?}

His massive figure seemed, while she spoke thus to him, to stiffen still more, the strong mouth hardened, a look of relentless obstinacy crept into the habitually lazy blue eyes.

\enquote{With what object, I pray you, Madame?} he asked coldly.

\enquote{I do not understand you.}

\enquote{Yet `tis simple enough,} he said with sudden bitterness, which seemed literally to surge through his words, though he was making visible efforts to suppress it, \enquote{I humbly put the question to you, for my slow wits are unable to grasp the cause of this, your ladyship's sudden new mood. Is it that you have the taste to renew the devilish sport which you played so successfully last year? Do you wish to see me once more a love-sick suppliant at your feet, so that you might again have the pleasure of kicking me aside, like a troublesome lap-dog?}

She had succeeded in rousing him for the moment: and again she looked straight at him, for it was thus she remembered him a year ago.

\enquote{Percy! I entreat you!} she whispered, \enquote{can we not bury the past?}

\enquote{Pardon me, Madame, but I understood you to say that your desire was to dwell in it.}

\enquote{Nay! I spoke not of \textit{that} past, Percy!} she said, while a tone of tenderness crept into her voice. \enquote{Rather did I speak of the time when you loved me still! and I... oh! I was vain and frivolous; your wealth and position allured me: I married you, hoping in my heart that your great love for me would beget in me a love for you... but, alas!...}

The moon had sunk low down behind a bank of clouds. In the east a soft grey light was beginning to chase away the heavy mantle of the night. He could only see her graceful outline now, the small queenly head, with its wealth of reddish golden curls, and the glittering gems forming the small, star-shaped, red flower which she wore as a diadem in her hair.

\enquote{Twenty-four hours after our marriage, Madame, the Marquis de St~Cyr and all his family perished on the guillotine, and the popular rumour reached me that it was the wife of Sir Percy Blakeney who helped to send them there.}

\enquote{Nay! I myself told you the truth of that odious tale.}

\enquote{Not till after it had been recounted to me by strangers, with all its horrible details.}

\enquote{And you believed them then and there,} she said with great vehemence, \enquote{without a proof or question---you believed that I, whom you vowed you loved more than life, whom you professed you worshipped, that \textit{I} could do a thing so base as these \textit{strangers} chose to recount. You thought I meant to deceive you about it all---that I ought to have spoken before I married you: yet, had you listened, I would have told you that up to the very morning on which St~Cyr went to the guillotine, I was straining every nerve, using every influence I possessed, to save him and his family. But my pride sealed my lips, when your love seemed to perish, as if under the knife of that same guillotine. Yet I would have told you how I was duped! Aye! I, whom that same popular rumour had endowed with the sharpest wits in France! I was tricked into doing this thing, by men who knew how to play upon my love for an only brother, and my desire for revenge. Was it unnatural?}

Her voice became choked with tears. She paused for a moment or two, trying to regain some sort of composure. She looked appealingly at him, almost as if he were her judge. He had allowed her to speak on in her own vehement, impassioned way, offering no comment, no word of sympathy: and now, while she paused, trying to swallow down the hot tears that gushed to her eyes, he waited, impassive and still. The dim, grey light of early dawn seemed to make his tall form look taller and more rigid. The lazy, good-natured face looked strangely altered. Marguerite, excited, as she was, could see that the eyes were no longer languid, the mouth no longer good-humoured and inane. A curious look of intense passion seemed to glow from beneath his drooping lids, the mouth was tightly closed, the lips compressed, as if the will alone held that surging passion in check.

Marguerite Blakeney was, above all, a woman, with all a woman's fascinating foibles, all a woman's most lovable sins. She knew in a moment that for the past few months she had been mistaken: that this man who stood here before her, cold as a statue, when her musical voice struck upon his ear, loved her, as he had loved her a year ago: that his passion might have been dormant, but that it was there, as strong, as intense, as overwhelming, as when first her lips met his in one long, maddening kiss.

Pride had kept him from her, and, woman-like, she meant to win back that conquest which had been hers before. Suddenly it seemed to her that the only happiness life could ever hold for her again would be in feeling that man's kiss once more upon her lips.

\enquote{Listen to the tale, Sir Percy,} she said, and her voice now was low, sweet, infinitely tender. \enquote{Armand was all in all to me! We had no parents, and brought one another up. He was my little father, and I, his tiny mother; we loved one another so. Then one day---do you mind me, Sir Percy? the Marquis de St~Cyr had my brother Armand thrashed---thrashed by his lacqueys---that brother whom I loved better than all the world! And his offence? That he, a plebeian, had dared to love the daughter of the aristocrat; for that he was waylaid and thrashed... thrashed like a dog within an inch of his life! Oh, how I suffered! his humiliation had eaten into my very soul! When the opportunity occurred, and I was able to take my revenge, I took it. But I only thought to bring that proud marquis to trouble and humiliation. He plotted with Austria against his own country. Chance gave me knowledge of this; I spoke of it, but I did not know---how could I guess?---they trapped and duped me. When I realised what I had done, it was too late.}

\enquote{It is perhaps a little difficult, Madame,} said Sir Percy, after a moment of silence between them, \enquote{to go back over the past. I have confessed to you that my memory is short, but the thought certainly lingered in my mind that, at the time of the Marquis’ death, I entreated you for an explanation of those same noisome popular rumours. If that same memory does not, even now, play me a trick, I fancy that you refused me \textit{all} explanation then, and demanded of my love a humiliating allegiance it was not prepared to give.}

\enquote{I wished to test your love for me, and it did not bear the test. You used to tell me that you drew the very breath of life but for me, and for love of me.}

\enquote{And to probe that love, you demanded that I should forfeit mine honour,} he said, whilst gradually his impassiveness seemed to leave him, his rigidity to relax; \enquote{that I should accept without murmur or question, as a dumb and submissive slave, every action of my mistress. My heart overflowing with love and passion, I \textit{asked} for no explanation---I \textit{waited} for one, not doubting---only hoping. Had you spoken but one word, from you I would have accepted any explanation and believed it. But you left me without a word, beyond a bald confession of the actual horrible facts; proudly you returned to your brother's house, and left me alone... for weeks... not knowing, now, in whom to believe, since the shrine, which contained my one illusion, lay shattered to earth at my feet.}

She need not complain now that he was cold and impassive; his very voice shook with an intensity of passion, which he was making superhuman efforts to keep in check.

\enquote{Aye! the madness of my pride!} she said sadly. \enquote{Hardly had I gone, already I had repented. But when I returned, I found you, oh, so altered! wearing already that mask of somnolent indifference which you have never laid aside until... until now.}

She was so close to him that her soft, loose hair was wafted against his cheek; her eyes, glowing with tears, maddened him, the music in her voice sent fire through his veins. But he would not yield to the magic charm of this woman whom he had so deeply loved, and at whose hands his pride had suffered so bitterly. He closed his eyes to shut out the dainty vision of that sweet face, of that snow-white neck and graceful figure, round which the faint rosy light of dawn was just beginning to hover playfully.

\enquote{Nay, Madame, it is no mask,} he said icily; \enquote{I swore to you... once, that my life was yours. For months now it has been your plaything... it has served its purpose.}

But now she knew that that very coldness was a mask. The trouble, the sorrow she had gone through last night, suddenly came back to her mind, but no longer with bitterness, rather with a feeling that this man who loved her, would help her to bear the burden.

\enquote{Sir Percy,} she said impulsively, \enquote{Heaven knows you have been at pains to make the task, which I had set to myself, terribly difficult to accomplish. You spoke of my mood just now; well! we will call it that, if you will. I wished to speak to you... because... because I was in trouble... and had need... of your sympathy.}

\enquote{It is yours to command, Madame.}

\enquote{How cold you are!} she sighed. \enquote{Faith! I can scarce believe that but a few months ago one tear in my eye had set you well-nigh crazy. Now I come to you... with a half-broken heart... and... and...}

\enquote{I pray you, Madame,} he said, whilst his voice shook almost as much as hers, \enquote{in what way can I serve you?}

\enquote{Percy!---Armand is in deadly danger. A letter of his... rash, impetuous, as were all his actions, and written to Sir Andrew Ffoulkes, has fallen into the hands of a fanatic. Armand is hopelessly compromised... to-morrow, perhaps he will be arrested... after that the guillotine... unless... unless... oh! it is horrible!}... she said, with a sudden wail of anguish, as all the events of the past night came rushing back to her mind, \enquote{horrible!... and you do not understand... you cannot... and I have no one to whom I can turn... for help... or even for sympathy...}

Tears now refused to be held back. All her trouble, her struggles, the awful uncertainty of Armand's fate overwhelmed her. She tottered, ready to fall, and leaning against the stone balustrade, she buried her face in her hands and sobbed bitterly.

At first mention of Armand St~Just's name and of the peril in which he stood, Sir Percy's face had become a shade more pale; and the look of determination and obstinacy appeared more marked than ever between his eyes. However, he said nothing for the moment, but watched her, as her delicate frame was shaken with sobs, watched her until unconsciously his face softened, and what looked almost like tears seemed to glisten in his eyes.

\enquote{And so,} he said with bitter sarcasm, \enquote{the murderous dog of the revolution is turning upon the very hands that fed it?... Begad, Madame,} he added very gently, as Marguerite continued to sob hysterically, \enquote{will you dry your tears?... I never could bear to see a pretty woman cry, and I...}

Instinctively, with sudden, overmastering passion, at sight of her helplessness and of her grief, he stretched out his arms, and the next, would have seized her and held her to him, protected from every evil with his very life, his very heart's blood... But pride had the better of it in this struggle once again; he restrained himself with a tremendous effort of will, and said coldly, though still very gently,\longdash


\enquote{Will you not turn to me, Madame, and tell me in what way I may have the honour to serve you?}

She made a violent effort to control herself, and turning her tear-stained face to him, she once more held out her hand, which he kissed with the same punctilious gallantry; but Marguerite's fingers, this time, lingered in his hand for a second or two longer than was absolutely necessary, and this was because she had felt that his hand trembled perceptibly and was burning hot, whilst his lips felt as cold as marble.

\enquote{Can you do aught for Armand?} she said sweetly and simply. \enquote{You have so much influence at court... so many friends...}

\enquote{Nay, Madame, should you not rather seek the influence of your French friend, M. Chauvelin? His extends, if I mistake not, even as far as the Republican Government of France.}

\enquote{I cannot ask him, Percy... Oh! I wish I dared to tell you... but... but... he has put a price on my brother's head, which...}

She would have given worlds if she had felt the courage then to tell him everything... all she had done that night---how she had suffered and how her hand had been forced. But she dared not give way to that impulse... not now, when she was just beginning to feel that he still loved her, when she hoped that she could win him back. She dared not make another confession to him. After all, he might not understand; he might not sympathise with her struggles and temptation. His love still dormant might sleep the sleep of death.

Perhaps he divined what was passing in her mind. His whole attitude was one of intense longing---a veritable prayer for that confidence, which her foolish pride withheld from him. When she remained silent he sighed, and said with marked coldness\longdash


\enquote{Faith, Madame, since it distresses you, we will not speak of it... As for Armand, I pray you have no fear. I pledge you my word that he shall be safe. Now, have I your permission to go? The hour is getting late, and...}

\enquote{You will at least accept my gratitude?} she said, as she drew quite close to him, and speaking with real tenderness.

With a quick, almost involuntary effort he would have taken her then in his arms, for her eyes were swimming in tears, which he longed to kiss away; but she had lured him once, just like this, then cast him aside like an ill-fitting glove. He thought this was but a mood, a caprice, and he was too proud to lend himself to it once again.

\enquote{It is too soon, Madame!} he said quietly; \enquote{I have done nothing as yet. The hour is late, and you must be fatigued. Your women will be waiting for you upstairs.}

He stood aside to allow her to pass. She sighed, a quick sigh of disappointment. His pride and her beauty had been in direct conflict, and his pride had remained the conqueror. Perhaps, after all, she had been deceived just now; what she took to be the light of love in his eyes might only have been the passion of pride or, who knows, of hatred instead of love. She stood looking at him for a moment or two longer. He was again as rigid, as impassive, as before. Pride had conquered, and he cared naught for her. The grey of dawn was gradually yielding to the rosy light of the rising sun. Birds began to twitter; Nature awakened, smiling in happy response to the warmth of this glorious October morning. Only between these two hearts there lay a strong, impassable barrier, built up of pride on both sides, which neither of them cared to be the first to demolish.

He had bent his tall figure in a low ceremonious bow, as she finally, with another bitter little sigh, began to mount the terrace steps.

The long train of her gold-embroidered gown swept the dead leaves off the steps, making a faint harmonious sh---sh---sh as she glided up, with one hand resting on the balustrade, the rosy light of dawn making an aureole of gold round her hair, and causing the rubies on her head and arms to sparkle. She reached the tall glass doors which led into the house. Before entering, she paused once again to look at him, hoping against hope to see his arms stretched out to her, and to hear his voice calling her back. But he had not moved; his massive figure looked the very personification of unbending pride, of fierce obstinacy.

Hot tears again surged to her eyes, and as she would not let him see them, she turned quickly within, and ran as fast as she could up to her own rooms.

Had she but turned back then, and looked out once more on to the rose-lit garden, she would have seen that which would have made her own sufferings seem but light and easy to bear---a strong man, overwhelmed with his own passion and his own despair. Pride had given way at last, obstinacy was gone: the will was powerless. He was but a man madly, blindly, passionately in love, and as soon as her light footsteps had died away within the house, he knelt down upon the terrace steps, and in the very madness of his love he kissed one by one the places where her small foot had trodden, and the stone balustrade there, where her tiny hand had rested last.