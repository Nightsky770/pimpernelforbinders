%!TeX root=../scarlettop.tex


\chapter{The Friend}
\lettrine[lines=4]{L}{ess} than half an hour later, Marguerite, buried in thoughts, sat inside her coach, which was bearing her swiftly to London. She had taken an affectionate farewell of little Suzanne, and seen the child safely started with her maid, and in her own coach, back to town. She had sent one courier with a respectful letter of excuse to His Royal Highness, begging for a postponement of the august visit on account of pressing and urgent business, and another on ahead to bespeak a fresh relay of horses at Faversham.

Then she had changed her muslin frock for a dark travelling costume and mantle, had provided herself with money—which her husband's lavishness always placed fully at her disposal—and had started on her way.

She did not attempt to delude herself with any vain and futile hopes; the safety of her brother Armand was to have been conditional on the imminent capture of the Scarlet Pimpernel. As Chauvelin had sent her back Armand's compromising letter, there was no doubt that he was quite satisfied in his own mind that Percy Blakeney was the man whose death he had sworn to bring about.

No! there was no room for any fond delusions! Percy, the husband whom she loved with all the ardour which her admiration for his bravery had kindled, was in immediate, deadly peril, through her hand. She had betrayed him to his enemy—unwittingly 'tis true—but she \textit{had} betrayed him, and if Chauvelin succeeded in trapping him, who so far was unaware of his danger, then his death would be at her door. His death! when with her very heart's blood, she would have defended him and given willingly her life for his.

She had ordered her coach to drive her to the »Crown« inn; once there, she told her coachman to give the horses food and rest. Then she ordered a chair, and had herself carried to the house in Pall Mall where Sir Andrew Ffoulkes lived.

Among all Percy's friends who were enrolled under his daring banner, she felt that she would prefer to confide in Sir Andrew Ffoulkes. He had always been her friend, and now his love for little Suzanne had brought him closer to her still. Had he been away from home, gone on the mad errand with Percy, perhaps, then she would have called on Lord Hastings or Lord Tony—for she wanted the help of one of these young men, or she would be indeed powerless to save her husband.

Sir Andrew Ffoulkes, however, was at home, and his servant introduced her ladyship immediately. She went upstairs to the young man's comfortable bachelor's chambers, and was shown into a small, though luxuriously furnished, dining-room. A moment or two later Sir Andrew himself appeared.

He had evidently been much startled when he heard who his lady visitor was, for he looked anxiously—even suspiciously—at Marguerite, whilst performing the elaborate bows before her, which the rigid etiquette of the time demanded.

Marguerite had laid aside every vestige of nervousness; she was perfectly calm, and having returned the young man's elaborate salute, she began very calmly,\longdash


»Sir Andrew, I have no desire to waste valuable time in much talk. You must take certain things I am going to tell you for granted. These will be of no importance. What is important is that your leader and comrade, the Scarlet Pimpernel\textellipsis \allowbreak  my husband\textellipsis \allowbreak  Percy Blakeney\textellipsis \allowbreak  is in deadly peril.«

Had she had the remotest doubt of the correctness of her deductions, she would have had them confirmed now, for Sir Andrew, completely taken by surprise, had grown very pale, and was quite incapable of making the slightest attempt at clever parrying.

»No matter how I know this, Sir Andrew,« she continued quietly, »thank God that I do, and that perhaps it is not too late to save him. Unfortunately, I cannot do this quite alone, and therefore have come to you for help.«

»Lady Blakeney,« said the young man, trying to recover himself, »I\textellipsis«

»Will you hear me first?« she interrupted. »This is how the matter stands. When the agent of the French Government stole your papers that night in Dover, he found amongst them certain plans, which you or your leader meant to carry out for the rescue of the Comte de Tournay and others. The Scarlet Pimpernel—Percy, my husband—has gone on this errand himself to-day. Chauvelin knows that the Scarlet Pimpernel and Percy Blakeney are one and the same person. He will follow him to Calais, and there will lay hands on him. You know as well as I do the fate that awaits him at the hands of the Revolutionary Government of France. No interference from England—from King George himself—would save him. Robespierre and his gang would see to it that the interference came too late. But not only that, the much-trusted leader will also have been unconsciously the means of revealing the hiding-place of the Comte de Tournay and of all those who, even now, are placing their hopes in him.«

She had spoken quietly, dispassionately, and with firm, unbending resolution. Her purpose was to make that young man trust and help her, for she could do nothing without him.

»I do not understand,« he repeated, trying to gain time, to think what was best to be done.

»Aye! but I think you do, Sir Andrew. You must know that I am speaking the truth. Look these facts straight in the face. Percy has sailed for Calais, I presume for some lonely part of the coast, and Chauvelin is on his track. \textit{He} has posted for Dover, and will cross the Channel probably to-night. What do you think will happen?«

The young man was silent.

»Percy will arrive at his destination: unconscious of being followed he will seek out de Tournay and the others—among these is Armand St~Just, my brother—he will seek them out, one after another, probably, not knowing that the sharpest eyes in the world are watching his every movement. When he has thus unconsciously betrayed those who blindly trust in him, when nothing can be gained from him, and he is ready to come back to England, with those whom he has gone so bravely to save, the doors of the trap will close upon him, and he will be sent to end his noble life upon the guillotine.«

Still Sir Andrew was silent.

»You do not trust me,« she said passionately. »Oh, God! cannot you see that I am in deadly earnest? Man, man,« she added, while, with her tiny hands she seized the young man suddenly by the shoulders, forcing him to look straight at her, »tell me, do I look like that vilest thing on earth—a woman who would betray her own husband?«

»God forbid, Lady Blakeney,« said the young man at last, »that I should attribute such evil motives to you, but\textellipsis«

»But what?\textellipsis \allowbreak  tell me\textellipsis \allowbreak  Quick, man!\textellipsis \allowbreak  the very seconds are precious!«

»Will you tell me,« he asked resolutely, and looking searchingly into her blue eyes, »whose hand helped to guide M. Chauvelin to the knowledge which you say he possesses?«

»Mine,« she said quietly, »I own it—I will not lie to you, for I wish you to trust me absolutely. But I had no idea—how \textit{could} I have?—of the identity of the Scarlet Pimpernel\textellipsis \allowbreak  and my brother's safety was to be my prize if I succeeded.«

»In helping Chauvelin to track the Scarlet Pimpernel?«

She nodded.

»It is no use telling you how he forced my hand. Armand is more than a brother to me, and\textellipsis \allowbreak  and\textellipsis \allowbreak  how \textit{could} I guess?\textellipsis \allowbreak  But we waste time, Sir Andrew\textellipsis \allowbreak  every second is precious\textellipsis \allowbreak  in the name of God!\textellipsis \allowbreak  my husband is in peril\textellipsis \allowbreak  your friend!—your comrade!—Help me to save him.«

Sir Andrew felt his position to be a very awkward one. The oath he had taken before his leader and comrade was one of obedience and secrecy; and yet the beautiful woman, who was asking him to trust her, was undoubtedly in earnest; his friend and leader was equally undoubtedly in imminent danger and\textellipsis \allowbreak  »Lady Blakeney,« he said at last, »God knows you have perplexed me, so that I do not know which way my duty lies. Tell me what you wish me to do. There are nineteen of us ready to lay down our lives for the Scarlet Pimpernel if he is in danger.«

»There is no need for lives just now, my friend,« she said drily; »my wits and four swift horses will serve the necessary purpose. But I must know where to find him. See,« she added, while her eyes filled with tears, »I have humbled myself before you, I have owned my fault to you; shall I also confess my weakness?—My husband and I have been estranged, because he did not trust me, and because I was too blind to understand. You must confess that the bandage which he put over my eyes was a very thick one. Is it small wonder that I did not see through it? But last night, after I led him unwittingly into such deadly peril, it suddenly fell from my eyes. If you will not help me, Sir Andrew, I would still strive to save my husband. I would still exert every faculty I possess for his sake; but I might be powerless, for I might arrive too late, and nothing would be left for you but lifelong remorse, and\textellipsis \allowbreak  and\textellipsis \allowbreak  for me, a broken heart.«

»But, Lady Blakeney,« said the young man, touched by the gentle earnestness of this exquisitely beautiful woman, »do you know that what you propose doing is man's work?—you cannot possibly journey to Calais alone. You would be running the greatest possible risks to yourself, and your chances of finding your husband now—were I to direct you ever so carefully—are infinitely remote.«

»Oh, I hope there are risks!« she murmured softly. »I hope there are dangers, too!—I have so much to atone for. But I fear you are mistaken. Chauvelin's eyes are fixed upon you all, he will scarce notice me. Quick, Sir Andrew!—the coach is ready, and there is not a moment to be lost\textellipsis \allowbreak  I \textit{must} get to him! I \textit{must}!« she repeated with almost savage energy, »to warn him that that man is on his track\textellipsis \allowbreak  Can't you see—can't you see, that I \textit{must} get to him\textellipsis \allowbreak  even\textellipsis \allowbreak  even if it be too late to save him\textellipsis \allowbreak  at least\textellipsis \allowbreak  to be by his side\textellipsis \allowbreak  at the last.«

»Faith, Madame, you must command me. Gladly would I or any of my comrades lay down our lives for your husband. If you \textit{will} go yourself\textellipsis«

»Nay, friend, do you not see that I would go mad if I let you go without me?« She stretched out her hand to him. »You \textit{will} trust me?«

»I await your orders,« he said simply.

»Listen, then. My coach is ready to take me to Dover. Do you follow me, as swiftly as horses will take you. We meet at nightfall at »The Fisherman's Rest«. Chauvelin would avoid it, as he is known there, and I think it would be the safest. I will gladly accept your escort to Calais\textellipsis \allowbreak  as you say, I might miss Sir Percy were you to direct me ever so carefully. We'll charter a schooner at Dover and cross over during the night. Disguised, if you will agree to it, as my lacquey, you will, I think, escape detection.«

»I am entirely at your service, Madame,« rejoined the young man earnestly. »I trust to God that you will sight the \textit{Day Dream} before we reach Calais. With Chauvelin at his heels, every step the Scarlet Pimpernel takes on French soil is fraught with danger.«

»God grant it, Sir Andrew. But now, farewell. We meet to-night at Dover! It will be a race between Chauvelin and me across the Channel to-night—and the prize—the life of the Scarlet Pimpernel.«

He kissed her hand, and then escorted her to her chair. A quarter of an hour later she was back at the »Crown« inn, where her coach and horses were ready and waiting for her. The next moment they thundered along the London streets, and then straight on to the Dover road at maddening speed.

She had no time for despair now. She was up and doing and had no leisure to think. With Sir Andrew Ffoulkes as her companion and ally, hope had once again revived in her heart.

God would be merciful. He would not allow so appalling a crime to be committed, as the death of a brave man, through the hand of a woman who loved him, and worshipped him, and who would gladly have died for his sake.

Marguerite's thoughts flew back to him, the mysterious hero, whom she had always unconsciously loved, when his identity was still unknown to her. Laughingly, in the olden days, she used to call him the shadowy king of her heart, and now she had suddenly found that this enigmatic personality whom she had worshipped, and the man who loved her so passionately, were one and the same: what wonder that one or two happier Visions began to force their way before her mind? She vaguely wondered what she would say to him when first they would stand face to face.

She had had so many anxieties, so much excitement during the past few hours, that she allowed herself the luxury of nursing these few more hopeful, brighter thoughts. Gradually the rumble of the coach wheels, with its incessant monotony, acted soothingly on her nerves: her eyes, aching with fatigue and many shed and unshed tears, closed involuntarily, and she fell into a troubled sleep.
