%!TeX root=../scarlettop.tex

\chapter{On the Track}
\lettrine[lines=4]{N}{ever} for a moment did Marguerite Blakeney hesitate. The last sounds outside the »Chat Gris« had died away in the night. She had heard Desgas giving orders to his men, and then starting off towards the fort, to get a reinforcement of a dozen more men: six were not thought sufficient to capture the cunning Englishman, whose resourceful brain was even more dangerous than his valour and his strength.

Then a few minutes later, she heard the Jew's husky voice again, evidently shouting to his nag, then the rumble of wheels, and noise of a rickety cart bumping over the rough road.

Inside the inn, everything was still. Brogard and his wife, terrified of Chauvelin, had given no sign of life; they hoped to be forgotten, and at any rate to remain unperceived: Marguerite could not even hear their usual volleys of muttered oaths.

She waited a moment or two longer, then she quietly slipped down the broken stairs, wrapped her dark cloak closely round her and slipped out of the inn.

The night was fairly dark, sufficiently so at any rate to hide her dark figure from view, whilst her keen ears kept count of the sound of the cart going on ahead. She hoped by keeping well within the shadow of the ditches which lined the road, that she would not be seen by Desgas' men, when they approached, or by the patrols, which she concluded were still on duty.

Thus she started to do this, the last stage of her weary journey, alone, at night, and on foot. Nearly three leagues to Miquelon, and then on to the Père Blanchard's hut, wherever that fatal spot might be, probably over rough roads: she cared not.

The Jew's nag could not get on very fast, and though she was weary with mental fatigue and nerve strain, she knew that she could easily keep up with it, on a hilly road, where the poor beast, who was sure to be half-starved, would have to be allowed long and frequent rests. The road lay some distance from the sea, bordered on either side by shrubs and stunted trees, sparsely covered with meagre foliage, all turning away from the North, with their branches looking in the semi-darkness, like stiff, ghostly hair, blown by a perpetual wind.

Fortunately, the moon showed no desire to peep between the clouds, and Marguerite hugging the edge of the road, and keeping close to the low line of shrubs, was fairly safe from view. Everything around her was so still: only from far, very far away, there came like a long, soft moan, the sound of the distant sea.

The air was keen and full of brine; after that enforced period of inactivity, inside the evil-smelling, squalid inn, Marguerite would have enjoyed the sweet scent of this autumnal night, and the distant melancholy rumble of the waves; she would have revelled in the calm and stillness of this lonely spot, a calm, broken only at intervals by the strident and mournful cry of some distant gull, and by the creaking of the wheels, some way down the road: she would have loved the cool atmosphere, the peaceful immensity of Nature, in this lonely part of the coast: but her heart was too full of cruel foreboding, of a great ache and longing for a being who had become infinitely dear to her.

Her feet slipped on the grassy bank, for she thought it safest not to walk near the centre of the road, and she found it difficult to keep up a sharp pace along the muddy incline. She even thought it best not to keep too near to the cart; everything was so still, that the rumble of the wheels could not fail to be a safe guide.

The loneliness was absolute. Already the few dim lights of Calais lay far behind, and on this road there was not a sign of human habitation, not even the hut of a fisherman or of a woodcutter anywhere near; far away on her right was the edge of the cliff, below it the rough beach, against which the incoming tide was dashing itself with its constant, distant murmur. And ahead the rumble of the wheels, bearing an implacable enemy to his triumph.

Marguerite wondered at what particular spot, on this lonely coast, Percy could be at this moment. Not very far surely, for he had had less than a quarter of an hour's start of Chauvelin. She wondered if he knew that in this cool, ocean-scented bit of France, there lurked many spies, all eager to sight his tall figure, to track him to where his unsuspecting friends waited for him, and then, to close the net over him and them.

Chauvelin, on ahead, jolted and jostled in the Jew's vehicle, was nursing comfortable thoughts. He rubbed his hands together, with content, as he thought of the web which he had woven, and through which that ubiquitous and daring Englishman could not hope to escape. As the time went on, and the old Jew drove him leisurely but surely along the dark road, he felt more and more eager for the grand finale of this exciting chase after the mysterious Scarlet Pimpernel.

The capture of the audacious plotter would be the finest leaf in Citoyen Chauvelin's wreath of glory. Caught, red-handed, on the spot, in the very act of aiding and abetting the traitors against the Republic of France, the Englishman could claim no protection from his own country. Chauvelin had, in any case, fully made up his mind that all intervention should come too late.

Never for a moment did the slightest remorse enter his heart, as to the terrible position in which he had placed the unfortunate wife, who had unconsciously betrayed her husband. As a matter of fact, Chauvelin had ceased even to think of her: she had been a useful tool, that was all.

The Jew's lean nag did little more than walk. She was going along at a slow jog trot, and her driver had to give her long and frequent halts.

»Are we a long way yet from Miquelon?« asked Chauvelin from time to time.

»Not very far, your Honour,« was the uniform placid reply.

»We have not yet come across your friend and mine, lying in a heap in the roadway,« was Chauvelin's sarcastic comment.

»Patience, noble Excellency,« rejoined the son of Moses, »they are ahead of us. I can see the imprint of the cart wheels, driven by that traitor, that son of the Amalekite.«

»You are sure of the road?«

»As sure as I am of the presence of those ten gold pieces in the noble Excellency's pockets, which I trust will presently be mine.«

»As soon as I have shaken hands with my friend the tall stranger, they will certainly be yours.«

»Hark, what was that?« said the Jew suddenly.

Through the stillness, which had been absolute, there could now be heard distinctly the sound of horses' hoofs on the muddy road.

»They are soldiers,« he added in an awed whisper.

»Stop a moment, I want to hear,« said Chauvelin.

Marguerite had also heard the sound of galloping hoofs, coming towards the cart, and towards herself. For some time she had been on the alert thinking that Desgas and his squad would soon overtake them, but these came from the opposite direction, presumably from Miquelon. The darkness lent her sufficient cover. She had perceived that the cart had stopped, and with utmost caution, treading noiselessly on the soft road, she crept a little nearer.

Her heart was beating fast, she was trembling in every limb; already she had guessed what news these mounted men would bring. »Every stranger on these roads or on the beach must be shadowed, especially if he be tall or stoops as if he would disguise his height; when sighted a mounted messenger must at once ride back and report.« Those had been Chauvelin's orders. Had then the tall stranger been sighted, and was this the mounted messenger, come to bring the great news, that the hunted hare had run its head into the noose at last?

Marguerite, realising that the cart had come to a standstill, managed to slip nearer to it in the darkness; she crept close up, hoping to get within earshot, to hear what the messenger had to say.

She heard the quick words of challenge\longdash


»Liberté, Fraternité, Egalité!« then Chauvelin's quick query:\longdash


»What news?«

Two men on horseback had halted beside the vehicle.

Marguerite could see them silhouetted against the midnight sky. She could hear their voices, and the snorting of their horses, and now, behind her, some little distance off, the regular and measured tread of a body of advancing men: Desgas and his soldiers.

There had been a long pause, during which, no doubt, Chauvelin satisfied the men as to his identity, for presently, questions and answers followed each other in quick succession.

»You have seen the stranger?« asked Chauvelin, eagerly.

»No, citoyen, we have seen no tall stranger; we came by the edge of the cliff.«

»Then?«

»Less than a quarter of a league beyond Miquelon, we came across a rough construction of wood, which looked like the hut of a fisherman, where he might keep his tools and nets. When we first sighted it, it seemed to be empty, and at first we thought that there was nothing suspicious about it, until we saw some smoke issuing through an aperture at the side. I dismounted and crept close to it. It was then empty, but in one corner of the hut, there was a charcoal fire, and a couple of stools were also in the hut. I consulted with my comrades, and we decided that they should take cover with the horses, well out of sight, and that I should remain on the watch, which I did.«

»Well! and did you see anything?«

»About half an hour later, I heard voices, citoyen, and presently, two men came along towards the edge of the cliff; they seemed to me to have come from the Lille Road. One was young, the other quite old. They were talking in a whisper, to one another, and I could not hear what they said.«

One was young, the other quite old. Marguerite's aching heart almost stopped beating as she listened: was the young one Armand?\allowbreak---\allowbreak her brother?\allowbreak---\allowbreak and the old one de Tournay\allowbreak---\allowbreak were they the two fugitives who, unconsciously, were used as a decoy, to entrap their fearless and noble rescuer.

»The two men presently went into the hut,« continued the soldier, whilst Marguerite's aching nerves seemed to catch the sound of Chauvelin's triumphant chuckle, »and I crept nearer to it then. The hut is very roughly built, and I caught snatches of their conversation.«

»Yes?\allowbreak---\allowbreak Quick!\allowbreak---\allowbreak What did you hear?«

»The old man asked the young one if he were sure that was the right place. »Oh, yes,« he replied, »'tis the place sure enough,« and by the light of the charcoal fire he showed to his companion a paper, which he carried. »Here is the plan,« he said, »which he gave me before I left London. We were to adhere strictly to that plan, unless I had contrary orders, and I have had none. Here is the road we followed, see... here the fork... here we cut across the St~Martin Road... and here is the footpath which brought us to the edge of the cliff.« I must have made a slight noise then, for the young man came to the door of the hut, and peered anxiously all round him. When he again joined his companion, they whispered so low, that I could no longer hear them.«

»Well?\allowbreak---\allowbreak and?« asked Chauvelin, impatiently.

»There were six of us altogether, patrolling that part of the beach, so we consulted together, and thought it best that four should remain behind and keep the hut in sight, and I and my comrade rode back at once to make report of what we had seen.«

»You saw nothing of the tall stranger?«

»Nothing, citoyen.«

»If your comrades see him, what would they do?«

»Not lose sight of him for a moment, and if he showed signs of escape, or any boat came in sight, they would close in on him, and, if necessary, they would shoot: the firing would bring the rest of the patrol to the spot. In any case they would not let the stranger go.«

»Aye! but I did not want the stranger hurt\allowbreak---\allowbreak not just yet,« murmured Chauvelin, savagely, »but there, you've done your best. The Fates grant that I may not be too late...«

»We met half a dozen men just now, who have been patrolling this road for several hours.«

»Well?«

»They have seen no stranger either.«

»Yet he is on ahead somewhere, in a cart or else... Here! there is not a moment to lose. How far is that hut from here?«

»About a couple of leagues, citoyen.«

»You can find it again?\allowbreak---\allowbreak at once?\allowbreak---\allowbreak without hesitation?«

»I have absolutely no doubt, citoyen.«

»The footpath, to the edge of the cliff?\allowbreak---\allowbreak Even in the dark?«

»It is not a dark night, citoyen, and I know I can find my way,« repeated the soldier firmly.

»Fall in behind then. Let your comrade take both your horses back to Calais. You won't want them. Keep beside the cart, and direct the Jew to drive straight ahead; then stop him, within a quarter of a league of the footpath; see that he takes the most direct road.«

Whilst Chauvelin spoke, Desgas and his men were fast approaching, and Marguerite could hear their footsteps within a hundred yards behind her now. She thought it unsafe to stay where she was, and unnecessary too, as she had heard enough. She seemed suddenly to have lost all faculty even for suffering: her heart, her nerves, her brain seemed to have become numb after all these hours of ceaseless anguish, culminating in this awful despair.

For now there was absolutely not the faintest hope. Within two short leagues of this spot, the fugitives were waiting for their brave deliverer. He was on his way, somewhere on this lonely road, and presently he would join them; then the well-laid trap would close, two dozen men, led by one whose hatred was as deadly as his cunning was malicious, would close round the small band of fugitives, and their daring leader. They would all be captured. Armand, according to Chauvelin's pledged word, would be restored to her, but her husband, Percy, whom with every breath she drew she seemed to love and worship more and more, he would fall into the hands of a remorseless enemy, who had no pity for a brave heart, no admiration for the courage of a noble soul, who would show nothing but hatred for the cunning antagonist, who had baffled him so long.

She heard the soldier giving a few brief directions to the Jew, then she retired quickly to the edge of the road, and cowered behind some low shrubs, whilst Desgas and his men came up.

All fell in noiselessly behind the cart, and slowly they all started down the dark road. Marguerite waited until she reckoned that they were well outside the range of earshot, then, she too in the darkness, which suddenly seemed to have become more intense, crept noiselessly along.