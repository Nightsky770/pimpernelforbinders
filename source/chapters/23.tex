%!TeX root=../scarlettop.tex

\chapter{Hope}
\lettrine[ante=`,lines=4]{F}{aith}, Madame!' said Sir Andrew, seeing that Marguerite seemed desirous to call her surly host back again, »I think we'd better leave him alone. We shall not get anything more out of him, and we might arouse his suspicions. One never knows what spies may be lurking around these God-forsaken places.«

»What care I?« she replied lightly, »now I know that my husband is safe, and that I shall see him almost directly!«

»Hush!« he said in genuine alarm, for she had talked quite loudly, in the fulness of her glee, »the very walls have ears in France, these days.«

He rose quickly from the table, and walked round the bare, squalid room, listening attentively at the door, through which Brogard had just disappeared, and whence only muttered oaths and shuffling footsteps could be heard. He also ran up the rickety steps that led to the attic, to assure himself that there were no spies of Chauvelin's about the place.

»Are we alone, Monsieur, my lacquey?« said Marguerite, gaily, as the young man once more sat down beside her. »May we talk?«

»As cautiously as possible!« he entreated.

»Faith, man! but you wear a glum face! As for me, I could dance with joy! Surely there is no longer any cause for fear. Our boat is on the beach, the \textit{Foam Crest} not two miles out at sea, and my husband will be here, under this very roof, within the next half hour perhaps. Sure! there is naught to hinder us. Chauvelin and his gang have not yet arrived.«

»Nay, madam! that I fear we do not know.«

»What do you mean?«

»He was at Dover at the same time that we were.«

»Held up by the same storm, which kept us from starting.«

»Exactly. But—I did not speak of it before, for I feared to alarm you—I saw him on the beach not five minutes before we embarked. At least, I swore to myself at the time that it was himself; he was disguised as a \textit{curé}, so that Satan, his own guardian, would scarce have known him. But I heard him then, bargaining for a vessel to take him swiftly to Calais; and he must have set sail less than an hour after we did.«

Marguerite's face had quickly lost its look of joy. The terrible danger in which Percy stood, now that he was actually on French soil, became suddenly and horribly clear to her. Chauvelin was close upon his heels; here in Calais, the astute diplomatist was all-powerful; a word from him and Percy could be tracked and arrested and\textellipsis \allowbreak  Every drop of blood seemed to freeze in her veins; not even during the moments of her wildest anguish in England had she so completely realised the imminence of the peril in which her husband stood. Chauvelin had sworn to bring the Scarlet Pimpernel to the guillotine, and now the daring plotter, whose anonymity hitherto had been his safeguard, stood revealed through her own hand, to his most bitter, most relentless enemy.

Chauvelin—when he waylaid Lord Tony and Sir Andrew Ffoulkes in the coffee-room of »The Fisherman's Rest«—had obtained possession of all the plans of this latest expedition. Armand St~Just, the Comte de Tournay and other fugitive royalists were to have met the Scarlet Pimpernel—or rather, as it had been originally arranged, two of his emissaries—on this day, the 2nd of October, at a place evidently known to the league, and vaguely alluded to as the »Père Blanchard's hut.«

Armand, whose connection with the Scarlet Pimpernel and disavowal of the brutal policy of the Reign of Terror was still unknown to his countrymen, had left England a little more than a week ago, carrying with him the necessary instructions, which would enable him to meet the other fugitives and to convey them to this place of safety.

This much Marguerite had fully understood from the first, and Sir Andrew Ffoulkes had confirmed her surmises. She knew, too, that when Sir Percy realised that his own plans and his directions to his lieutenants had been stolen by Chauvelin, it was too late to communicate with Armand, or to send fresh instructions to the fugitives.

They would, of necessity, be at the appointed time and place, not knowing how grave was the danger which now awaited their brave rescuer.

Blakeney, who as usual had planned and organised the whole expedition, would not allow any of his younger comrades to run the risk of almost certain capture. Hence his hurried note to them at Lord Grenville's ball—»Start myself to-morrow—alone.«

And now with his identity known to his most bitter enemy, his every step would be dogged, the moment he set foot in France. He would be tracked by Chauvelin's emissaries, followed until he reached that mysterious hut where the fugitives were waiting for him, and there the trap would be closed on him and on them.

There was but one hour—the hour's start which Marguerite and Sir Andrew had of their enemy—in which to warn Percy of the imminence of his danger, and to persuade him to give up the foolhardy expedition, which could only end in his own death.

But there \textit{was} that one hour.

»Chauvelin knows of this inn, from the papers he stole,« said Sir Andrew, earnestly, »and on landing will make straight for it.«

»He has not landed yet,« she said, »we have an hour's start on him, and Percy will be here directly. We shall be mid-Channel ere Chauvelin has realised that we have slipped through his fingers.«

She spoke excitedly and eagerly, wishing to infuse into her young friend some of that buoyant hope which still clung to her heart. But he shook his head sadly.

»Silent again, Sir Andrew?« she said with some impatience. »Why do you shake your head and look so glum?«

»Faith, Madame,« he replied, »'tis only because in making your rose-coloured plans, you are forgetting the most important factor.«

»What in the world do you mean?—I am forgetting nothing\textellipsis \allowbreak  What factor do you mean?« she added with more impatience.

»It stands six foot odd high,« replied Sir Andrew, quietly, »and hath name Percy Blakeney.«

»I don't understand,« she murmured.

»Do you think that Blakeney would leave Calais without having accomplished what he set out to do?«

»You mean\textellipsis \allowbreak  ?«

»There's the old Comte de Tournay\textellipsis«

»The Comte\textellipsis \allowbreak  ?« she murmured.

»And St~Just\textellipsis \allowbreak  and others\textellipsis«

»My brother!« she said with a heart-broken sob of anguish. »Heaven help me, but I fear I had forgotten.«

»Fugitives as they are, these men at this moment await with perfect confidence and unshaken faith the arrival of the Scarlet Pimpernel, who has pledged his honour to take them safely across the Channel.«

Indeed, she had forgotten! With the sublime selfishness of a woman who loves with her whole heart, she had in the last twenty-four hours had no thought save for him. His precious, noble life, his danger—he, the loved one, the brave hero, he alone dwelt in her mind.

»My brother!« she murmured, as one by one the heavy tears gathered in her eyes, as memory came back to her of Armand, the companion and darling of her childhood, the man for whom she had committed the deadly sin, which had so hopelessly imperilled her brave husband's life.

»Sir Percy Blakeney would not be the trusted, honoured leader of a score of English gentlemen,« said Sir Andrew, proudly, »if he abandoned those who placed their trust in him. As for breaking his word, the very thought is preposterous!«

There was silence for a moment or two. Marguerite had buried her face in her hands, and was letting the tears slowly trickle through her trembling fingers. The young man said nothing; his heart ached for this beautiful woman in her awful grief. All along he had felt the terrible \textit{impasse} in which her own rash act had plunged them all. He knew his friend and leader so well, with his reckless daring, his mad bravery, his worship of his own word of honour. Sir Andrew knew that Blakeney would brave any danger, run the wildest risks sooner than break it, and, with Chauvelin at his very heels, would make a final attempt, however desperate, to rescue those who trusted in him.

»Faith, Sir Andrew,« said Marguerite at last, making brave efforts to dry her tears, »you are right, and I would not now shame myself by trying to dissuade him from doing his duty. As you say, I should plead in vain. God grant him strength and ability,« she added fervently and resolutely, »to outwit his pursuers. He will not refuse to take you with him, perhaps, when he starts on his noble work; between you, you will have cunning as well as valour! God guard you both! In the meanwhile I think we should lose no time. I still believe that his safety depends upon his knowing that Chauvelin is on his track.«

»Undoubtedly. He has wonderful resources at his command. As soon as he is aware of his danger he will exercise more caution: his ingenuity is a veritable miracle.«

»Then, what say you to a voyage of reconnaissance in the village whilst I wait here against his coming!—You might come across Percy's track and thus save valuable time. If you find him, tell him to beware!—his bitterest enemy is on his heels!«

»But this is such a villainous hole for you to wait in.«

»Nay, that I do not mind!—But you might ask our surly host if he could let me wait in another room, where I could be safer from the prying eyes of any chance traveller. Offer him some ready money, so that he should not fail to give me word the moment the tall Englishman returns.«

She spoke quite calmly, even cheerfully now, thinking out her plans, ready for the worst if need be; she would show no more weakness, she would prove herself worthy of him, who was about to give his life for the sake of his fellow-men.

Sir Andrew obeyed her without further comment. Instinctively he felt that hers now was the stronger mind; he was willing to give himself over to her guidance, to become the hand, whilst she was the directing head.

He went to the door of the inner room, through which Brogard and his wife had disappeared before, and knocked; as usual, he was answered by a salvo of muttered oaths.

»Hey! friend Brogard!« said the young man peremptorily, »my lady would wish to rest here awhile. Could you give her the use of another room? She would wish to be alone.«

He took some money out of his pocket, and allowed it to jingle significantly in his hand. Brogard had opened the door, and listened, with his usual surly apathy, to the young man's request. At sight of the gold, however, his lazy attitude relaxed slightly; he took his pipe from his mouth and shuffled into the room.

He then pointed over his shoulder at the attic up in the wall.

»She can wait up there!« he said with a grunt. »It's comfortable, and I have no other room.«

»Nothing could be better,« said Marguerite in English; she at once realised the advantages such a position hidden from view would give her. »Give him the money, Sir Andrew; I shall be quite happy up there, and can see everything without being seen.«

She nodded to Brogard, who condescended to go up to the attic, and to shake up the straw that lay on the floor.

»May I entreat you, madam, to do nothing rash,« said Sir Andrew, as Marguerite prepared in her turn to ascend the rickety flight of steps. »Remember this place is infested with spies. Do not, I beg of you, reveal yourself to Sir Percy, unless you are absolutely certain that you are alone with him.«

Even as he spoke, he felt how unnecessary was this caution: Marguerite was as calm, as clear-headed as any man. There was no fear of her doing anything that was rash.

»Nay,« she said with a slight attempt at cheerfulness, »that can I faithfully promise you. I would not jeopardise my husband's life, nor yet his plans, by speaking to him before strangers. Have no fear, I will watch my opportunity, and serve him in the manner I think he needs it most.«

Brogard had come down the steps again, and Marguerite was ready to go up to her safe retreat.

»I dare not kiss your hand, madam,« said Sir Andrew, as she began to mount the steps, »since I am your lacquey, but I pray you be of good cheer. If I do not come across Blakeney in half an hour, I shall return, expecting to find him here.«

»Yes, that will be best. We can afford to wait for half an hour. Chauvelin cannot possibly be here before that. God grant that either you or I may have seen Percy by then. Good luck to you, friend! Have no fear for me.«

Lightly she mounted the rickety wooden steps that led to the attic. Brogard was taking no further heed of her. She could make herself comfortable there or not as she chose. Sir Andrew watched her until she had reached the loft and sat down upon the straw. She pulled the tattered curtains across, and the young man noted that she was singularly well placed there, for seeing and hearing, whilst remaining unobserved.

He had paid Brogard well; the surly old innkeeper would have no object in betraying her. Then Sir Andrew prepared to go. At the door he turned once again and looked up at the loft. Through the ragged curtains Marguerite's sweet face was peeping down at him, and the young man rejoiced to see that it looked serene, and even gently smiling. With a final nod of farewell to her, he walked out into the night.