%!TeX root=../scarlettop.tex

\chapter{The Secret Orchard}
\lettrine[lines=4]{O}{nce} outside the noisy coffee-room, alone in the dimly-lighted passage, Marguerite Blakeney  seemed to breathe more freely. She heaved a deep sigh, like one who had long been oppressed with the heavy weight of constant self-control, and she allowed a few tears to fall unheeded down her cheeks.

Outside the rain had ceased, and through the swiftly passing clouds, the pale rays of an after-storm sun shone upon the beautiful white coast of Kent and the quaint, irregular houses that clustered round the Admiralty Pier. Marguerite Blakeney stepped on to the porch and looked out to sea. Silhouetted against the ever-changing sky, a graceful schooner, with white sails set, was gently dancing in the breeze. The \textit{Day Dream} it was, Sir Percy Blakeney's yacht, which was ready to take Armand St~Just back to France into the very midst of that seething, bloody Revolution which was overthrowing a monarchy, attacking a religion, destroying a society, in order to try and rebuild upon the ashes of tradition a new Utopia, of which a few men dreamed, but which none had the power to establish.

In the distance two figures were approaching »The Fisherman's Rest«: one, an oldish man, with a curious fringe of grey hairs round a rotund and massive chin, and who walked with that peculiar rolling gait which invariably betrays the seafaring man: the other, a young, slight figure, neatly and becomingly dressed in a dark, many-caped overcoat; he was clean-shaved, and his dark hair was taken well back over a clear and noble forehead.

»Armand!« said Marguerite Blakeney, as soon as she saw him approaching from the distance, and a happy smile shone on her sweet face, even through the tears.

A minute or two later brother and sister were locked in each other's arms, while the old skipper stood respectfully on one side.

»How much time have we got, Briggs?« asked Lady Blakeney, »before M. St~Just need go on board?«

»We ought to weigh anchor before half an hour, your ladyship,« replied the old man, pulling at his grey forelock.

Linking her arm in his, Marguerite led her brother towards the cliffs.

»Half an hour,« she said, looking wistfully out to sea, »half an hour more and you'll be far from me, Armand! Oh! I can't believe that you are going, dear! These last few days—whilst Percy has been away, and I've had you all to myself, have slipped by like a dream.«

»I am not going far, sweet one,« said the young man gently, »a narrow channel to cross—a few miles of road—I can soon come back.«

»Nay, 'tis not the distance, Armand—but that awful Paris\textellipsis \allowbreak  just now\textellipsis«

They had reached the edge of the cliff. The gentle sea-breeze blew Marguerite's hair about her face, and sent the ends of her soft lace fichu waving round her, like a white and supple snake. She tried to pierce the distance far away, beyond which lay the shores of France: that relentless and stern France which was exacting her pound of flesh, the blood-tax from the noblest of her sons.

»Our own beautiful country, Marguerite,« said Armand, who seemed to have divined her thoughts.

»They are going too far, Armand,« she said vehemently. »You are a republican, so am I\textellipsis \allowbreak  we have the same thoughts, the same enthusiasm for liberty and equality\textellipsis \allowbreak  but even \textit{you} must think that they are going too far\textellipsis«

»Hush!\longdash« said Armand, instinctively, as he threw a quick, apprehensive glance around him.

»Ah! you see: you don't think yourself that it is safe even to speak of these things—here in England!« She clung to him suddenly with strong, almost motherly, passion: »Don't go, Armand!« she begged; »don't go back! What should I do if\textellipsis \allowbreak  if\textellipsis \allowbreak  if\textellipsis«

Her voice was choked in sobs, her eyes, tender, blue and loving, gazed appealingly at the young man, who in his turn looked steadfastly into hers.

»You would in any case be my own brave sister,« he said gently, »who would remember that, when France is in peril, it is not for her sons to turn their backs on her.«

Even as he spoke, that sweet, childlike smile crept back into her face, pathetic in the extreme, for it seemed drowned in tears.

»Oh! Armand!« she said quaintly, »I sometimes wish you had not so many lofty virtues\textellipsis \allowbreak  I assure you little sins are far less dangerous and uncomfortable. But you \textit{will} be prudent?« she added earnestly.

»As far as possible\textellipsis \allowbreak  I promise you.«

»Remember, dear, I have only you\textellipsis \allowbreak  to\textellipsis \allowbreak  to care for me\textellipsis«

»Nay, sweet one, you have other interests now. Percy cares for you\textellipsis«

A look of strange wistfulness crept into her eyes as she murmured,\longdash


»He did\textellipsis \allowbreak  once\textellipsis«

»But surely\textellipsis«

»There, there, dear, don't distress yourself on my account. Percy is very good\textellipsis«

»Nay!« he interrupted energetically, »I will distress myself on your account, my Margot. Listen, dear, I have not spoken of these things to you before; something always seemed to stop me when I wished to question you. But, somehow, I feel as if I could not go away and leave you now without asking you one question\textellipsis \allowbreak  You need not answer it if you do not wish,« he added, as he noted a sudden hard look, almost of apprehension, darting through her eyes.

»What is it?« she asked simply.

»Does Sir Percy Blakeney know that\textellipsis \allowbreak  I mean, does he know the part you played in the arrest of the Marquis de St~Cyr?«

She laughed—a mirthless, bitter, contemptuous laugh, which was like a jarring chord in the music of her voice.

»That I denounced the Marquis de St~Cyr, you mean, to the tribunal that ultimately sent him and all his family to the guillotine? Yes, he does know\textellipsis \allowbreak  I told him after I married him\textellipsis«

»You told him all the circumstances—which so completely exonerated you from any blame?«

»It was too late to talk of »circumstances«; he heard the story from other sources; my confession came too tardily, it seems. I could no longer plead extenuating circumstances: I could not bemean myself by trying to explain\longdash«

»And?«

»And now I have the satisfaction, Armand, of knowing that the biggest fool in England has the most complete contempt for his wife.«

She spoke with vehement bitterness this time, and Armand St~Just, who loved her so dearly, felt that he had placed a somewhat clumsy finger upon an aching wound.

»But Sir Percy loved you, Margot,« he repeated gently.

»Loved me?—Well, Armand, I thought at one time that he did, or I should not have married him. I daresay,« she added, speaking very rapidly, as if she were glad at last to lay down a heavy burden, which had oppressed her for months, »I daresay that even you thought—as everybody else did—that I married Sir Percy because of his wealth—but I assure you, dear, that it was not so. He seemed to worship me with a curious intensity of concentrated passion, which went straight to my heart. I had never loved anyone before, as you know, and I was four-and-twenty then—so I naturally thought that it was not in my nature to love. But it has always seemed to me that it \textit{must} be \textit{heavenly} to be loved blindly, passionately, wholly\textellipsis \allowbreak  worshipped, in fact—and the very fact that Percy was slow and stupid was an attraction for me, as I thought he would love me all the more. A clever man would naturally have other interests, an ambitious man other hopes\textellipsis \allowbreak  I thought that a fool would worship, and think of nothing else. And I was ready to respond, Armand; I would have allowed myself to be worshipped, and given infinite tenderness in return\textellipsis«

She sighed—and there was a world of disillusionment in that sigh. Armand St~Just had allowed her to speak on without interruption: he listened to her, whilst allowing his own thoughts to run riot. It was terrible to see a young and beautiful woman—a girl in all but name—still standing almost at the threshold of her life, yet bereft of hope, bereft of illusions, bereft of those golden and fantastic dreams, which should have made her youth one long, perpetual holiday.

Yet perhaps—though he loved his sister dearly—perhaps he understood: he had studied men in many countries, men of all ages, men of every grade of social and intellectual status, and inwardly he understood what Marguerite had left unsaid. Granted that Percy Blakeney was dull-witted, but in his slow-going mind, there would still be room for that ineradicable pride of a descendant of a long line of English gentlemen. A Blakeney had died on Bosworth Field, another had sacrificed life and fortune for the sake of a treacherous Stuart: and that same pride—foolish and prejudiced as the republican Armand would call it—must have been stung to the quick on hearing of the sin which lay at Lady Blakeney's door. She had been young, misguided, ill-advised perhaps. Armand knew that: and those who took advantage of Marguerite's youth, her impulses and imprudence, knew it still better; but Blakeney was slow-witted, he would not listen to »circumstances,« he only clung to facts, and these had shown him Lady Blakeney denouncing a fellow-man to a tribunal that knew no pardon: and the contempt he would feel for the deed she had done, however unwittingly, would kill that same love in him, in which sympathy and intellectuality could never have had a part.

Yet even now, his own sister puzzled him. Life and love have such strange vagaries. Could it be that with the waning of her husband's love, Marguerite's heart had awakened with love for him? Strange extremes meet in love's pathway: this woman, who had had half intellectual Europe at her feet, might perhaps have set her affections on a fool. Marguerite was gazing out towards the sunset. Armand could not see her face, but presently it seemed to him that something which glittered for a moment in the golden evening light, fell from her eyes onto her dainty fichu of lace.

But he could not broach that subject with her. He knew her strange, passionate nature so well, and knew that reserve which lurked behind her frank, open ways.

They had always been together, these two, for their parents had died when Armand was still a youth, and Marguerite but a child. He, some eight years her senior, had watched over her until her marriage; had chaperoned her during those brilliant years spent in the flat of the Rue de Richelieu, and had seen her enter upon this new life of hers, here in England, with much sorrow and some foreboding.

This was his first visit to England since her marriage, and the few months of separation had already seemed to have built up a slight, thin partition between brother and sister; the same deep, intense love was still there, on both sides, but each now seemed to have a secret orchard, into which the other dared not penetrate.

There was much Armand St~Just could not tell his sister; the political aspect of the revolution in France was changing almost every day; she might not understand how his own views and sympathies might become modified, even as the excesses, committed by those who had been his friends, grew in horror and in intensity. And Marguerite could not speak to her brother about the secrets of her heart; she hardly understood them herself, she only knew that, in the midst of luxury, she felt lonely and unhappy.

And now Armand was going away; she feared for his safety, she longed for his presence. She would not spoil these last few sadly-sweet moments by speaking about herself. She led him gently along the cliffs, then down to the beach; their arms linked in one another's, they had still so much to say that lay just outside that secret orchard of theirs.
