%!TeX root=../scarlettop.tex

\chapter{Dover: »The Fisherman's Rest«}
\lettrine[lines=4]{I}{n} the kitchen Sally was extremely busy—sauce\-pans and frying-pans were standing in rows on the gigantic hearth, the huge stock-pot stood in a corner, and the jack turned with slow deliberation, and presented alternately to the glow every side of a noble sirloin of beef. The two little kitchen-maids bustled around, eager to help, hot and panting, with cotton sleeves well tucked up above the dimpled elbows, and giggling over some private jokes of their own, whenever Miss Sally's back was turned for a moment. And old Jemima, stolid in temper and solid in bulk, kept up a long and subdued grumble, while she stirred the stock-pot methodically over the fire.

»What ho! Sally!« came in cheerful if none too melodious accents from the coffee-room close by.

»Lud bless my soul!« exclaimed Sally, with a good-humoured laugh, »what be they all wanting now, I wonder!«

»Beer, of course,« grumbled Jemima, »you don't 'xpect Jimmy Pitkin to 'ave done with one tankard, do ye?«

»Mr 'Arry, 'e looked uncommon thirsty too,« simpered Martha, one of the little kitchen-maids; and her beady black eyes twinkled as they met those of her companion, whereupon both started on a round of short and suppressed giggles.

Sally looked cross for a moment, and thoughtfully rubbed her hands against her shapely hips; her palms were itching, evidently, to come in contact with Martha's rosy cheeks—but inherent good-humour prevailed, and with a pout and a shrug of the shoulders, she turned her attention to the fried potatoes.

»What ho, Sally! hey, Sally!«

And a chorus of pewter mugs, tapped with impatient hands against the oak tables of the coffee-room, accompanied the shouts for mine host's buxom daughter.

»Sally!« shouted a more persistent voice, »are ye goin' to be all night with that there beer?«

»I do think father might get the beer for them,« muttered Sally, as Jemima, stolidly and without further comment, took a couple of foam-crowned jugs from the shelf, and began filling a number of pewter tankards with some of that home-brewed ale for which »The Fisherman's Rest« had been famous since the days of King Charles. »'E knows 'ow busy we are in 'ere.«

»Your father is too busy discussing politics with Mr 'Empseed to worry 'isself about you and the kitchen,« grumbled Jemima under her breath.

Sally had gone to the small mirror which hung in a corner of the kitchen, and was hastily smoothing her hair and setting her frilled cap at its most becoming angle over her dark curls; then she took up the tankards by their handles, three in each strong, brown hand, and laughing, grumbling, blushing, carried them through into the coffee-room.

There, there was certainly no sign of that bustle and activity which kept four women busy and hot in the glowing kitchen beyond.

The coffee-room of »The Fisherman's Rest« is a show place now at the beginning of the twentieth century. At the end of the eighteenth, in the year of grace 1792, it had not yet gained that notoriety and importance which a hundred additional years and the craze of the age have since bestowed upon it. Yet it was an old place, even then, for the oak rafters and beams were already black with age—as were the panelled seats, with their tall backs, and the long polished tables between, on which innumerable pewter tankards had left fantastic patterns of many-sized rings. In the leaded window, high up, a row of pots of scarlet geraniums and blue larkspur gave the bright note of colour against the dull background of the oak.

That Mr Jellyband, landlord of »The Fisherman's Rest« at Dover, was a prosperous man, was of course clear to the most casual observer. The pewter on the fine old dressers, the brass above the gigantic hearth, shone like silver and gold—the red-tiled floor was as brilliant as the scarlet geranium on the window sill—this meant that his servants were good and plentiful, that the custom was constant, and of that order which necessitated the keeping up of the coffee-room to a high standard of elegance and order.

As Sally came in, laughing through her frowns, and displaying a row of dazzling white teeth, she was greeted with shouts and chorus of applause.

»Why, here's Sally! What ho, Sally! Hurrah for pretty Sally!«

»I thought you'd grown deaf in that kitchen of yours,« muttered Jimmy Pitkin, as he passed the back of his hand across his very dry lips.

»All ri'! all ri'!« laughed Sally, as she deposited the freshly-filled tankards upon the tables, »why, what a 'urry, to be sure! And is your gran'mother a-dyin' an' you wantin' to see the pore soul afore she'm gone! I never see'd such a mighty rushin'!«

A chorus of good-humoured laughter greeted this witticism, which gave the company there present food for many jokes, for some considerable time. Sally now seemed in less of a hurry to get back to her pots and pans. A young man with fair curly hair, and eager, bright blue eyes, was engaging most of her attention and the whole of her time, whilst broad witticisms anent Jimmy Pitkin's fictitious grandmother flew from mouth to mouth, mixed with heavy puffs of pungent tobacco smoke.

Facing the hearth, his legs wide apart, a long clay pipe in his mouth, stood mine host himself, worthy Mr Jellyband, landlord of »The Fisherman's Rest,« as his father had been before him, aye, and his grandfather and great-grandfather too, for that matter. Portly in build, jovial in countenance and somewhat bald of pate, Mr Jellyband was indeed a typical rural John Bull of those days—the days when our prejudiced insularity was at its height, when to an Englishman, be he lord, yeoman, or peasant, the whole of the continent of Europe was a den of immorality, and the rest of the world an unexploited land of savages and cannibals.

There he stood, mine worthy host, firm and well set up on his limbs, smoking his long churchwarden and caring nothing for nobody at home, and despising everybody abroad. He wore the typical scarlet waistcoat, with shiny brass buttons, the corduroy breeches, the grey worsted stockings and smart buckled shoes, that characterised every self-respecting innkeeper in Great Britain in these days—and while pretty, motherless Sally had need of four pairs of brown hands to do all the work that fell on her shapely shoulders, worthy Jellyband discussed the affairs of nations with his most privileged guests.

The coffee-room indeed, lighted by two well-polished lamps, which hung from the raftered ceiling, looked cheerful and cosy in the extreme. Through the dense clouds of tobacco smoke that hung about in every corner, the faces of Mr Jellyband's customers appeared red and pleasant to look at, and on good terms with themselves, their host and all the world; from every side of the room loud guffaws accompanied pleasant, if not highly intellectual, conversation—while Sally's repeated giggles testified to the good use Mr Harry Waite was making of the short time she seemed inclined to spare him.

They were mostly fisher-folk who patronised Mr Jellyband's coffee-room, but fishermen are known to be very thirsty people; the salt which they breathe in, when they are on the sea, accounts for their parched throats when on shore. But »The Fisherman's Rest« was something more than a rendezvous for these humble folk. The London and Dover coach started from the hostel daily, and passengers who had come across the Channel, and those who started for the »grand tour,« all became acquainted with Mr Jellyband, his French wines and his home-brewed ales.

It was towards the close of September, 1792, and the weather which had been brilliant and hot throughout the month had suddenly broken up; for two days torrents of rain had deluged the south of England, doing its level best to ruin what chances the apples and pears and late plums had of becoming really fine, self-respecting fruit. Even now it was beating against the leaded windows, and tumbling down the chimney, making the cheerful wood fire sizzle in the hearth.

»Lud! did you ever see such a wet September, Mr Jellyband?« asked Mr Hempseed.

He sat in one of the seats inside the hearth, did Mr Hempseed, for he was an authority and an important personage not only at »The Fisherman's Rest,« where Mr Jellyband always made a special selection of him as a foil for political arguments, but throughout the neighbourhood, where his learning and notably his knowledge of the Scriptures was held in the most profound awe and respect. With one hand buried in the capacious pockets of his corduroys underneath his elaborately-worked, well-worn smock, the other holding his long clay pipe, Mr Hempseed sat there looking dejectedly across the room at the rivulets of moisture which trickled down the window panes.

»No,« replied Mr Jellyband, sententiously, »I dunno, Mr 'Emp\-seed, as I ever did. An' I've been in these parts nigh on sixty years.«

»Aye! you wouldn't rec'llect the first three years of them sixty, Mr Jellyband,« quietly interposed Mr Hempseed. »I dunno as I ever see'd an infant take much note of the weather, leastways not in these parts, an' \textit{I}'ve lived 'ere nigh on seventy-five years, Mr Jellyband.«

The superiority of this wisdom was so incontestable that for the moment Mr Jellyband was not ready with his usual flow of argument.

»It do seem more like April than September, don't it?« continued Mr Hempseed, dolefully, as a shower of raindrops fell with a sizzle upon the fire.

»Aye! that it do,« assented the worthy host, »but then what can you 'xpect, Mr 'Empseed, I says, with sich a government as we've got?«

Mr Hempseed shook his head with an infinity of wisdom, tempered by deeply-rooted mistrust of the British climate and the British Government.

»I don't 'xpect nothing, Mr Jellyband,« he said. »Pore folks like us is of no account up there in Lunnon, I knows that, and it's not often as I do complain. But when it comes to sich wet weather in September, and all me fruit a-rottin' and a-dyin' like the 'Guptian mother's first-born, and doin' no more good than they did, pore dears, save to a lot of Jews, pedlars and sich, with their oranges and sich like foreign ungodly fruit, which nobody'd buy if English apples and pears was nicely swelled. As the Scriptures say\longdash«

»That's quite right, Mr 'Empseed,« retorted Jellyband, »and as I says, what can you 'xpect? There's all them Frenchy devils over the Channel yonder a-murderin' their king and nobility, and Mr Pitt and Mr Fox and Mr Burke a-fightin' and a-wranglin' between them, if we Englishmen should 'low them to go on in their ungodly way. »Let 'em murder!« says Mr Pitt. »Stop 'em!« says Mr Burke.«

»And let 'em murder, says I, and be demmed to 'em,« said Mr Hempseed, emphatically, for he had but little liking for his friend Jellyband's political arguments, wherein he always got out of his depth, and had but little chance for displaying those pearls of wisdom which had earned for him so high a reputation in the neighbourhood and so many free tankards of ale at »The Fisherman's Rest.«

»Let 'em murder,« he repeated again, »but don't let's 'ave sich rain in September, for that is agin the law and the Scriptures which says\longdash«

»Lud! Mr 'Arry, 'ow you made me jump!«

It was unfortunate for Sally and her flirtation that this remark of hers should have occurred at the precise moment when Mr Hempseed was collecting his breath, in order to deliver himself of one of those Scriptural utterances which had made him famous, for it brought down upon her pretty head the full flood of her father's wrath.

»Now then, Sally, me girl, now then!« he said, trying to force a frown upon his good-humoured face, »stop that fooling with them young jackanapes and get on with the work.«

»The work's gettin' on all ri', father.«

But Mr Jellyband was peremptory. He had other views for his buxom daughter, his only child, who would in God's good time become the owner of »The Fisherman's Rest,« than to see her married to one of these young fellows who earned but a precarious livelihood with their net.

»Did ye hear me speak, me girl?« he said in that quiet tone, which no one inside the inn dared to disobey. »Get on with my Lord Tony's supper, for, if it ain't the best we can do, and 'e not satisfied, see what you'll get, that's all.«

Reluctantly Sally obeyed.

»Is you 'xpecting special guests then to-night, Mr Jellyband?« asked Jimmy Pitkin, in a loyal attempt to divert his host's attention from the circumstances connected with Sally's exit from the room.

»Aye! that I be,« replied Jellyband, »friends of my Lord Tony hisself. Dukes and duchesses from over the water yonder, whom the young lord and his friend, Sir Andrew Ffoulkes, and other young noblemen have helped out of the clutches of them murderin' devils.«

But this was too much for Mr Hempseed's querulous philosophy.

»Lud!« he said, »what they do that for, I wonder? I don't 'old not with interferin' in other folks' ways. As the Scriptures say\longdash«

»Maybe, Mr 'Empseed,« interrupted Jellyband, with biting sarcasm, »as you're a personal friend of Mr Pitt, and as you says along with Mr Fox: »Let 'em murder!« says you.«

»Pardon me, Mr Jellyband,« feebly protested Mr Hempseed, »I dunno as I ever did.«

But Mr Jellyband had at last succeeded in getting upon his favourite hobby-horse, and had no intention of dismounting in any hurry.

»Or maybe you've made friends with some of them French chaps 'oo they do say have come over here o' purpose to make us Englishmen agree with their murderin' ways.«

»I dunno what you mean, Mr Jellyband,« suggested Mr Hempseed, »all I know is\longdash«

»All \textit{I} know is,« loudly asserted mine host, »that there was my friend Peppercorn, 'oo owns the »Blue-Faced Boar«, an' as true and loyal an Englishman as you'd see in the land. And now look at 'im!—'E made friends with some o' them frog-eaters, 'obnobbed with them just as if they was Englishmen, and not just a lot of immoral, God-forsaking furrin' spies. Well! and what happened? Peppercorn 'e now ups and talks of revolutions, and liberty, and down with the aristocrats, just like Mr 'Empseed over 'ere!«

»Pardon me, Mr Jellyband,« again interposed Mr Hempseed, feebly, »I dunno as I ever did\longdash«

Mr Jellyband had appealed to the company in general, who were listening awe-struck and open-mouthed at the recital of Mr Peppercorn's defalcations. At one table two customers—gentlemen apparently by their clothes—had pushed aside their half-finished game of dominoes, and had been listening for some time, and evidently with much amusement at Mr Jellyband's international opinions. One of them now, with a quiet, sarcastic smile still lurking round the corners of his mobile mouth, turned towards the centre of the room where Mr Jellyband was standing.

»You seem to think, mine honest friend,« he said quietly, »that these Frenchmen—spies I think you called them—are mighty clever fellows to have made mincemeat so to speak of your friend Mr Peppercorn's opinions. How did they accomplish that now, think you?«

»Lud! sir, I suppose they talked 'im over. Those Frenchies, I've 'eard it said, 'ave got the gift of gab—and Mr 'Empseed 'ere will tell you 'ow it is that they just twist some people round their little finger like.«

»Indeed, and is that so, Mr Hempseed?« inquired the stranger politely.

»Nay, sir!« replied Mr Hempseed, much irritated, »I dunno as I can give you the information you require.«

»Faith, then,« said the stranger, »let us hope, my worthy host, that these clever spies will not succeed in upsetting your extremely loyal opinions.«

But this was too much for Mr Jellyband's pleasant equanimity. He burst into an uproarious fit of laughter, which was soon echoed by those who happened to be in his debt.

»Hahaha! hohoho! hehehe!« He laughed in every key, did my worthy host, and laughed until his sides ached, and his eyes streamed. »At me! hark at that! Did ye 'ear 'im say that they'd be upsettin' my opinions?—Eh?—Lud love you, sir, but you do say some queer things.«

»Well, Mr Jellyband,« said Mr Hempseed, sententiously, »you know what the Scriptures say: »Let 'im 'oo stands take 'eed lest 'e fall.««

»But then hark'ee, Mr 'Empseed,« retorted Jellyband, still holding his sides with laughter, »the Scriptures didn't know me. Why, I wouldn't so much as drink a glass of ale with one o' them murderin' Frenchmen, and nothin' 'd make me change my opinions. Why! I've 'eard it said that them frog-eaters can't even speak the King's English, so, of course, if any of 'em tried to speak their God-forsaken lingo to me, why, I should spot them directly, see!—and forewarned is forearmed, as the saying goes.«

»Aye! my honest friend,« assented the stranger cheerfully, »I see that you are much too sharp, and a match for any twenty Frenchmen, and here's to your very good health, my worthy host, if you'll do me the honour to finish this bottle of mine with me.«

»I am sure you're very polite, sir,« said Mr Jellyband, wiping his eyes which were still streaming with the abundance of his laughter, »and I don't mind if I do.«

The stranger poured out a couple of tankards full of wine, and having offered one to mine host, he took the other himself.

»Loyal Englishmen as we all are,« he said, whilst the same humorous smile played round the corners of his thin lips—»loyal as we are, we must admit that this at least is one good thing which comes to us from France.«

»Aye! we'll none of us deny that, sir,« assented mine host.

»And here's to the best landlord in England, our worthy host, Mr Jellyband,« said the stranger in a loud tone of voice.

»Hip, hip, hurrah!« retorted the whole company present. Then there was loud clapping of hands, and mugs and tankards made a rattling music upon the tables to the accompaniment of loud laughter at nothing in particular, and of Mr Jellyband's muttered exclamations:

»Just fancy \textit{me} bein' talked over by any God-forsaken furriner! —What?—Lud love you, sir, but you do say some queer things.«

To which obvious fact the stranger heartily assented. It was certainly a preposterous suggestion that anyone could ever upset Mr Jellyband's firmly-rooted opinions anent the utter worthlessness of the inhabitants of the whole continent of Europe.