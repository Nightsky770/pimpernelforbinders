%!TeX root=../scarlettop.tex

\chapter{The Scrap of Paper}
\lettrine[lines=4]{M}{arguerite} suffered intensely. Though she laughed and chatted, though she was more admired, more surrounded, more \textit{fêted} than any woman there, she felt like one condemned to death, living her last day upon this earth.

Her nerves were in a state of painful tension, which had increased a hundredfold during that brief hour which she had spent in her husband's company, between the opera and the ball. The short ray of hope\allowbreak---\allowbreak that she might find in this good-natured, lazy individual a valuable friend and adviser\allowbreak---\allowbreak had vanished as quickly as it had come, the moment she found herself alone with him. The same feeling of good-humoured contempt which one feels for an animal or a faithful servant, made her turn away with a smile from the man who should have been her moral support in this heart-rending crisis through which she was passing: who should have been her cool-headed adviser, when feminine sympathy and sentiment tossed her hither and thither, between her love for her brother, who was far away and in mortal peril, and horror of the awful service which Chauvelin had exacted from her, in exchange for Armand's safety.

There he stood, the moral support, the cool-headed adviser, surrounded by a crowd of brainless, empty-headed young fops, who were even now repeating from mouth to mouth, and with every sign of the keenest enjoyment, a doggerel quatrain which he had just given forth.

Everywhere the absurd, silly words met her: people seemed to have little else to speak about, even the Prince had asked her, with a laugh, whether she appreciated her husband's latest poetic efforts.

»All done in the tying of a cravat,« Sir Percy had declared to his clique of admirers.

\settowidth{\versewidth}{We seek him here, we seek him there,}
\begin{verse}[\versewidth]
\begin{altverse}
We seek him here, we seek him there,\\
Those Frenchies seek him everywhere.\\
Is he in heaven?\allowbreak---\allowbreak Is he in hell?\\
That demmed, elusive Pimpernel?
\end{altverse}
\end{verse}


Sir Percy's \textit{bon mot} had gone the round of the brilliant reception-rooms. The Prince was enchanted. He vowed that life without Blakeney would be but a dreary desert. Then, taking him by the arm, had led him to the card-room, and engaged him in a long game of hazard.

Sir Percy, whose chief interest in most social gatherings seemed to centre round the card-table, usually allowed his wife to flirt, dance, to amuse or bore herself as much as she liked. And to-night, having delivered himself of his \textit{bon mot}, he had left Marguerite surrounded by a crowd of admirers of all ages, all anxious and willing to help her to forget that somewhere in the spacious reception-rooms, there was a long, lazy being who had been fool enough to suppose that the cleverest woman in Europe would settle down to the prosaic bonds of English matrimony.

Her still overwrought nerves, her excitement and agitation, lent beautiful Marguerite Blakeney much additional charm: escorted by a veritable bevy of men of all ages and of most nationalities, she called forth many exclamations of admiration from everyone as she passed.

She would not allow herself any more time to think. Her early, somewhat Bohemian training had made her something of a fatalist. She felt that events would shape themselves, that the directing of them was not in her hands. From Chauvelin she knew that she could expect no mercy. He had set a price upon Armand's head, and left it to her to pay or not, as she chose.

Later on in the evening she caught sight of Sir Andrew Ffoulkes and Lord Antony Dewhurst, who seemingly had just arrived. She noticed at once that Sir Andrew immediately made for little Suzanne de Tournay, and that the two young people soon managed to isolate themselves in one of the deep embrasures of the mullioned windows, there to carry on a long conversation, which seemed very earnest and very pleasant on both sides.

Both the young men looked a little haggard and anxious, but otherwise they were irreproachably dressed, and there was not the slightest sign, about their courtly demeanour, of the terrible catastrophe, which they must have felt hovering round them and round their chief.

That the League of the Scarlet Pimpernel had no intention of abandoning its cause, she had gathered through little Suzanne herself, who spoke openly of the assurance she and her mother had had that the Comte de Tournay would be rescued from France by the league, within the next few days. Vaguely she began to wonder, as she looked at the brilliant and fashionable crowd in the gaily-lighted ball-room, which of these worldly men round her was the mysterious »Scarlet Pimpernel,« who held the threads of such daring plots, and the fate of valuable lives in his hands.

A burning curiosity seized her to know him: although for months she had heard of him and had accepted his anonymity, as everyone else in society had done; but now she longed to know\allowbreak---\allowbreak quite impersonally, quite apart from Armand, and oh! quite apart from Chauvelin\allowbreak---\allowbreak only for her own sake, for the sake of the enthusiastic admiration she had always bestowed on his bravery and cunning.

He was at the ball, of course, somewhere, since Sir Andrew Ffoulkes and Lord Antony Dewhurst were here, evidently expecting to meet their chief\allowbreak---\allowbreak and perhaps to get a fresh \textit{mot d'ordre} from him.

Marguerite looked round at everyone, at the aristocratic high-typed Norman faces, the squarely-built, fair-haired Saxon, the more gentle, humorous caste of the Celt, wondering which of these betrayed the power, the energy, the cunning which had imposed its will and its leadership upon a number of high-born English gentlemen, among whom rumour asserted was His Royal Highness himself.

Sir Andrew Ffoulkes? Surely not, with his gentle blue eyes, which were looking so tenderly and longingly after little Suzanne, who was being led away from the pleasant \textit{tête-à-tête} by her stern mother. Marguerite watched him across the room, as he finally turned away with a sigh, and seemed to stand, aimless and lonely, now that Suzanne's dainty little figure had disappeared in the crowd.

She watched him as he strolled towards the doorway, which led to a small boudoir beyond, then paused and leaned against the framework of it, looking still anxiously all round him.

Marguerite contrived for the moment to evade her present attentive cavalier, and she skirted the fashionable crowd, drawing nearer to the doorway, against which Sir Andrew was leaning. Why she wished to get closer to him, she could not have said: perhaps she was impelled by an all-powerful fatality, which so often seems to rule the destinies of men.

Suddenly she stopped: her very heart seemed to stand still, her eyes, large and excited, flashed for a moment towards that doorway, then as quickly were turned away again. Sir Andrew Ffoulkes was still in the same listless position by the door, but Marguerite had distinctly seen that Lord Hastings\allowbreak---\allowbreak a young buck, a friend of her husband's and one of the Prince's set\allowbreak---\allowbreak had, as he quickly brushed past him, slipped something into his hand.

For one moment longer\allowbreak---\allowbreak oh! it was the merest flash\allowbreak---\allowbreak Marguerite paused: the next she had, with admirably played unconcern, resumed her walk across the room\allowbreak---\allowbreak but this time more quickly towards that doorway whence Sir Andrew had now disappeared.

All this, from the moment that Marguerite had caught sight of Sir Andrew leaning against the doorway, until she followed him into the little boudoir beyond, had occurred in less than a minute. Fate is usually swift when she deals a blow.

Now Lady Blakeney had suddenly ceased to exist. It was Marguerite St~Just who was there only: Marguerite St~Just who had passed her childhood, her early youth, in the protecting arms of her brother Armand. She had forgotten everything else\allowbreak---\allowbreak her rank, her dignity, her secret enthusiasms\allowbreak---\allowbreak everything save that Armand stood in peril of his life, and that there, not twenty feet away from her, in the small boudoir which was quite deserted, in the very hands of Sir Andrew Ffoulkes, might be the talisman which would save her brother's life.

Barely another thirty seconds had elapsed between the moment when Lord Hastings slipped the mysterious »something« into Sir Andrew's hand, and the one when she, in her turn, reached the deserted boudoir. Sir Andrew was standing with his back to her and close to a table upon which stood a massive silver candelabra. A slip of paper was in his hand, and he was in the very act of perusing its contents.

Unperceived, her soft clinging robe making not the slightest sound upon the heavy carpet, not daring to breathe until she had accomplished her purpose, Marguerite slipped close behind him... At that moment he looked round and saw her; she uttered a groan, passed her hand across her forehead, and murmured faintly,\longdash


»The heat in the room was terrible... I felt so faint... Ah!...«

She tottered almost as if she would fall, and Sir Andrew, quickly recovering himself, and crumpling in his hand the tiny note he had been reading, was only, apparently, just in time to support her.

»You are ill, Lady Blakeney?« he asked with much concern. »Let me...«

»No, no, nothing\longdash« she interrupted quickly. »A chair\allowbreak---\allowbreak quick.«

She sank into a chair close to the table, and throwing back her head, closed her eyes.

»There!« she murmured, still faintly; »the giddiness is passing off... Do not heed me, Sir Andrew; I assure you I already feel better.«

At moments like these there is no doubt\allowbreak---\allowbreak and psychologists actually assert it\allowbreak---\allowbreak that there is in us a sense which has absolutely nothing to do with the other five: it is not that we see, it is not that we hear or touch, yet we seem to do all three at once. Marguerite sat there with her eyes apparently closed. Sir Andrew was immediately behind her, and on her right was the table with the five-armed candelabra upon it. Before her mental vision there was absolutely nothing but Armand's face. Armand, whose life was in the most imminent danger, and who seemed to be looking at her from a background upon which were dimly painted the seething crowd of Paris, the bare walls of the Tribunal of Public Safety, with Foucquier-Tinville, the Public Prosecutor, demanding Armand's life in the name of the people of France, and the lurid guillotine with its stained knife waiting for another victim... Armand!... For one moment there was dead silence in the little boudoir. Beyond, from the brilliant ball-room, the sweet notes of the gavotte, the frou-frou of rich dresses, the talk and laughter of a large and merry crowd, came as a strange, weird accompaniment to the drama which was being enacted here.

Sir Andrew had not uttered another word. Then it was that that extra sense became potent in Marguerite Blakeney. She could not see, for her eyes were closed; she could not hear, for the noise from the ball-room drowned the soft rustle of that momentous scrap of paper; nevertheless she knew\allowbreak---\allowbreak as if she had both seen and heard\allowbreak---\allowbreak that Sir Andrew was even now holding the paper to the flame of one of the candles.

At the exact moment that it began to catch fire, she opened her eyes, raised her hand and, with two dainty fingers, had taken the burning scrap of paper from the young man's hand. Then she blew out the flame, and held the paper to her nostril with perfect unconcern.

»How thoughtful of you, Sir Andrew,« she said gaily, »surely 'twas your grandmother who taught you that the smell of burnt paper was a sovereign remedy against giddiness.«

She sighed with satisfaction, holding the paper tightly between her jewelled fingers; that talisman which perhaps would save her brother Armand's life. Sir Andrew was staring at her, too dazed for the moment to realise what had actually happened; he had been taken so completely by surprise, that he seemed quite unable to grasp the fact that the slip of paper, which she held in her dainty hand, was one perhaps on which the life of his comrade might depend.

Marguerite burst into a long, merry peal of laughter.

»Why do you stare at me like that?« she said playfully. »I assure you I feel much better; your remedy has proved most effectual. This room is most delightfully cool,« she added, with the same perfect composure, »and the sound of the gavotte from the ball-room is fascinating and soothing.«

She was prattling on in the most unconcerned and pleasant way, whilst Sir Andrew, in an agony of mind, was racking his brains as to the quickest method he could employ to get that bit of paper out of that beautiful woman's hand. Instinctively, vague and tumultuous thoughts  rushed through his mind: he suddenly remembered her nationality, and worst of all, recollected that horrible tale anent the Marquis de St~Cyr, which in England no one had credited, for the sake of Sir Percy, as well as for her own.

»What? Still dreaming and staring?« she said, with a merry laugh, »you are most ungallant, Sir Andrew; and now I come to think of it, you seemed more startled than pleased when you saw me just now. I do believe, after all, that it was not concern for my health, nor yet a remedy taught you by your grandmother that caused you to burn this tiny scrap of paper... I vow it must have been your lady love's last cruel epistle you were trying to destroy. Now confess!« she added, playfully holding up the scrap of paper, »does this contain her final \textit{congé}, or a last appeal to kiss and make friends?«

»Whichever it is, Lady Blakeney,« said Sir Andrew, who was gradually recovering his self-possession, »this little note is undoubtedly mine, and...«

Not caring whether his action was one that would be styled ill-bred towards a lady, the young man had made a bold dash for the note; but Marguerite's thoughts flew quicker than his own; her actions, under pressure of this intense excitement, were swifter and more sure. She was tall and strong; she took a quick step backwards and knocked over the small Sheraton table which was already top-heavy, and which fell down with a crash, together with the massive candelabra upon it.

She gave a quick cry of alarm:

»The candles, Sir Andrew\allowbreak---\allowbreak quick!«

There was not much damage done; one or two of the candles had blown out as the candelabra fell; others had merely sent some grease upon the valuable carpet; one had ignited the paper shade over it. Sir Andrew quickly and dexterously put out the flames and replaced the candelabra upon the table; but this had taken him a few seconds to do, and those seconds had been all that Marguerite needed to cast a quick glance at the paper, and to note its contents\allowbreak---\allowbreak a dozen words in the same distorted handwriting she had seen before, and bearing the same device\allowbreak---\allowbreak a star-shaped flower drawn in red ink.

When Sir Andrew once more looked at her, he only saw on her face alarm at the untoward accident and relief at its happy issue; whilst the tiny and momentous note had apparently fluttered to the ground. Eagerly the young man picked it up, and his face looked much relieved, as his fingers closed tightly over it.

»For shame, Sir Andrew,« she said, shaking her head with a playful sigh, »making havoc in the heart of some impressionable duchess, whilst conquering the affections of my sweet little Suzanne. Well, well! I do believe it was Cupid himself who stood by you, and threatened the entire Foreign Office with destruction by fire, just on purpose to make me drop love's message, before it had been polluted by my indiscreet eyes. To think that, a moment longer, and I might have known the secrets of an erring duchess.«

»You will forgive me, Lady Blakeney,« said Sir Andrew, now as calm as she was herself, »if I resume the interesting occupation which you had interrupted?«

»By all means, Sir Andrew! How should I venture to thwart the love-god again? Perhaps he would mete out some terrible chastisement against my presumption. Burn your love-token, by all means!«

Sir Andrew had already twisted the paper into a long spill, and was once again holding it to the flame of the candle, which had remained alight. He did not notice the strange smile on the face of his fair \textit{vis-à-vis}, so intent was he on the work of destruction; perhaps, had he done so, the look of relief would have faded from his face. He watched the fateful note, as it curled under the flame. Soon the last fragment fell on the floor, and he placed his heel upon the ashes.

»And now, Sir Andrew,« said Marguerite Blakeney, with the pretty nonchalance peculiar to herself, and with the most winning of smiles, »will you venture to excite the jealousy of your fair lady by asking me to dance the minuet?«