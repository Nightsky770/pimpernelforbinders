%!TeX root=../scarlettop.tex

\chapter{Doubt}
\lettrine[lines=4]{M}{arguerite} Blakeney had watched the slight sable-clad figure of Chauvelin, as he worked his way through the ball-room. Then perforce she had had to wait, while her nerves tingled with excitement.

Listlessly she sat in the small, still deserted boudoir, looking out through the curtained doorway on the dancing couples beyond: looking at them, yet seeing nothing, hearing the music, yet conscious of naught save a feeling of expectancy, of anxious, weary waiting.

Her mind conjured up before her the vision of what was, perhaps at this very moment, passing downstairs. The half-deserted dining-room, the fateful hour---Chauvelin on the watch!---then, precise to the moment, the entrance of a man, he, the Scarlet Pimpernel, the mysterious leader, who to Marguerite had become almost unreal, so strange, so weird was this hidden identity.

She wished she were in the supper-room, too, at this moment, watching him as he entered; she knew that her woman's penetration would at once recognise in the  stranger's face---whoever he might be---that strong individuality which belongs to a leader of men---to a hero: to the mighty, high-soaring eagle, whose daring wings were becoming entangled in the ferret's trap.

Woman-like, she thought of him with unmixed sadness; the irony of that fate seemed so cruel which allowed the fearless lion to succumb to the gnawing of a rat! Ah! had Armand's life not been at stake!... \enquote{Faith! your ladyship must have thought me very remiss,} said a voice suddenly, close to her elbow. \enquote{I had a deal of difficulty in delivering your message, for I could not find Blakeney anywhere at first...}

Marguerite had forgotten all about her husband and her message to him; his very name, as spoken by Lord Fancourt, sounded strange and unfamiliar to her, so completely had she in the last five minutes lived her old life in the Rue de Richelieu again, with Armand always near her to love and protect her, to guard her from the many subtle intrigues which were forever raging in Paris in those days.

\enquote{I did find him at last,} continued Lord Fancourt, \enquote{and gave him your message. He said that he would give orders at once for the horses to be put to.}

\enquote{Ah!} she said, still very absently, \enquote{you found my husband, and gave him my message?}

\enquote{Yes; he was in the dining-room fast asleep. I could not manage to wake him up at first.}

\enquote{Thank you very much,} she said mechanically, trying to collect her thoughts.

\enquote{Will your ladyship honour me with the \textit{contredanse} until your coach is ready?} asked Lord Fancourt.

\enquote{No, I thank you, my lord, but---and you will forgive me---I really am too tired, and the heat in the ball-room has become oppressive.}

\enquote{The conservatory is deliciously cool; let me take you there, and then get you something. You seem ailing, Lady Blakeney.}

\enquote{I am only very tired,} she repeated wearily, as she allowed Lord Fancourt to lead her, where subdued lights and green plants lent coolness to the air. He got her a chair, into which she sank. This long interval of waiting was intolerable. Why did not Chauvelin come and tell her the result of his watch?

Lord Fancourt was very attentive. She scarcely heard what he said, and suddenly startled him by asking abruptly,\longdash


\enquote{Lord Fancourt, did you perceive who was in the dining-room just now besides Sir Percy Blakeney?}

\enquote{Only the agent of the French Government, M. Chauvelin, equally fast asleep in another corner,} he said. \enquote{Why does your ladyship ask?}

\enquote{I know not... I... Did you notice the time when you were there?}

\enquote{It must have been about five or ten minutes past one... I wonder what your ladyship is thinking about,} he added, for evidently the fair lady's thoughts were very far away, and she had not been listening to his intellectual conversation.

But indeed her thoughts were not very far away: only one storey below, in this same house, in the dining-room where sat Chauvelin still on the watch. Had he failed? For one instant that possibility rose before her as a hope---the hope that the Scarlet Pimpernel had been warned by Sir Andrew, and that Chauvelin's trap had failed to catch his bird; but that hope soon gave way to fear. Had he failed? But then---Armand!

Lord Fancourt had given up talking since he found that he had no listener. He wanted an opportunity for slipping away: for sitting opposite to a lady, however fair, who is evidently not heeding the most vigorous efforts made for her entertainment, is not exhilarating, even to a Cabinet Minister.

\enquote{Shall I find out if your ladyship's coach is ready,} he said at last, tentatively.

\enquote{Oh, thank you... thank you... if you would be so kind... I fear I am but sorry company... but I am really tired... and, perhaps, would be best alone.}

She had been longing to be rid of him, for she hoped that, like the fox he so resembled, Chauvelin would be prowling round, thinking to find her alone.

But Lord Fancourt went, and still Chauvelin did not come. Oh! what had happened? She felt Armand's fate trembling in the balance... she feared---now with a deadly fear---that Chauvelin \textit{had} failed, and that the mysterious Scarlet Pimpernel had proved elusive once more; then she knew that she need hope for no pity, no mercy, from him.

He had pronounced his \enquote{Either---or\longdash} and nothing less would content him: he was very spiteful, and would affect the belief that she had wilfully misled him, and having failed to trap the eagle once again, his revengeful mind would be content with the humble prey---Armand!

Yet she had done her best; had strained every nerve for Armand's sake. She could not bear to think that all had failed. She could not sit still; she wanted to go and hear the worst at once; she wondered even that Chauvelin had not come yet, to vent his wrath and satire upon her.

Lord Grenville himself came presently to tell her that her coach was ready, and that Sir Percy was already waiting for her---ribbons in hand. Marguerite said \enquote{Farewell} to her distinguished host; many of her friends stopped her, as she crossed the rooms, to talk to her, and exchange pleasant \textit{au revoirs}.

The Minister only took final leave of beautiful Lady Blakeney on the top of the stairs; below, on the landing, a veritable army of gallant gentlemen were waiting to bid \enquote{Good-bye} to the queen of beauty and fashion, whilst outside, under the massive portico, Sir Percy's magnificent bays were impatiently pawing the ground.

At the top of the stairs, just after she had taken final leave of her host, she suddenly saw Chauvelin; he was coming up the stairs slowly, and rubbing his thin hands very softly together.

There was a curious look on his mobile face, partly amused and wholly puzzled, and as his keen eyes met Marguerite's they became strangely sarcastic.

\enquote{M. Chauvelin,} she said, as he stopped on the top of the stairs, bowing elaborately before her, \enquote{my coach is outside; may I claim your arm?}

As gallant as ever, he offered her his arm and led her downstairs. The crowd was very great, some of the Minister's guests were departing, others were leaning against the banisters watching the throng as it filed up and down the wide staircase.

\enquote{Chauvelin,} she said at last desperately, \enquote{I must know what has happened.}

\enquote{What has happened, dear lady?} he said, with affected surprise. \enquote{Where? When?}

\enquote{You are torturing me, Chauvelin. I have helped you to-night... surely I have the right to know. What happened in the dining-room at one o'clock just now?}

She spoke in a whisper, trusting that in the general hubbub of the crowd her words would remain unheeded by all, save the man at her side.

\enquote{Quiet and peace reigned supreme, fair lady; at that hour I was asleep in the corner of one sofa and Sir Percy Blakeney in another.}

\enquote{Nobody came into the room at all?}

\enquote{Nobody.}

\enquote{Then we have failed, you and I?...}

\enquote{Yes! we have failed---perhaps...}

\enquote{But Armand?} she pleaded.

\enquote{Ah! Armand St~Just's chances hang on a thread... pray heaven, dear lady, that that thread may not snap.}

\enquote{Chauvelin, I worked for you, sincerely, earnestly... remember...}

\enquote{I remember my promise,} he said quietly; \enquote{the day that the Scarlet Pimpernel and I meet on French soil, St~Just will be in the arms of his charming sister.}

\enquote{Which means that a brave man's blood will be on my hands,} she said, with a shudder.

\enquote{His blood, or that of your brother. Surely at the present moment you must hope, as I do, that the enigmatical Scarlet Pimpernel will start for Calais to-day\longdash}

\enquote{I am only conscious of one hope, citoyen.}

\enquote{And that is?}

\enquote{That Satan, your master, will have need of you elsewhere, before the sun rises to-day.}

\enquote{You flatter me, citoyenne.}

She had detained him for a while, midway down the stairs, trying to get at the thoughts which lay beyond that thin, fox-like mask. But Chauvelin remained urbane, sarcastic, mysterious; not a line betrayed to the poor, anxious woman whether she need fear or whether she dared to hope.

Downstairs on the landing she was soon surrounded. Lady Blakeney never stepped from any house into her coach, without an escort of fluttering human moths around the dazzling light of her beauty. But before she finally turned away from Chauvelin, she held out a tiny hand to him, with that pretty gesture of childish appeal which was so essentially her own.

\enquote{Give me some hope, my little Chauvelin,} she pleaded.

With perfect gallantry he bowed over that tiny hand, which looked so dainty and white through the delicately transparent black lace mitten, and kissing the tips of the rosy fingers:\longdash


\enquote{Pray heaven that the thread may not snap,} he repeated, with his enigmatic smile.

And stepping aside, he allowed the moths to flutter more closely round the candle, and the brilliant throng of the \textit{jeunesse dorée}, eagerly attentive to Lady Blakeney's every movement, hid the keen, fox-like face from her view.