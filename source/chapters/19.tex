%!TeX root=../scarlettop.tex

\chapter{The Scarlet Pimpernel}
\lettrine[lines=4]{A}{t} what particular moment the strange doubt first crept into Marguerite's mind, she could not herself afterwards have said. With the ring tightly clutched in her hand, she had run out of the room, down the stairs, and out into the garden, where, in complete seclusion, alone with the flowers, and the river and the birds, she could look again at the ring, and study that device more closely.

Stupidly, senselessly, now, sitting beneath the shade of an overhanging sycamore, she was looking at the plain gold shield, with the star-shaped little flower engraved upon it.

Bah! It was ridiculous! she was dreaming! her nerves were overwrought, and she saw signs and mysteries in the most trivial coincidences. Had not everybody about town recently made a point of affecting the device of that mysterious and heroic Scarlet Pimpernel?

Did she not herself wear it embroidered on her gowns? set in gems and enamel in her hair? What was there strange in the fact that Sir Percy should have chosen to use the device as a seal-ring? He might easily have done that... yes... quite easily... and... besides... what connection could there be between her exquisite dandy of a husband, with his fine clothes and refined, lazy ways, and the daring plotter who rescued French victims from beneath the very eyes of the leaders of a bloodthirsty revolution?

Her thoughts were in a whirl\allowbreak---\allowbreak her mind a blank... She did not see anything that was going on around her, and was quite startled when a fresh young voice called to her across the garden.

»\textit{Chérie}!\allowbreak---\allowbreak \textit{chérie}! where are you?« and little Suzanne, fresh as a rosebud, with eyes dancing with glee, and brown curls fluttering in the soft morning breeze, came running across the lawn.

»They told me you were in the garden,« she went on prattling merrily, and throwing herself with pretty, girlish impulse into Marguerite's arms, \enquote{so I ran out to give you a surprise. You did not expect me quite so soon, did you, my darling little Margot \textit{chérie}?}

Marguerite, who had hastily concealed the ring in the folds of her kerchief, tried to respond gaily and unconcernedly to the young girl's impulsiveness.

»Indeed, sweet one,« she said with a smile, »it is delightful to have you all to myself, and for a nice whole long day... You won't be bored?«

»Oh! bored! Margot, how \textit{can} you say such a wicked thing. Why! when we were in the dear old convent together, we were always happy when we were allowed to be alone together.«

»And to talk secrets.«

The two young girls had linked their arms in one another's and began wandering round the garden.

»Oh! how lovely your home is, Margot, darling,« said little Suzanne, enthusiastically, »and how happy you must be!«

»Aye, indeed! I ought to be happy\allowbreak---\allowbreak oughtn't I, sweet one?« said Marguerite, with a wistful little sigh.

»How sadly you say it, \textit{chérie}... Ah, well, I suppose now that you are a married woman you won't care to talk secrets with me any longer. Oh! what lots and lots of secrets we used to have at school! Do you remember?\allowbreak---\allowbreak some we did not even confide to Sister Theresa of the Holy Angels\allowbreak---\allowbreak though she was such a dear.«

»And now you have one all-important secret, eh, little one?« said Marguerite, merrily, »which you are forthwith going to confide to me. Nay, you need not blush, \textit{chérie},« she added, as she saw Suzanne's pretty little face crimson with blushes. »Faith, there's naught to be ashamed of! He is a noble and true man, and one to be proud of as a lover, and... as a husband.«

»Indeed, \textit{chérie}, I am not ashamed,« rejoined Suzanne, softly; »and it makes me very, very proud to hear you speak so well of him. I think \textit{maman} will consent,« she added thoughtfully, »and I shall be\allowbreak---\allowbreak oh! so happy\allowbreak---\allowbreak but, of course, nothing is to be thought of until papa is safe...«

Marguerite started. Suzanne's father! the Comte de Tournay! \allowbreak---\allowbreak one of those whose life would be jeopardised if Chauvelin succeeded in establishing the identity of the Scarlet Pimpernel.

She had understood all along from the Comtesse, and also from one or two of the members of the league, that their mysterious leader had pledged his honour to bring the fugitive Comte de Tournay safely out of France. Whilst little Suzanne\allowbreak---\allowbreak unconscious of all\allowbreak---\allowbreak save her own  all-important little secret, went prattling on, Marguerite's thoughts went back to the events of the past night.

Armand's peril, Chauvelin's threat, his cruel »Either\allowbreak---\allowbreak or\longdash« which she had accepted.

And then her own work in the matter, which should have culminated at one o'clock in Lord Grenville's dining-room, when the relentless agent of the French Government would finally learn who was this mysterious Scarlet Pimpernel, who so openly defied an army of spies and placed himself so boldly, and for mere sport, on the side of the enemies of France.

Since then she had heard nothing from Chauvelin. She had concluded that he had failed, and yet, she had not felt anxious about Armand, because her husband had promised her that Armand would be safe.

But now, suddenly, as Suzanne prattled merrily along, an awful horror came upon her for what she had done. Chauvelin had told her nothing, it was true; but she remembered how sarcastic and evil he looked when she took final leave of him after the ball. Had he discovered something then? Had he already laid his plans for catching the daring plotter, red-handed, in France, and sending him to the guillotine without compunction or delay?

Marguerite turned sick with horror, and her hand convulsively clutched the ring in her dress.

»You are not listening, \textit{chérie},« said Suzanne, reproachfully, as she paused in her long, highly interesting narrative.

»Yes, yes, darling\allowbreak---\allowbreak indeed I am,« said Marguerite with an effort, forcing herself to smile. »I love to hear you talking... and your happiness makes me so very glad... Have no fear, we will manage to propitiate \textit{maman}. Sir Andrew Ffoulkes is a noble English gentleman; he has money and position, the Comtesse will not refuse her consent... But... now, little one... tell me... what is the latest news about your father?«

»Oh!« said Suzanne, with mad glee, »the best we could possibly hear. My Lord Hastings came to see \textit{maman} early this morning. He said that all is now well with dear papa, and we may safely expect him here in England in less than four days.«

»Yes,« said Marguerite, whose glowing eyes were fastened on Suzanne's lips, as she continued merrily:

»Oh, we have no fear now! You don't know, \textit{chérie}, that that great and noble Scarlet Pimpernel himself has gone to save papa. He has gone, \textit{chérie}... actually gone...« added Suzanne excitedly. »He was in London this morning; he will be in Calais, perhaps, to-morrow... where he will meet papa... and then... and then...«

The blow had fallen. She had expected it all along, though she had tried for the last half-hour to delude herself and to cheat her fears. He had gone to Calais, had been in London this morning... he... the Scarlet Pimpernel... Percy Blakeney... her husband... whom she had betrayed last night to Chauvelin...

Percy... Percy... her husband... the Scarlet Pimpernel... Oh! how could she have been so blind? She understood it now\allowbreak---\allowbreak all at once... that part he played\allowbreak---\allowbreak the mask he wore... in order to throw dust in everybody's eyes.

And all for sheer sport and devilry of course!\allowbreak---\allowbreak saving men, women and children from death, as other men destroy and kill animals for the excitement, the love of the thing. The idle, rich man wanted some aim in life\allowbreak---\allowbreak he, and the few young bucks he enrolled under his banner, had amused themselves for months in risking their lives for the sake of an innocent few.

Perhaps he had meant to tell her when they were first married; and then the story of the Marquis de St~Cyr had come to his ears, and he had suddenly turned from her, thinking, no doubt, that she might some day betray him and his comrades, who had sworn to follow him; and so he had tricked her, as he tricked all others, whilst hundreds now owed their lives to him, and many families owed him both life and happiness.

The mask of the inane fop had been a good one, and the part consummately well played. No wonder that Chauvelin's spies had failed to detect, in the apparently brainless nincompoop, the man whose reckless daring and resourceful ingenuity had baffled the keenest French spies, both in France and in England. Even last night when Chauvelin went to Lord Grenville's dining-room to seek that daring Scarlet Pimpernel, he only saw that inane Sir Percy Blakeney fast asleep in a corner of the sofa.

Had his astute mind guessed the secret, then? Here lay the whole awful, horrible, amazing puzzle. In betraying a nameless stranger to his fate in order to save her brother, had Marguerite Blakeney sent her husband to his death?

No! no! no! a thousand times no! Surely Fate could not deal a blow like that: Nature itself would rise in revolt: her hand, when it held that tiny scrap of paper last night, would surely have been struck numb ere it committed a deed so appalling and so terrible.

»But what is it, \textit{chérie}?« said little Suzanne, now genuinely alarmed, for Marguerite's colour had become dull and ashen. »Are you ill, Marguerite? What is it?«

»Nothing, nothing, child,« she murmured, as in a dream. »Wait a moment... let me think... think!... You said... the Scarlet Pimpernel had gone to-day... ?«

»Marguerite, \textit{chérie}, what is it? You frighten me...«

»It is nothing, child, I tell you... nothing... I must be alone a minute\allowbreak---\allowbreak and\allowbreak---\allowbreak dear one... I may have to curtail our time together to-day... I may have to go away\allowbreak---\allowbreak you'll understand?«

»I understand that something has happened, \textit{chérie}, and that you want to be alone. I won't be a hindrance to you. Don't think of me. My maid, Lucile, has not yet gone... we will go back together... don't think of me.«

She threw her arms impulsively round Marguerite. Child as she was, she felt the poignancy of her friend's grief, and with the infinite tact of her girlish tenderness, she did not try to pry into it, but was ready to efface herself.

She kissed Marguerite again and again, then walked sadly back across the lawn. Marguerite did not move, she remained there, thinking... wondering what was to be done.

Just as little Suzanne was about to mount the terrace steps, a groom came running round the house towards his mistress. He carried a sealed letter in his hand. Suzanne instinctively turned back; her heart told her that here perhaps was further ill news for her friend, and she felt that her poor Margot was not in a fit state to bear any more.

The groom stood respectfully beside his mistress, then he handed her the sealed letter.

»What is that?« asked Marguerite.

»Just come by runner, my lady.«

Marguerite took the letter mechanically, and turned it over in her trembling fingers.

»Who sent it?« she said.

»The runner said, my lady,« replied the groom, »that his orders were to deliver this, and that your ladyship would understand from whom it came.«

Marguerite tore open the envelope. Already her instinct had told her what it contained, and her eyes only glanced at it mechanically.

It was a letter written by Armand St~Just to Sir Andrew Ffoulkes\allowbreak---\allowbreak the letter which Chauvelin's spies had stolen at »The Fisherman's Rest,« and which Chauvelin had held as a rod over her to enforce her obedience.

Now he had kept his word\allowbreak---\allowbreak he had sent her back St~Just's compromising letter... for he was on the track of the Scarlet Pimpernel.

Marguerite's senses reeled, her very soul seemed to be leaving her body; she tottered, and would have fallen but for Suzanne's arm round her waist. With superhuman effort she regained control over herself\allowbreak---\allowbreak there was yet much to be done.

»Bring that runner here to me,« she said to the servant, with much calm. »He has not gone?«

»No, my lady.«

The groom went, and Marguerite turned to Suzanne.

»And you, child, run within. Tell Lucile to get ready. I fear I must send you home, child. And\allowbreak---\allowbreak stay, tell one of the maids to prepare a travelling dress and cloak for me.«

Suzanne made no reply. She kissed Marguerite tenderly, and obeyed without a word; the child was overawed by the terrible, nameless misery in her friend's face.

A minute later the groom returned, followed by the runner who had brought the letter.

»Who gave you this packet?« asked Marguerite.

»A gentleman, my lady,« replied the man, »at »The Rose and Thistle« inn opposite Charing Cross. He said you would understand.«

»At »The Rose and Thistle«? What was he doing?«

»He was waiting for the coach, your ladyship, which he had ordered.«

»The coach?«

»Yes, my lady. A special coach he had ordered. I understood from his man that he was posting straight to Dover.«

»That's enough. You may go.« Then she turned to the groom: »My coach and the four swiftest horses in the stables, to be ready at once.«

The groom and runner both went quickly off to obey. Marguerite remained standing for a moment on the lawn quite alone. Her graceful figure was as rigid as a statue, her eyes were fixed, her hands were tightly clasped across her breast; her lips moved as they murmured with pathetic heart-breaking persistence,\longdash


»What's to be done? What's to be done? Where to find him?\allowbreak---\allowbreak Oh, God! grant me light.«

But this was not the moment for remorse and despair. She had done\allowbreak---\allowbreak unwittingly\allowbreak---\allowbreak an awful and terrible thing\allowbreak---\allowbreak the very worst crime, in her eyes, that woman ever  committed\allowbreak---\allowbreak she saw it in all its horror. Her very blindness in not having guessed her husband's secret seemed now to her another deadly sin. She ought to have known! she ought to have known!

How could she imagine that a man who could love with so much intensity as Percy Blakeney had loved her from the first\allowbreak---\allowbreak how could such a man be the brainless idiot he chose to appear? She, at least, ought to have known that he was wearing a mask, and having found that out, she should have torn it from his face, whenever they were alone together.

Her love for him had been paltry and weak, easily crushed by her own pride; and she, too, had worn a mask in assuming a contempt for him, whilst, as a matter of fact, she completely misunderstood him.

But there was no time now to go over the past. By her own blindness she had sinned; now she must repay, not by empty remorse, but by prompt and useful action.

Percy had started for Calais, utterly unconscious of the fact that his most relentless enemy was on his heels. He had set sail early that morning from London Bridge. Provided he had a favourable wind, he would no doubt be in France within twenty-four hours; no doubt he had reckoned on the wind and chosen this route.

Chauvelin, on the other hand, would post to Dover, charter a vessel there, and undoubtedly reach Calais much about the same time. Once in Calais, Percy would meet all those who were eagerly waiting for the noble and brave Scarlet Pimpernel, who had come to rescue them from horrible and unmerited death. With Chauvelin's eyes now fixed upon his every movement, Percy would thus not only be endangering his own life, but that of Suzanne's father, the old Comte de Tournay, and of those other fugitives who were waiting for him and trusting in him. There was also Armand, who had gone to meet de Tournay, secure in the knowledge that the Scarlet Pimpernel was watching over his safety.

All these lives, and that of her husband, lay in Marguerite's hands; these she must save, if human pluck and ingenuity were equal to the task.

Unfortunately, she could not do all this quite alone. Once in Calais she would not know where to find her husband, whilst Chauvelin, in stealing the papers at Dover, had obtained the whole itinerary. Above everything, she wished to warn Percy.

She knew enough about him by now to understand that he would never abandon those who trusted in him, that he would not turn back from danger, and leave the Comte de Tournay to fall into the bloodthirsty hands that knew of no mercy. But if he were warned, he might form new plans, be more wary, more prudent. Unconsciously, he might fall into a cunning trap, but\allowbreak---\allowbreak once warned\allowbreak---\allowbreak he might yet succeed.

And if he failed\allowbreak---\allowbreak if indeed Fate, and Chauvelin, with all the resources at his command, proved too strong for the daring plotter after all\allowbreak---\allowbreak then at least she would be there by his side, to comfort, love and cherish, to cheat death perhaps at the last by making it seem sweet, if they died both together, locked in each other's arms, with the supreme happiness of knowing that passion had responded to passion, and that all misunderstandings were at an end.

Her whole body stiffened as with a great and firm resolution. This she meant to do, if God gave her wits and strength. Her eyes lost their fixed look; they glowed with inward fire at the thought of meeting him again so soon, in the very midst of most deadly perils; they sparkled with the joy of sharing these dangers with him\allowbreak---\allowbreak of helping him perhaps\allowbreak---\allowbreak of being with him at the last\allowbreak---\allowbreak if she failed.

The childlike sweet face had become hard and set, the curved mouth was closed tightly over her clenched teeth. She meant to do or die, with him and for his sake. A frown, which spoke of an iron will and unbending resolution, appeared between the two straight brows; already her plans were formed. She would go and find Sir Andrew Ffoulkes first; he was Percy's best friend, and Marguerite remembered with a thrill, with what blind enthusiasm the young man always spoke of his mysterious leader.

He would help her where she needed help; her coach was ready. A change of raiment, and a farewell to little Suzanne, and she could be on her way.

Without haste, but without hesitation, she walked quietly into the house.