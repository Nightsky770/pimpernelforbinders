%!TeX root=../scarlettop.tex

\chapter{The Père Blanchard's Hut}
\lettrine[lines=4]{A}{s} in a dream, Marguerite followed on; the web was drawing more and more tightly every moment round the beloved life, which had become dearer than all. To see her husband once again, to tell him how she had suffered, how much she had wronged, and how little understood him, had become now her only aim. She had abandoned all hope of saving him: she saw him gradually hemmed in on all sides, and, in despair, she gazed round her into the darkness, and wondered whence he would presently come, to fall into the death-trap which his relentless enemy had prepared for him.

The distant roar of the waves now made her shudder; the occasional dismal cry of an owl, or a sea-gull, filled her with unspeakable horror. She thought of the ravenous beasts---in human shape---who lay in wait for their prey, and destroyed them, as mercilessly as any hungry wolf, for the satisfaction of their own appetite of hate. Marguerite was not afraid of the darkness, she only feared that man, on ahead, who was sitting at the bottom of a rough wooden cart, nursing thoughts of vengeance, which would have made the very demons in hell chuckle with delight.

Her feet were sore. Her knees shook under her, from sheer bodily fatigue. For days now she had lived in a wild turmoil of excitement; she had not had a quiet rest for three nights; now, she had walked on a slippery road for nearly two hours, and yet her determination never swerved for a moment. She would see her husband, tell him all, and, if he was ready to forgive the crime, which she had committed in her blind ignorance, she would yet have the happiness of dying by his side.

She must have walked on almost in a trance, instinct alone keeping her up, and guiding her in the wake of the enemy, when suddenly her ears, attuned to the slightest sound, by that same blind instinct, told her that the cart had stopped, and that the soldiers had halted. They had come to their destination. No doubt on the right, somewhere close ahead, was the footpath that led to the edge of the cliff and to the hut.

Heedless of any risks, she crept quite close up to where Chauvelin stood, surrounded by his little troop: he had descended from the cart, and was giving some orders to the men. These she wanted to hear: what little chance she yet had, of being useful to Percy, consisted in hearing absolutely every word of his enemy's plans.

The spot where all the party had halted must have lain some eight hundred mètres from the coast; the sound of the sea came only very faintly, as from a distance. Chauvelin and Desgas, followed by the soldiers, had turned off sharply to the right of the road, apparently on to the footpath, which led to the cliffs. The Jew had remained on the road, with his cart and nag.

Marguerite, with infinite caution, and literally crawling on her hands and knees, had also turned off to the right: to accomplish this she had to creep through the rough, low shrubs, trying to make as little noise as possible as she went along, tearing her face and hands against the dry twigs, intent only upon hearing without being seen or heard. Fortunately---as is usual in this part of France---the footpath was bordered by a low, rough hedge, beyond which was a dry ditch, filled with coarse grass. In this Marguerite managed to find shelter; she was quite hidden from view, yet could contrive to get within three yards of where Chauvelin stood, giving orders to his men.

\enquote{Now,} he was saying in a low and peremptory whisper, \enquote{where is the Père Blanchard's hut?}

\enquote{About eight hundred mètres from here, along the footpath,} said the soldier who had lately been directing the party, \enquote{and half-way down the cliff.}

\enquote{Very good. You shall lead us. Before we begin to descend the cliff, you shall creep down to the hut, as noiselessly as possible, and ascertain if the traitor royalists are there? Do you understand?}

\enquote{I understand, citoyen.}

\enquote{Now listen very attentively, all of you,} continued Chauvelin, impressively, and addressing the soldiers collectively, \enquote{for after this we may not be able to exchange another word, so remember every syllable I utter, as if your very lives depended on your memory. Perhaps they do,} he added drily.

\enquote{We listen, citoyen,} said Desgas, \enquote{and a soldier of the Republic never forgets an order.}

\enquote{You, who have crept up to the hut, will try to peep inside. If an Englishman is there with those traitors, a man who is tall above the average, or who stoops as if he would disguise his height, then give a sharp, quick whistle as a signal to your comrades. All of you,} he added, once more speaking to the soldiers collectively, \enquote{then quickly surround and rush into the hut, and each seize one of the men there, before they have time to draw their firearms; if any of them struggle, shoot at their legs or arms, but on no account kill the tall man. Do you understand?}

\enquote{We understand, citoyen.}

\enquote{The man who is tall above the average is probably also strong above the average; it will take four or five of you at least to overpower him.}

There was a little pause, then Chauvelin continued,\longdash


\enquote{If the royalist traitors are still alone, which is more than likely to be the case, then warn your comrades who are lying in wait there, and all of you creep and take cover behind the rocks and boulders round the hut, and wait there, in dead silence, until the tall Englishman arrives; then only rush the hut, when he is safely within its doors. But remember that you must be as silent as the wolf is at night, when he prowls around the pens. I do not wish those royalists to be on the alert---the firing of a pistol, a shriek or call on their part would be sufficient, perhaps, to warn the tall personage to keep clear of the cliffs, and of the hut, and,} he added emphatically, \enquote{it is the tall Englishman whom it is your duty to capture to-night.}

\enquote{You shall be implicitly obeyed, citoyen.}

\enquote{Then get along as noiselessly as possible, and I will follow you.}

\enquote{What about the Jew, citoyen?} asked Desgas, as silently like noiseless shadows, one by one the soldiers began to creep along the rough and narrow footpath.

\enquote{Ah, yes; I had forgotten the Jew,} said Chauvelin, and, turning towards the Jew, he called him peremptorily.

\enquote{Here, you... Aaron, Moses, Abraham, or whatever your confounded name may be,} he said to the old man, who had quietly stood beside his lean nag, as far away from the soldiers as possible.

\enquote{Benjamin Rosenbaum, so it please your Honour,} he replied humbly.

\enquote{It does not please me to hear your voice, but it does please me to give you certain orders, which you will find it wise to obey.}

\enquote{So it please your Honour...}

\enquote{Hold your confounded tongue. You shall stay here, do you hear? with your horse and cart until our return. You are on no account to utter the faintest sound, or even to breathe louder than you can help; nor are you, on any consideration whatever, to leave your post, until I give you orders to do so. Do you understand?}

\enquote{But your Honour\longdash} protested the Jew pitiably.

\enquote{There is no question of \enquote{but} or of any argument,} said Chauvelin, in a tone that made the timid old man tremble from head to foot. \enquote{If, when I return, I do not find you here, I most solemnly assure you that, wherever you may try to hide yourself, I can find you, and that punishment swift, sure and terrible, will sooner or later overtake you. Do you hear me?}

\enquote{But your Excellency...}

\enquote{I said, do you hear me?}

The soldiers had all crept away; the three men stood alone together in the dark and lonely road, with Marguerite there, behind the hedge, listening to Chauvelin's orders, as she would to her own death sentence.

\enquote{I heard your Honour,} protested the Jew again, while he tried to draw nearer to Chauvelin, \enquote{and I swear by Abraham, Isaac and Jacob that I would obey your Honour most absolutely, and that I would not move from this place until your Honour once more deigned to shed the light of your countenance upon your humble servant; but remember, your Honour, I am a poor old man; my nerves are not as strong as those of a young soldier. If midnight marauders should come prowling round this lonely road, I might scream or run in my fright! And is my life to be forfeit, is some terrible punishment to come on my poor old head for that which I cannot help?}

The Jew seemed in real distress; he was shaking from head to foot. Clearly he was not the man to be left by himself on this lonely road. The man spoke truly; he might unwittingly, in sheer terror, utter the shriek that might prove a warning to the wily Scarlet Pimpernel.

Chauvelin reflected for a moment.

\enquote{Will your horse and cart be safe alone, here, do you think?} he asked roughly.

\enquote{I fancy, citoyen,} here interposed Desgas, \enquote{that they will be safer without that dirty, cowardly Jew than with him. There seems no doubt that, if he gets scared, he will either make a bolt of it, or shriek his head off.}

\enquote{But what am I to do with the brute?}

\enquote{Will you send him back to Calais, citoyen?}

\enquote{No, for we shall want him to drive back the wounded presently,} said Chauvelin, with grim significance.

There was a pause again---Desgas, waiting for the decision of his chief, and the old Jew whining beside his nag.

\enquote{Well, you lazy, lumbering old coward,} said Chauvelin at last, \enquote{you had better shuffle along behind us. Here, Citoyen Desgas, tie this handkerchief tightly round the fellow's mouth.}

Chauvelin handed a scarf to Desgas, who solemnly began winding it round the Jew's mouth. Meekly Benjamin Rosenbaum allowed himself to be gagged; he, evidently, preferred this uncomfortable state to that of being left alone, on the dark St~Martin Road. Then the three men fell in line.

\enquote{Quick!} said Chauvelin, impatiently, \enquote{we have already wasted much valuable time.}

And the firm footsteps of Chauvelin and Desgas, the shuffling gait of the old Jew, soon died away along the footpath.

Marguerite had not lost a single one of Chauvelin's words of command. Her every nerve was strained to completely grasp the situation first, then to make a final appeal to those wits which had so often been called the sharpest in Europe, and which alone might be of service now.

Certainly the situation was desperate enough; a tiny band of unsuspecting men, quietly awaiting the arrival of their rescuer, who was equally unconscious of the trap laid for them all. It seemed so horrible, this net, as it were drawn in a circle, at dead of night, on a lonely beach, round a few defenceless men, defenceless because they were tricked and unsuspecting; of these one was the husband she idolised, another the brother she loved. She vaguely wondered who the others were, who were also calmly waiting for the Scarlet Pimpernel, while death lurked behind every boulder of the cliffs.

For the moment she could do nothing but follow the soldiers and Chauvelin. She feared to lose her way, or she would have rushed forward and found that wooden hut, and perhaps been in time to warn the fugitives and their brave deliverer yet.

For a second, the thought flashed through her mind of uttering the piercing shrieks, which Chauvelin seemed to dread, as a possible warning to the Scarlet Pimpernel and his friends---in the wild hope that they would hear, and have yet time to escape before it was too late. But she did not know how far from the edge of the cliff she was; she did not know if her shrieks would reach the ears of the doomed men. Her effort might be premature, and she would never be allowed to make another. Her mouth would be securely gagged, like that of the Jew, and she, a helpless prisoner in the hands of Chauvelin's men.

Like a ghost she flitted noiselessly behind that hedge: she had taken her shoes off, and her stockings were by now torn off her feet. She felt neither soreness nor weariness; indomitable will to reach her husband in spite of adverse Fate, and of a cunning enemy, killed all sense of bodily pain within her, and rendered her instincts doubly acute.

She heard nothing save the soft and measured footsteps of Percy's enemies on in front; she saw nothing but---in her mind's eye---that wooden hut, and he, her husband, walking blindly to his doom.

Suddenly, those same keen instincts within her made her pause in her mad haste, and cower still further within the shadow of the hedge. The moon, which had proved a friend to her by remaining hidden behind a bank of clouds, now emerged in all the glory of an early autumn night, and in a moment flooded the weird and lonely landscape with a rush of brilliant light.

There, not two hundred mètres ahead, was the edge of the cliff, and below, stretching far away to free and happy England, the sea rolled on smoothly and peaceably. Marguerite's gaze rested for an instant on the brilliant, silvery waters; and as she gazed, her heart, which had been numb with pain for all these hours, seemed to soften and distend, and her eyes filled with hot tears: not three miles away, with white sails set, a graceful schooner lay in wait.

Marguerite had guessed rather than recognised her. It was the \textit{Day Dream}, Percy's favourite yacht, with old Briggs, that prince of skippers, aboard, and all her crew of British sailors: her white sails, glistening in the moonlight, seemed to convey a message to Marguerite of joy and hope, which yet she feared could never be. She waited there, out at sea, waited for her master, like a beautiful white bird all ready to take flight, and he would never reach her, never see her smooth deck again, never gaze any more on the white cliffs of England, the land of liberty and of hope.

The sight of the schooner seemed to infuse into the poor, wearied woman the superhuman strength of despair. There was the edge of the cliff, and some way below was the hut, where presently, her husband would meet his death. But the moon was out: she could see her way now: she would see the hut from a distance, run to it, rouse them all, warn them at any rate to be prepared and to sell their lives dearly, rather than be caught like so many rats in a hole.

She stumbled on behind the hedge in the low, thick grass of the ditch. She must have run on very fast, and had outdistanced Chauvelin and Desgas, for presently she reached the edge of the cliff, and heard their footsteps distinctly behind her. But only a very few yards away, and now the moonlight was full upon her, her figure must have been distinctly silhouetted against the silvery background of the sea.

Only for a moment, though; the next she had cowered, like some animal doubled up within itself. She peeped down the great rugged cliffs---the descent would be easy enough, as they were not precipitous, and the great boulders afforded plenty of foothold. Suddenly, as she gazed, she saw at some little distance on her left, and about midway down the cliffs, a rough wooden construction, through the walls of which a tiny red light glimmered like a beacon. Her very heart seemed to stand still, the eagerness of joy was so great that it felt like an awful pain.

She could not gauge how distant the hut was, but without hesitation she began the steep descent, creeping from boulder to boulder, caring nothing for the enemy behind, or for the soldiers, who evidently had all taken cover since the tall Englishman had not yet appeared.

On she pressed, forgetting the deadly foe on her track, running, stumbling, foot-sore, half-dazed, but still on... When, suddenly, a crevice, or stone, or slippery bit of rock, threw her violently to the ground. She struggled again to her feet, and started running forward once more to give them that timely warning, to beg them to flee before he came, and to tell him to keep away---away from this death-trap---away from this awful doom. But now she realised that other steps, quicker than her own, were already close at her heels. The next instant a hand dragged at her skirt, and she was down on her knees again, whilst something was wound round her mouth to prevent her uttering a scream.

Bewildered, half frantic with the bitterness of disappointment, she looked round her helplessly, and, bending down quite close to her, she saw through the mist, which seemed to gather round her, a pair of keen, malicious eyes, which appeared to her excited brain to have a weird, supernatural green light in them.

She lay in the shadow of a great boulder; Chauvelin could not see her features, but he passed his thin, white fingers over her face.

\enquote{A woman!} he whispered, \enquote{by all the Saints in the calendar.}

\enquote{We cannot let her loose, that's certain,} he muttered to himself. \enquote{I wonder now...}

Suddenly he paused, and after a few seconds of deadly silence, he gave forth a long, low, curious chuckle, while once again Marguerite felt, with a horrible shudder, his thin fingers wandering over her face.

\enquote{Dear me! dear me!} he whispered, with affected gallantry, \enquote{this is indeed a charming surprise,} and Marguerite felt her resistless hand raised to Chauvelin's thin, mocking lips.

The situation was indeed grotesque, had it not been at the same time so fearfully tragic: the poor, weary woman, broken in spirit, and half frantic with the bitterness of her disappointment, receiving on her knees the \textit{banal} gallantries of her deadly enemy.

Her senses were leaving her; half choked with the tight grip round her mouth, she had no strength to move or to utter the faintest sound. The excitement which all along had kept up her delicate body seemed at once to have subsided, and the feeling of blank despair to have completely paralysed her brain and nerves.

Chauvelin must have given some directions, which she was too dazed to hear, for she felt herself lifted from off her feet: the bandage round her mouth was made more secure, and a pair of strong arms carried her towards that tiny, red light, on ahead, which she had looked upon as a beacon and the last faint glimmer of hope.

