%!TeX root=../scarlettop.tex

\chapter{Marguerite}
\lettrine[lines=4]{I}{n} a moment the pleasant oak-raftered coffee-room of the inn became the scene of hopeless confusion and discomfort. At the first announcement made by the stable boy, Lord Antony, with a fashionable oath, had jumped up from his seat and was now giving many and confused directions to poor bewildered Jellyband, who seemed at his wits’ end what to do.

\enquote{For goodness’ sake, man,} admonished his lordship, \enquote{try to keep Lady Blakeney talking outside for a moment, while the ladies withdraw. Zounds!} he added, with another more emphatic oath, \enquote{this is most unfortunate.}

\enquote{Quick, Sally! the candles!} shouted Jellyband, as hopping about from one leg to another, he ran hither and thither, adding to the general discomfort of everybody.

The Comtesse, too, had risen to her feet: rigid and erect, trying to hide her excitement beneath more becoming \textit{sang-froid}, she repeated mechanically,---

\enquote{I will not see her!---I will not see her!}

Outside, the excitement attendant upon the arrival of very important guests grew apace.

\enquote{Good-day, Sir Percy!---Good-day to your ladyship! Your servant, Sir Percy!}---was heard in one long, continued chorus, with alternate more feeble tones of---\enquote{Remember the poor blind man! of your charity, lady and gentleman!}

Then suddenly a singularly sweet voice was heard through all the din.

\enquote{Let the poor man be---and give him some supper at my expense.}

The voice was low and musical, with a slight sing-song in it, and a faint \textit{soupçon} of foreign intonation in the pronunciation of the consonants.

Everyone in the coffee-room heard it and paused, instinctively listening to it for a moment. Sally was holding the candles by the opposite door, which led to the bedrooms upstairs, and the Comtesse was in the act of beating a hasty retreat before that enemy who owned such a sweet musical voice; Suzanne reluctantly was preparing to follow her mother, whilst casting regretful glances towards the door, where she hoped still to see her dearly-beloved, erstwhile school-fellow.

Then Jellyband threw open the door, still stupidly and blindly hoping to avert the catastrophe which he felt was in the air, and the same low, musical voice said, with a merry laugh and mock consternation,---

\enquote{B-r-r-r-r! I am as wet as a herring! \textit{Dieu}! has anyone ever seen such a contemptible climate?}

\enquote{Suzanne, come with me at once---I wish it,} said the Comtesse, peremptorily.

\enquote{Oh! Mama!} pleaded Suzanne.

\enquote{My lady... er... h'm!... my lady!...} came in feeble accents from Jellyband, who stood clumsily trying to bar the way.

\enquote{\textit{Pardieu}, my good man,} said Lady Blakeney, with some impatience, \enquote{what are you standing in my way for, dancing about like a turkey with a sore foot? Let me get to the fire, I am perished with the cold.}

And the next moment Lady Blakeney, gently pushing mine host on one side, had swept into the coffee-room.

There are many portraits and miniatures extant of Marguerite St~Just---Lady Blakeney as she was then---but it is doubtful if any of these really do her singular beauty justice. Tall, above the average, with magnificent presence and regal figure, it is small wonder that even the Comtesse paused for a moment in involuntary admiration before turning her back on so fascinating an apparition.

Marguerite Blakeney was then scarcely five-and-twenty, and her beauty was at its most dazzling stage. The large hat, with its undulating and waving plumes, threw a soft shadow across the classic brow with the aureole of auburn hair---free at the moment from any powder; the sweet, almost childlike mouth, the straight chiselled nose, round chin, and delicate throat, all seemed set off by the picturesque costume of the period. The rich blue velvet robe moulded in its every line the graceful contour of the figure, whilst one tiny hand held, with a dignity all its own, the tall stick adorned with a large bunch of ribbons which fashionable ladies of the period had taken to carrying recently.

With a quick glance all around the room, Marguerite Blakeney had taken stock of everyone there. She nodded pleasantly to Sir Andrew Ffoulkes, whilst extending a hand to Lord Antony.

\enquote{Hello! my Lord Tony, why---what are \textit{you} doing here in Dover?} she said merrily.

Then, without waiting for a reply, she turned and faced the Comtesse and Suzanne. Her whole face lighted up with additional brightness, as she stretched out both arms towards the young girl.

\enquote{Why! if that isn't my little Suzanne over there. \textit{Pardieu}, little citizeness, how came you to be in England? And Madame too!}

She went up effusively to them both, with not a single touch of embarrassment in her manner or in her smile. Lord Tony and Sir Andrew watched the little scene with eager apprehension. English though they were, they had often been in France, and had mixed sufficiently with the French to realise the unbending hauteur, the bitter hatred with which the old \textit{noblesse} of France viewed all those who had helped to contribute to their downfall. Armand St~Just, the brother of beautiful Lady Blakeney---though known to hold moderate and conciliatory views---was an ardent republican; his feud with the ancient family of St~Cyr---the rights and wrongs of which no outsider ever knew---had culminated in the downfall, the almost total extinction, of the latter. In France, St~Just and his party had triumphed, and here in England, face to face with these three refugees driven from their country, flying for their lives, bereft of all which centuries of luxury had given them, there stood a fair scion of those same republican families which had hurled down a throne, and uprooted an aristocracy whose origin was lost in the dim and distant vista of bygone centuries.

She stood there before them, in all the unconscious insolence of beauty, and stretched out her dainty hand to them, as if she would, by that one act, bridge over the conflict and bloodshed of the past decade.

\enquote{Suzanne, I forbid you to speak to that woman,} said the Comtesse, sternly, as she placed a restraining hand upon her daughter's arm.

She had spoken in English, so that all might hear and understand; the two young English gentlemen as well as the common innkeeper and his daughter. The latter literally gasped with horror at this foreign insolence, this impudence before her ladyship---who was English, now that she was Sir Percy's wife, and a friend of the Princess of Wales to boot.

As for Lord Antony and Sir Andrew Ffoulkes, their very hearts seemed to stand still with horror at this gratuitous insult. One of them uttered an exclamation of appeal, the other one of warning, and instinctively both glanced hurriedly towards the door, whence a slow, drawly, not unpleasant voice had already been heard.

Alone among those present Marguerite Blakeney and the Comt\-esse de Tournay had remained seemingly unmoved. The latter, rigid, erect and defiant, with one hand still upon her daughter's arm, seemed the very personification of unbending pride. For the moment Marguerite's sweet face had become as white as the soft fichu which swathed her throat, and a very keen observer might have noted that the hand which held the tall, beribboned stick was clenched, and trembled somewhat.

But this was only momentary; the next instant the delicate eyebrows were raised slightly, the lips curved sarcastically upwards, the clear blue eyes looked straight at the rigid Comtesse, and with a slight shrug of the shoulders---

\enquote{Hoity-toity, citizeness,} she said gaily, \enquote{what fly stings you, pray?}

\enquote{We are in England now, Madame,} rejoined the Comtesse, coldly, \enquote{and I am at liberty to forbid my daughter to touch your hand in friendship. Come, Suzanne.}

She beckoned to her daughter, and without another look at Marguerite Blakeney, but with a deep, old-fashioned curtsey to the two young men, she sailed majestically out of the room.

There was silence in the old inn parlour for a moment, as the rustle of the Comtesse's skirts died away down the passage. Marguerite, rigid as a statue, followed with hard, set eyes the upright figure, as it disappeared through the doorway---but as little Suzanne, humble and obedient, was about to follow her mother, the hard, set expression suddenly vanished, and a wistful, almost pathetic and childlike look stole into Lady Blakeney's eyes.

Little Suzanne caught that look; the child's sweet nature went out to the beautiful woman, scarce older than herself; filial obedience vanished before girlish sympathy; at the door she turned, ran back to Marguerite, and putting her arms round her, kissed her effusively; then only did she follow her mother, Sally bringing up the rear, with a pleasant smile on her dimpled face, and with a final curtsey to my lady.

Suzanne's sweet and dainty impulse had relieved the unpleasant tension. Sir Andrew's eyes followed the pretty little figure, until it had quite disappeared, then they met Lady Blakeney's with unassumed merriment.

Marguerite, with dainty affectation, had kissed her hand to the ladies, as they disappeared through the door, then a humorous smile began hovering round the corners of her mouth.

\enquote{So that's it, is it?} she said gaily. \enquote{La! Sir Andrew, did you ever see such an unpleasant person? I hope when I grow old I sha'n’t look like that.}

She gathered up her skirts, and assuming a majestic gait, stalked towards the fireplace.

\enquote{Suzanne,} she said, mimicking the Comtesse's voice, \enquote{I forbid you to speak to that woman!}

The laugh which accompanied this sally sounded perhaps a trifle forced and hard, but neither Sir Andrew nor Lord Tony were very keen observers. The mimicry was so perfect, the tone of the voice so accurately reproduced, that both the young men joined in a hearty cheerful \enquote{Bravo!}

\enquote{Ah! Lady Blakeney!} added Lord Tony, \enquote{how they must miss you at the Comédie Française, and how the Parisians must hate Sir Percy for having taken you away.}

\enquote{Lud, man,} rejoined Marguerite, with a shrug of her graceful shoul\-ders, \enquote{’tis impossible to hate Sir Percy for anything; his witty sallies would disarm even Madame la Comtesse herself.}

The young Vicomte, who had not elected to follow his mother in her dignified exit, now made a step forward, ready to champion the Comtesse should Lady Blakeney aim any further shafts at her. But before he could utter a preliminary word of protest, a pleasant, though distinctly inane laugh, was heard from outside, and the next moment an unusually tall and very richly dressed figure appeared in the doorway.