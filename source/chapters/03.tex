%!TeX root=../scarlettop.tex

\chapter{The Refugees}
\lettrine[lines=4]{F}{eeling} in every part of England certainly ran very high at this time against the French and their doings. Smugglers and legitimate traders between the French and English coasts brought snatches of news from over the water, which made every honest Englishman's blood boil, and made him long to have »a good go« at those murderers, who had imprisoned their king and all his family, subjected the queen and the royal children to every species of indignity, and were even now loudly demanding the blood of the whole Bourbon family and of every one of its adherents.

The execution of the Princesse de Lamballe, Marie Antoinette's young and charming friend, had filled everyone in England with unspeakable horror, the daily execution of scores of royalists of good family, whose only sin was their aristocratic name, seemed to cry for vengeance to the whole of civilised Europe.

Yet, with all that, no one dared to interfere. Burke had exhausted all his eloquence in trying to induce the British Government to fight the revolutionary government of France, but Mr Pitt, with characteristic prudence, did not feel that this country was fit yet to embark on another arduous and costly war. It was for Austria to take the initiative; Austria, whose fairest daughter was even now a dethroned queen, imprisoned and insulted by a howling mob; and surely 'twas not—so argued Mr Fox—for the whole of England to take up arms, because one set of Frenchmen chose to murder another.

As for Mr Jellyband and his fellow John Bulls, though they looked upon all foreigners with withering contempt, they were royalist and anti-revolutionists to a man, and at this present moment were furious with Pitt for his caution and moderation, although they naturally understood nothing of the diplomatic reasons which guided that great man's policy.

But now Sally came running back, very excited and very eager. The joyous company in the coffee-room had heard nothing of the noise outside, but she had spied a dripping horse and rider who had stopped at the door of »The Fisherman's Rest,« and while the stable boy ran forward to take charge of the horse, pretty Miss Sally went to the front door to greet the welcome visitor.

»I think I see'd my Lord Antony's horse out in the yard, father,« she said, as she ran across the coffee-room.

But already the door had been thrown open from outside, and the next moment an arm, covered in drab cloth and dripping with the heavy rain, was round pretty Sally's waist, while a hearty voice echoed along the polished rafters of the coffee-room.

»Aye, and bless your brown eyes for being so sharp, my pretty Sally,« said the man who had just entered, whilst worthy Mr Jellyband came bustling forward, eager, alert and fussy, as became the advent of one of the most favoured guests of his hostel.

»Lud, I protest, Sally,« added Lord Antony, as he deposited a kiss on Miss Sally's blooming cheeks, »but you are growing prettier and prettier every time I see you—and my honest friend, Jellyband here, must have hard work to keep the fellows off that slim waist of yours. What say you, Mr Waite?«

Mr Waite—torn between his respect for my lord and his dislike of that particular type of joke—only replied with a doubtful grunt.

Lord Antony Dewhurst, one of the sons of the Duke of Exeter, was in those days a very perfect type of a young English gentleman—tall, well set-up, broad of shoulders and merry of face, his laughter rang loudly wherever he went. A good sportsman, a lively companion, a courteous, well-bred man of the world, with not too much brains to spoil his temper, he was a universal favourite in London drawing-rooms or in the coffee-rooms of village inns. At »The Fisherman's Rest« everyone knew him—for he was fond of a trip across to France, and always spent a night under worthy Mr Jellyband's roof on his way there or back.

He nodded to Waite, Pitkin and the others as he at last released Sally's waist, and crossed over to the hearth to warm and dry himself: as he did so, he cast a quick, somewhat suspicious glance at the two strangers, who had quietly resumed their game of dominoes, and for a moment a look of deep earnestness, even of anxiety, clouded his jovial young face.

But only for a moment; the next he had turned to Mr Hempseed, who was respectfully touching his forelock.

»Well, Mr Hempseed, and how is the fruit?«

»Badly, my lord, badly,« replied Mr Hempseed, dolefully, »but what can you 'xpect with this 'ere government favourin' them rascals over in France, who would murder their king and all their nobility.«

»Odd's life!« retorted Lord Antony; »so they would, honest Hempseed,—at least those they can get hold of, worse luck! But we have got some friends coming here to-night, who at any rate have evaded their clutches.«

It almost seemed, when the young man said these words, as if he threw a defiant look towards the quiet strangers in the corner.

»Thanks to you, my lord, and to your friends, so I've heard it said,« said Mr Jellyband.

But in a moment Lord Antony's hand fell warningly on mine host's arm.

»Hush!« he said peremptorily, and instinctively once again looked towards the strangers.

»Oh! Lud love you, they are all right, my lord,« retorted Jellyband; »don't you be afraid. I wouldn't have spoken, only I knew we were among friends. That gentleman over there is as true and loyal a subject of King George as you are yourself, my lord, saving your presence. He is but lately arrived in Dover, and is settling down in business in these parts.«

»In business? Faith, then, it must be as an undertaker, for I vow I never beheld a more rueful countenance.«

»Nay, my lord, I believe that the gentleman is a widower, which no doubt would account for the melancholy of his bearing—but he is a friend, nevertheless, I'll vouch for that—and you will own, my lord, that who should judge of a face better than the landlord of a popular inn\longdash«

»Oh, that's all right, then, if we are among friends,« said Lord Antony, who evidently did not care to discuss the subject with his host. »But, tell me, you have no one else staying here, have you?«

»No one, my lord, and no one coming, either,  leastways\longdash«

»Leastways?«

»No one your lordship would object to, I know.«

»Who is it?«

»Well, my lord, Sir Percy Blakeney and his lady will be here presently, but they ain't a-goin' to stay\longdash«

»Lady Blakeney?« queried Lord Antony, in some astonishment.

»Aye, my lord. Sir Percy's skipper was here just now. He says that my lady's brother is crossing over to France to-day in the \textit{Day Dream}, which is Sir Percy's yacht, and Sir Percy and my lady will come with him as far as here to see the last of him. It don't put you out, do it, my lord?«

»No, no, it doesn't put me out, friend; nothing will put me out, unless that supper is not the very best which Miss Sally can cook, and which has ever been served in »The Fisherman's Rest.««

»You need have no fear of that, my lord,« said Sally, who all this while had been busy setting the table for supper. And very gay and inviting it looked, with a large bunch of brilliantly coloured dahlias in the centre, and the bright pewter goblets and blue china about.

»How many shall I lay for, my lord?«

»Five places, pretty Sally, but let the supper be enough for ten at least—our friends will be tired, and, I hope, hungry. As for me, I vow I could demolish a baron of beef to-night.«

»Here they are, I do believe,« said Sally, excitedly, as a distant clatter of horses and wheels could now be distinctly heard, drawing rapidly nearer.

There was general commotion in the coffee-room. Everyone was curious to see my Lord Antony's swell friends from over the water. Miss Sally cast one or two quick glances at the little bit of mirror which hung on the wall, and worthy Mr Jellyband bustled out in order to give the first welcome himself to his distinguished guests. Only the two strangers in the corner did not participate in the general excitement. They were calmly finishing their game of dominoes, and did not even look once towards the door.

»Straight ahead, Comtesse, the door on your right,« said a pleasant voice outside.

»Aye! there they are, all right enough,« said Lord Antony, joyfully; »off with you, my pretty Sally, and see how quickly you can dish up the soup.«

The door was thrown wide open, and, preceded by Mr Jellyband, who was profuse in his bows and welcomes, a party of four—two ladies and two gentlemen—entered the coffee-room.

»Welcome! Welcome to old England!« said Lord Antony, effusively, as he came eagerly forward with both hands outstretched towards the newcomers.

»Ah, you are Lord Antony Dewhurst, I think,« said one of the ladies, speaking with a strong foreign accent.

»At your service, Madame,« he replied, as he ceremoniously kissed the hands of both the ladies, then turned to the men and shook them both warmly by the hand.

Sally was already helping the ladies to take off their travelling cloaks, and both turned, with a shiver, towards the brightly-blazing hearth.

There was a general movement among the company in the coffee-room. Sally had bustled off to her kitchen, whilst Jellyband, still profuse with his respectful salutations, arranged one or two chairs around the fire. Mr Hempseed, touching his forelock, was quietly vacating the seat in the hearth. Everyone was staring curiously, yet deferentially, at the foreigners.

»Ah, Messieurs! what can I say?« said the elder of the two ladies, as she stretched a pair of fine, aristocratic hands to the warmth of the blaze, and looked with unspeakable gratitude first at Lord Antony, then at one of the young men who had accompanied her party, and who was busy divesting himself of his heavy, caped coat.

»Only that you are glad to be in England, Comtesse,« replied Lord Antony, »and that you have not suffered too much from your trying voyage.«

»Indeed, indeed, we are glad to be in England,« she said, while her eyes filled with tears, »and we have already forgotten all that we have suffered.«

Her voice was musical and low, and there was a great deal of calm dignity and of many sufferings nobly endured marked in the handsome, aristocratic face, with its wealth of snow-white hair dressed high above the forehead, after the fashion of the times.

»I hope my friend, Sir Andrew Ffoulkes, proved an entertaining travelling companion, Madame?«

»Ah, indeed, Sir Andrew was kindness itself. How could my children and I ever show enough gratitude to you all, Messieurs?«

Her companion, a dainty, girlish figure, childlike and pathetic in its look of fatigue and of sorrow, had said nothing as yet, but her eyes, large, brown, and full of tears, looked up from the fire and sought those of Sir Andrew Ffoulkes, who had drawn near to the hearth and to her; then, as they met his, which were fixed with unconcealed admiration upon the sweet face before him, a thought of warmer colour rushed up to her pale cheeks.

»So this is England,« she said, as she looked round with childlike curiosity at the great open hearth, the oak rafters, and the yokels with their elaborate smocks and jovial, rubicund, British countenances.

»A bit of it, Mademoiselle,« replied Sir Andrew, smiling, »but all of it, at your service.«

The young girl blushed again, but this time a bright smile, fleet and sweet, illumined her dainty face. She said nothing, and Sir Andrew too was silent, yet those two young people understood one another, as young people have a way of doing all the world over, and have done since the world began.

»But, I say, supper!« here broke in Lord Antony's jovial voice, »supper, honest Jellyband. Where is that pretty wench of yours and the dish of soup? Zooks, man, while you stand there gaping at the ladies, they will faint with hunger.«

»One moment! one moment, my lord,« said Jellyband, as he threw open the door that led to the kitchen and shouted lustily: »Sally! Hey, Sally there, are ye ready, my girl?«

Sally was ready, and the next moment she appeared in the doorway carrying a gigantic tureen, from which rose a cloud of steam and an abundance of savoury odour.

»Odd's my life, supper at last!« ejaculated Lord Antony, merrily, as he gallantly offered his arm to the Comtesse.

»May I have the honour?« he added ceremoniously, as he led her towards the supper table.

There was general bustle in the coffee-room: Mr Hempseed and most of the yokels and fisher-folk had gone to make way for »the quality,« and to finish smoking their pipes elsewhere. Only the two strangers stayed on, quietly and unconcernedly playing their game of dominoes and sipping their wine; whilst at another table Harry Waite, who was fast losing his temper, watched pretty Sally bustling round the table.

She looked a very dainty picture of English rural life, and no wonder that the susceptible young Frenchman could scarce take his eyes off her pretty face. The Vicomte de Tournay was scarce nineteen, a beardless boy, on whom the terrible tragedies which were being enacted in his own country had made but little impression. He was elegantly and even foppishly dressed, and once safely landed in England he was evidently ready to forget the horrors of the Revolution in the delights of English life.

»Pardi, if zis is England,« he said as he continued to ogle Sally with marked satisfaction, »I am of it satisfied.«

It would be impossible at this point to record the exact exclamation which escaped through Mr Harry Waite's clenched teeth. Only respect for »the quality,« and notably for my Lord Antony, kept his marked disapproval of the young foreigner in check.

»Nay, but this \textit{is} England, you abandoned young reprobate,« interposed Lord Antony with a laugh, »and do not, I pray, bring your loose foreign ways into this most moral country.«

Lord Antony had already sat down at the head of the table with the Comtesse on his right. Jellyband was bustling round, filling glasses and putting chairs straight. Sally waited, ready to hand round the soup. Mr Harry Waite's friends had at last succeeded in taking him out of the room, for his temper was growing more and more violent under the Vicomte's obvious admiration for Sally.

»Suzanne,« came in stern, commanding accents from the rigid Com\-tesse.

Suzanne blushed again; she had lost count of time and of place whilst she had stood beside the fire, allowing the handsome young Englishman's eyes to dwell upon her sweet face, and his hand, as if unconsciously, to rest upon hers. Her mother's voice brought her back to reality once more, and with a submissive »Yes, Mama,« she too took her place at the supper table.
