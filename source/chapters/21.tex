%!TeX root=../scarlettop.tex

\chapter{Suspense}
\lettrine[lines=4]{I}{t} was late into the night when she at last reached \enquote{The Fisherman's Rest.} She had done the whole journey in less than eight hours, thanks to innumerable changes of horses at the various coaching stations, for which she always paid lavishly, thus obtaining the very best and swiftest that could be had.

Her coachman, too, had been indefatigable; the promise of special and rich reward had no doubt helped to keep him up, and he had literally burned the ground beneath his mistress’ coach wheels.

The arrival of Lady Blakeney in the middle of the night caused a considerable flutter at \enquote{The Fisherman's Rest.} Sally jumped hastily out of bed, and Mr Jellyband was at great pains how to make his important guest comfortable.

Both these good folk were far too well drilled in the manners appertaining to innkeepers, to exhibit the slightest surprise at Lady Blakeney's arrival, alone, at this extraordinary hour. No doubt they thought all the more, but Marguerite was far too absorbed in the importance---the deadly earnestness---of her journey, to stop and ponder over trifles of that sort.

The coffee-room---the scene lately of the dastardly outrage on two English gentlemen---was quite deserted. Mr Jellyband hastily relit the lamp, rekindled a cheerful bit of fire in the great hearth, and then wheeled a comfortable chair by it, into which Marguerite gratefully sank.

\enquote{Will your ladyship stay the night?} asked pretty Miss Sally, who was already busy laying a snow-white cloth on the table, preparatory to providing a simple supper for her ladyship.

\enquote{No! not the whole night,} replied Marguerite. \enquote{At any rate, I shall not want any room but this, if I can have it to myself for an hour or two.}

\enquote{It is at your ladyship's service,} said honest Jellyband, whose rubicund face was set in its tightest folds, lest it should betray before \enquote{the quality} that boundless astonishment which the worthy fellow had begun to feel.

\enquote{I shall be crossing over at the first turn of the tide,} said Marguerite, \enquote{and in the first schooner I can get. But my coachman and men will stay the night, and probably several days longer, so I hope you will make them comfortable.}

\enquote{Yes, my lady; I'll look after them. Shall Sally bring your ladyship some supper?}

\enquote{Yes, please. Put something cold on the table, and as soon as Sir Andrew Ffoulkes comes, show him in here.}

\enquote{Yes, my lady.}

Honest Jellyband's face now expressed distress in spite of himself. He had great regard for Sir Percy Blakeney, and did not like to see his lady running away with young Sir Andrew. Of course, it was no business of his, and Mr Jellyband was no gossip. Still, in his heart, he recollected that her ladyship was after all only one of them \enquote{furriners}; what wonder that she was immoral like the rest of them?

\enquote{Don't sit up, honest Jellyband,} continued Marguerite, kindly, \enquote{nor you either, Mistress Sally. Sir Andrew may be late.}

Jellyband was only too willing that Sally should go to bed. He was beginning not to like these goings-on at all. Still, Lady Blakeney would pay handsomely for the accommodation, and it certainly was no business of his.

Sally arranged a simple supper of cold meat, wine, and fruit on the table, then with a respectful curtsey, she retired, wondering in her little mind why her ladyship looked so serious, when she was about to elope with her gallant.

Then commenced a period of weary waiting for Marguerite. She knew that Sir Andrew---who would have to provide himself with clothes befitting a lacquey---could not possibly reach Dover for at least a couple of hours. He was a splendid horseman of course, and would make light in such an emergency of the seventy odd miles between London and Dover. He would, too, literally burn the ground beneath his horse's hoofs, but he might not always get very good remounts, and in any case, he could not have started from London until at least an hour after she did.

She had seen nothing of Chauvelin on the road. Her coachman, whom she questioned, had not seen anyone answering the description his mistress gave him of the wizened figure of the little Frenchman.

Evidently, therefore, he had been ahead of her all the time. She had not dared to question the people at the various inns, where they had stopped to change horses. She feared that Chauvelin had spies all along the route, who might overhear her questions, then outdistance her and warn her enemy of her approach.

Now she wondered at what inn he might be stopping, or whether he had had the good luck of chartering a vessel already, and was now himself on the way to France. That thought gripped her at the heart as with an iron vice. If indeed she should be too late already!

The loneliness of the room overwhelmed her; everything within was so horribly still; the ticking of the grandfather's clock---dreadfully slow and measured---was the only sound which broke this awful loneliness.

Marguerite had need of all her energy, all her steadfastness of purpose, to keep up her courage through this weary midnight waiting.

Everyone else in the house but herself must have been asleep. She had heard Sally go upstairs. Mr Jellyband had gone to see to her coachman and men, and then had returned and taken up a position under the porch outside, just where Marguerite had first met Chauvelin about a week ago. He evidently meant to wait up for Sir Andrew Ffoulkes, but was soon overcome by sweet slumbers, for presently---in addition to the slow ticking of the clock---Marguerite could hear the monotonous and dulcet tones of the worthy fellow's breathing.

For some time now, she had realised that the beautiful warm October's day, so happily begun, had turned into a rough and cold night. She had felt very chilly, and was glad of the cheerful blaze in the hearth: but gradually, as time wore on, the weather became more rough, and the sound of the great breakers against the Admiralty Pier, though some distance from the inn, came to her as the noise of muffled thunder.

The wind was becoming boisterous, rattling the leaded windows and the massive doors of the old-fashioned house: it shook the trees outside and roared down the vast chimney. Marguerite wondered if the wind would be favourable for her journey. She had no fear of the storm, and would have braved worse risks sooner than delay the crossing by an hour.

A sudden commotion outside roused her from her meditations. Evidently it was Sir Andrew Ffoulkes, just arrived in mad haste, for she heard his horse's hoofs thundering on the flag-stones outside, then Mr Jellyband's sleepy, yet cheerful tones bidding him welcome.

For a moment, then, the awkwardness of her position struck Marguerite; alone at this hour, in a place where she was well known, and having made an assignation with a young cavalier equally well known, and who arrives in disguise! What food for gossip to those mischievously inclined.

The idea struck Marguerite chiefly from its humorous side: there was such quaint contrast between the seriousness of her errand, and the construction which would naturally be put on her actions by honest Mr Jellyband, that, for the first time since many hours, a little smile began playing round the corners of her childlike mouth, and when, presently, Sir Andrew, almost unrecognisable in his lacquey-like garb, entered the coffee-room, she was able to greet him with quite a merry laugh.

\enquote{Faith! Monsieur, my lacquey,} she said, \enquote{I am satisfied with your appearance!}

Mr Jellyband had followed Sir Andrew, looking strangely perplexed. The young gallant's disguise had confirmed his worst suspicions. With\-out a smile upon his jovial face, he drew the cork from the bottle of wine, set the chairs ready, and prepared to wait.

\enquote{Thanks, honest friend,} said Marguerite, who was still smiling at the thought of what the worthy fellow must be thinking at that very moment, \enquote{we shall require nothing more; and here's for all the trouble you have been put to on our account.}

She handed two or three gold pieces to Jellyband, who took them respectfully, and with becoming gratitude.

\enquote{Stay, Lady Blakeney,} interposed Sir Andrew, as Jellyband was about to retire, \enquote{I am afraid we shall require something more of my friend Jelly's hospitality. I am sorry to say we cannot cross over to-night.}

\enquote{Not cross over to-night?} she repeated in amazement. \enquote{But we must, Sir Andrew, we must! There can be no question of cannot, and whatever it may cost, we must get a vessel to-night.}

But the young man shook his head sadly.

\enquote{I am afraid it is not a question of cost, Lady Blakeney. There is a nasty storm blowing from France, the wind is dead against us, we cannot possibly sail until it has changed.}

Marguerite became deadly pale. She had not foreseen this. Nature herself was playing her a horrible, cruel trick. Percy was in danger, and she could not go to him, because the wind happened to blow from the coast of France.

\enquote{But we must go!---we must!} she repeated with strange, persistent energy, \enquote{you know, we must go!---can't you find a way?}

\enquote{I have been down to the shore already,} he said, \enquote{and had a talk to one or two skippers. It is quite impossible to set sail to-night, so every sailor assured me. No one,} he added, looking significantly at Marguerite, \enquote{\textit{no one} could possibly put out of Dover to-night.}

Marguerite at once understood what he meant. \textit{No one} included Chauvelin as well as herself. She nodded pleasantly to Jellyband.

\enquote{Well, then, I must resign myself,} she said to him. \enquote{Have you a room for me?}

\enquote{Oh, yes, your ladyship. A nice, bright, airy room. I'll see to it at once... And there is another one for Sir Andrew---both quite ready.}

\enquote{That's brave now, mine honest Jelly,} said Sir Andrew, gaily, and clapping his worthy host vigorously on the back. \enquote{You unlock both those rooms, and leave our candles here on the dresser. I vow you are dead with sleep, and her ladyship must have some supper before she retires. There, have no fear, friend of the rueful countenance, her ladyship's visit, though at this unusual hour, is a great honour to thy house, and Sir Percy Blakeney will reward thee doubly, if thou seest well to her privacy and comfort.}

Sir Andrew had no doubt guessed the many conflicting doubts and fears which raged in honest Jellyband's head; and, as he was a gallant gentleman, he tried by this brave hint to allay some of the worthy innkeeper's suspicions. He had the satisfaction of seeing that he had partially succeeded. Jellyband's rubicund countenance brightened somewhat, at mention of Sir Percy's name.

\enquote{I'll go and see to it at once, sir,} he said with alacrity, and with less frigidity in his manner. \enquote{Has her ladyship everything she wants for supper?}

\enquote{Everything, thanks, honest friend, and as I am famished and dead with fatigue, I pray you see to the rooms.}

\enquote{Now tell me,} she said eagerly, as soon as Jellyband had gone from the room, \enquote{tell me all your news.}

\enquote{There is nothing else much to tell you, Lady Blakeney,} replied the young man. \enquote{The storm makes it quite impossible for any vessel to put out of Dover this tide. But, what seemed to you at first a terrible calamity is really a blessing in disguise. If we cannot cross over to France to-night, Chauvelin is in the same quandary.}

\enquote{He may have left before the storm broke out.}

\enquote{God grant he may,} said Sir Andrew, merrily, \enquote{for very likely then he'll have been driven out of his course! Who knows? He may now even be lying at the bottom of the sea, for there is a furious storm raging, and it will fare ill with all small craft which happen to be out. But I fear me we cannot build our hopes upon the shipwreck of that cunning devil, and of all his murderous plans. The sailors I spoke to, all assured me that no schooner had put out of Dover for several hours: on the other hand, I ascertained that a stranger had arrived by coach this afternoon, and had, like myself, made some inquiries about crossing over to France.}

\enquote{Then Chauvelin is still in Dover?}

\enquote{Undoubtedly. Shall I go waylay him and run my sword through him? That were indeed the quickest way out of the difficulty.}

\enquote{Nay! Sir Andrew, do not jest! Alas! I have often since last night caught myself wishing for that fiend's death. But what you suggest is impossible! The laws of this country do not permit of murder! It is only in our beautiful France that wholesale slaughter is done lawfully, in the name of Liberty and of brotherly love.}

Sir Andrew had persuaded her to sit down to the table, to partake of some supper and to drink a little wine. This enforced rest of at least twelve hours, until the next tide, was sure to be terribly difficult to bear in the state of intense excitement in which she was. Obedient in these small matters like a child, Marguerite tried to eat and drink.

Sir Andrew, with that profound sympathy born in all those who are in love, made her almost happy by talking to her about her husband. He recounted to her some of the daring escapes the brave Scarlet Pimpernel had contrived for the poor French fugitives, whom a relentless and bloody revolution was driving out of their country. He made her eyes glow with enthusiasm by telling her of his bravery, his ingenuity, his resourcefulness, when it meant snatching the lives of men, women, and even children from beneath the very edge of that murderous, ever-ready guillotine.

He even made her smile quite merrily by telling her of the Scarlet Pimpernel's quaint and many disguises, through which he had baffled the strictest watch set against him at the barricades of Paris. This last time, the escape of the Comtesse de Tournay and her children had been a veritable masterpiece---Blakeney disguised as a hideous old market-woman, in filthy cap and straggling grey locks, was a sight fit to make the gods laugh.

Marguerite laughed heartily as Sir Andrew tried to describe Blakeney's appearance, whose gravest difficulty always consisted in his great height, which in France made disguise doubly difficult.

Thus an hour wore on. There were many more to spend in enforced inactivity in Dover. Marguerite rose from the table with an impatient sigh. She looked forward with dread to the night in the bed upstairs, with terribly anxious thoughts to keep her company, and the howling of the storm to help chase sleep away.

She wondered where Percy was now. The \textit{Day Dream} was a strong, well-built, sea-going yacht. Sir Andrew had expressed the opinion that no doubt she had got in the lee of the wind before the storm broke out, or else perhaps had not ventured into the open at all, but was lying quietly at Gravesend.

Briggs was an expert skipper, and Sir Percy handled a schooner as well as any master mariner. There was no danger for them from the storm.

It was long past midnight when at last Marguerite retired to rest. As she had feared, sleep sedulously avoided her eyes. Her thoughts were of the blackest during these long, weary hours, whilst that incessant storm raged which was keeping her away from Percy. The sound of the distant breakers made her heart ache with melancholy. She was in the mood when the sea has a saddening effect upon the nerves. It is only when we are very happy, that we can bear to gaze merrily upon the vast and limitless expanse of water, rolling on and on with such persistent, irritating monotony, to the accompaniment of our thoughts, whether grave or gay. When they are gay, the waves echo their gaiety; but when they are sad, then every breaker, as it rolls, seems to bring additional sadness, and to speak to us of hopelessness and of the pettiness of all our joys.