%!TeX root=../scarlettop.tex

\chapter{The League of the Scarlet Pimpernel}

\lettrine[lines=4]{T}{hey} all looked a merry, even a happy party, as they sat round the table; Sir Andrew Ffoulkes and Lord Antony Dewhurst, two typical good-looking, well-born and well-bred Englishmen of that year of grace 1792, and the aristocratic French comtesse with her two children, who had just escaped from such dire perils, and found a safe retreat at last on the shores of protecting England.

In the corner the two strangers had apparently finished their game; one of them arose, and standing with his back to the merry company at the table, he adjusted with much deliberation his large triple caped coat. As he did so, he gave one quick glance all around him. Everyone was busy laughing and chatting, and he murmured the words »All safe!«: his companion then, with the alertness borne of long practice, slipped on to his knees in a moment, and the next had crept noiselessly under the oak bench. The stranger then, with a loud »Good-night,« quietly walked out of the coffee-room.

Not one of those at the supper table had noticed this curious and silent manœuvre, but when the stranger finally closed the door of the coffee-room behind him, they all instinctively sighed a sigh of relief.

»Alone, at last!« said Lord Antony, jovially.

Then the young Vicomte de Tournay rose, glass in hand, and with the graceful affectation peculiar to the times, he raised it aloft, and said in broken English,\longdash


»To His Majesty George Three of England. God bless him for his hospitality to us all, poor exiles from France.«

»His Majesty the King!« echoed Lord Antony and Sir Andrew as they drank loyally to the toast.

»To His Majesty King Louis of France,« added Sir Andrew, with solemnity. »May God protect him, and give him victory over his enemies.«

Everyone rose and drank this toast in silence. The fate of the unfortunate King of France, then a prisoner of his own people, seemed to cast a gloom even over Mr Jellyband's pleasant countenance.

»And to M. le Comte de Tournay de Basserive,« said Lord Antony, merrily. »May we welcome him in England before many days are over.«

»Ah, Monsieur,« said the Comtesse, as with a slightly trembling hand she conveyed her glass to her lips, »I scarcely dare to hope.«

But already Lord Antony had served out the soup, and for the next few moments all conversation ceased, while Jellyband and Sally handed round the plates and everyone began to eat.

»Faith, Madame!« said Lord Antony, after a while, »mine was no idle toast; seeing yourself, Mademoiselle Suzanne and my friend the Vicomte safely in England now, surely you must feel reassured as to the fate of Monsieur le Comte.«

»Ah, Monsieur,« replied the Comtesse, with a heavy sigh, »I trust in God\allowbreak---\allowbreak I can but pray\allowbreak---\allowbreak and hope...«

»Aye, Madame!« here interposed Sir Andrew Ffoulkes, »trust in God by all means, but believe also a little in your English friends, who have sworn to bring the Count safely across the Channel, even as they have brought you to-day.«

»Indeed, indeed, Monsieur,« she replied, »I have the fullest confidence in you and in your friends. Your fame, I assure you, has spread throughout the whole of France. The way some of my own friends have escaped from the clutches of that awful revolutionary tribunal was nothing short of a miracle\allowbreak---\allowbreak and all done by you and your  friends\longdash«

»We were but the hands, Madame la Comtesse...«

»But my husband, Monsieur,« said the Comtesse, whilst unshed tears seemed to veil her voice, »he is in such deadly peril\allowbreak---\allowbreak I would never have left him, only... there were my children... I was torn between my duty to him, and to them. They refused to go without me... and you and your friends assured me so solemnly that my husband would be safe. But, oh! now that I am here\allowbreak---\allowbreak amongst you all\allowbreak---\allowbreak in this beautiful, free England\allowbreak---\allowbreak I think of him, flying for his life, hunted like a poor beast... in such peril... Ah! I should not have left him... I should not have left him!...«

The poor woman had completely broken down; fatigue, sorrow and emotion had overmastered her rigid, aristocratic bearing. She was crying gently to herself, whilst Suzanne ran up to her and tried to kiss away her tears.

Lord Antony and Sir Andrew had said nothing to interrupt the Comtesse whilst she was speaking. There was no doubt that they felt deeply for her; their very silence testified to that\allowbreak---\allowbreak but in every century, and ever since England has been what it is, an Englishman has always felt somewhat ashamed of his own emotion and of his own sympathy. And so the two young men said nothing, and busied themselves in trying to hide their feelings, only succeeding in looking immeasurably sheepish.

»As for me, Monsieur,« said Suzanne, suddenly, as she looked through a wealth of brown curls across at Sir Andrew, »I trust you absolutely, and I \textit{know} that you will bring my dear father safely to England, just as you brought us to-day.«

This was said with so much confidence, such unuttered hope and belief, that it seemed as if by magic to dry the mother's eyes, and to bring a smile upon everybody's lips.

»Nay! you shame me, Mademoiselle,« replied Sir Andrew; »though my life is at your service, I have been but a humble tool in the hands of our great leader, who organised and effected your escape.«

He had spoken with so much warmth and vehemence that Suzanne's eyes fastened upon him in undisguised wonder.

»Your leader, Monsieur?« said the Comtesse, eagerly. »Ah! of course, you must have a leader. And I did not think of that before! But tell me where is he? I must go to him at once, and I and my children must throw ourselves at his feet, and thank him for all that he has done for us.«

»Alas, Madame!« said Lord Antony, »that is impossible.«

»Impossible?\allowbreak---\allowbreak Why?«

»Because the Scarlet Pimpernel works in the dark, and his identity is only known under a solemn oath of secrecy to his immediate followers.«

»The Scarlet Pimpernel?« said Suzanne, with a merry laugh. »Why! what a droll name! What is the Scarlet Pimpernel, Monsieur?«

She looked at Sir Andrew with eager curiosity. The young man's face had become almost transfigured. His eyes shone with enthusiasm; hero-worship, love, admiration for his leader seemed literally to glow upon his face.

»The Scarlet Pimpernel, Mademoiselle,« he said at last, »is the name of a humble English wayside flower; but it is also the name chosen to hide the identity of the best and bravest man in all the world, so that he may better succeed in accomplishing the noble task he has set himself to do.«

»Ah, yes,« here interposed the young Vicomte, »I have heard speak of this Scarlet Pimpernel. A little flower\allowbreak---\allowbreak red?\allowbreak---\allowbreak yes! They say in Paris that every time a royalist escapes to England that devil, Foucquier-Tinville, the Public Prosecutor, receives a paper with that little flower dessinated in red upon it... Yes?«

»Yes, that is so,« assented Lord Antony.

»Then he will have received one such paper to-day?«

»Undoubtedly.«

»Oh! I wonder what he will say!« said Suzanne, merrily. »I have heard that the picture of that little red flower is the only thing that frightens him.«

»Faith, then,« said Sir Andrew, »he will have many more opportunities of studying the shape of that small scarlet flower.«

»Ah! Monsieur,« sighed the Comtesse, »it all sounds like a romance, and I cannot understand it all.«

»Why should you try, Madame?«

»But, tell me, why should your leader\allowbreak---\allowbreak why should you all\allowbreak---\allowbreak spend your money and risk your lives\allowbreak---\allowbreak for it is your lives you risk, Messieurs, when you set foot in France\allowbreak---\allowbreak and all for us French men and women, who are nothing to you?«

»Sport, Madame la Comtesse, sport,« asserted Lord Antony, with his jovial, loud and pleasant voice; »we are a nation of sportsmen, you know, and just now it is the fashion to pull the hare from between the teeth of the hound.«

»Ah, no, no, not sport only, Monsieur... you have a more noble motive, I am sure, for the good work you do.«

»Faith, Madame, I would like you to find it then... as for me, I vow, I love the game, for this is the finest sport I have yet encountered.\allowbreak---\allowbreak Hair-breadth escapes... the devil's own risks!\allowbreak---\allowbreak Tally ho!\allowbreak---\allowbreak and away we go!«

But the Comtesse shook her head, still incredulously. To her it seemed preposterous that these young men and their great leader, all of them rich, probably well-born, and young, should for no other motive than sport, run the terrible risks, which she knew they were constantly doing. Their nationality, once they had set foot in France, would be no safeguard to them. Anyone found harbouring or assisting suspected royalists would be ruthlessly condemned and summarily executed, whatever his nationality might be. And this band of young Englishmen had, to her own knowledge, bearded the implacable and bloodthirsty tribunal of the Revolution, within the very walls of Paris itself, and had snatched away condemned victims, almost from the very foot of the guillotine. With a shudder, she recalled the events of the last few days, her escape from Paris with her two children, all three of them hidden beneath the hood of a rickety cart, and lying amidst a heap of turnips and cabbages, not daring to breathe, whilst the mob howled »À la lanterne les aristos!« at that awful West Barricade.

It had all occurred in such a miraculous way; she and her husband had understood that they had been placed on the list of »suspected persons,« which meant that their trial and death were but a matter of days\allowbreak---\allowbreak of hours, perhaps.

Then came the hope of salvation; the mysterious epistle, signed with the enigmatical scarlet device; the clear, peremptory directions; the parting from the Comte de Tournay, which had torn the poor wife's heart in two; the hope of reunion; the flight with her two children; the covered cart; that awful hag driving it, who looked like some horrible evil demon, with the ghastly trophy on her whip handle!

The Comtesse looked round at the quaint, old-fashioned English inn, the peace of this land of civil and religious liberty, and she closed her eyes to shut out the haunting vision of that West Barricade, and of the mob retreating panic-stricken when the old hag spoke of the plague.

Every moment under that cart she expected recognition, arrest, herself and her children tried and condemned, and these young Englishmen, under the guidance of their brave and mysterious leader, had risked their lives to save them all, as they had already saved scores of other innocent people.

And all only for sport? Impossible! Suzanne's eyes as she sought those of Sir Andrew plainly told him that she thought that \textit{he} at any rate rescued his fellow-men from terrible and unmerited death, through a higher and nobler motive than his friend would have her believe.

»How many are there in your brave league, Monsieur?« she asked timidly.

»Twenty all told, Mademoiselle,« he replied, »one to command, and nineteen to obey. All of us Englishmen, and all pledged to the same cause\allowbreak---\allowbreak to obey our leader and to rescue the innocent.«

»May God protect you all, Messieurs,« said the Comtesse, fervently.

»He has done that so far, Madame.«

»It is wonderful to me, wonderful!\allowbreak---\allowbreak That you should all be so brave, so devoted to your fellow-men\allowbreak---\allowbreak yet you are English!\allowbreak---\allowbreak and in France treachery is rife\allowbreak---\allowbreak all in the name of liberty and fraternity.«

»The women even, in France, have been more bitter against us aristocrats than the men,« said the Vicomte, with a sigh.

»Ah, yes,« added the Comtesse, whilst a look of haughty disdain and intense bitterness shot through her melancholy eyes. »There was that woman, Marguerite St~Just, for instance. She denounced the Marquis de St~Cyr and all his family to the awful tribunal of the Terror.«

»Marguerite St~Just?« said Lord Antony, as he shot a quick and apprehensive glance across at Sir Andrew. »Marguerite St~Just?\allowbreak---\allowbreak  Surely...«

»Yes!« replied the Comtesse, »surely you know her. She was a leading actress of the Comédie Française, and she married an Englishman lately. You must know her\longdash«

»Know her?« said Lord Antony. »Know Lady Blakeney\allowbreak---\allowbreak the most fashionable woman in London\allowbreak---\allowbreak the wife of the richest man in England? Of course, we all know Lady Blakeney.«

»She was a school-fellow of mine at the convent in Paris,« interposed Suzanne, »and we came over to England together to learn your language. I was very fond of Marguerite, and I cannot believe that she ever did anything so wicked.«

»It certainly seems incredible,« said Sir Andrew. »You say that she actually denounced the Marquis de St~Cyr? Why should she have done such a thing? Surely there must be some mistake\longdash«

»No mistake is possible, Monsieur,« rejoined the Comtesse, coldly. »Marguerite St~Just's brother is a noted republican. There was some talk of a family feud between him and my cousin, the Marquis de St~Cyr. The St~Justs are quite plebeian, and the republican government employs many spies. I assure you there is no mistake... You had not heard this story?«

»Faith, Madame, I did hear some vague rumours of it, but in England no one would credit it... Sir Percy Blakeney, her husband, is a very wealthy man, of high social position, the intimate friend of the Prince of Wales... and Lady Blakeney leads both fashion and society in London.«

»That may be, Monsieur, and we shall, of course, lead a very quiet life in England, but I pray God that while I remain in this beautiful country, I may never meet Marguerite St~Just.«

The proverbial wet-blanket seemed to have fallen over the merry little company gathered round the table. Suzanne looked sad and silent; Sir Andrew fidgeted uneasily with his fork, whilst the Comtesse, encased in the plate-armour of her aristocratic prejudices, sat, rigid and unbending, in her straight-backed chair. As for Lord Antony, he looked extremely uncomfortable, and glanced once or twice apprehensively towards Jellyband, who looked just as uncomfortable as himself.

»At what time do you expect Sir Percy and Lady Blakeney?« he contrived to whisper unobserved, to mine host.

»Any moment, my lord,« whispered Jellyband in reply.

Even as he spoke, a distant clatter was heard of an approaching coach; louder and louder it grew, one or two shouts became distinguishable, then the rattle of horses' hoofs on the uneven cobble stones, and the next moment a stable boy had thrown open the coffee-room door and rushed in excitedly.

»Sir Percy Blakeney and my lady,« he shouted at the top of his voice, »they're just arriving.«

And with more shouting, jingling of harness, and iron hoofs upon the stones, a magnificent coach, drawn by four superb bays, had halted outside the porch of »The Fisherman's Rest.«
