%!TeX root=../scarlettop.tex

\chapter{Farewell}
\lettrine[lines=4]{W}{hen} Marguerite reached her room, she found her maid terribly anxious about her. \enquote{Your ladyship will be so tired,} said the poor woman, whose own eyes were half closed with sleep. \enquote{It is past five o'clock.}

\enquote{Ah, yes, Louise, I daresay I shall be tired presently,} said Marguerite, kindly; \enquote{but you are very tired now, so go to bed at once. I'll get into bed alone.}

\enquote{But, my lady...}

\enquote{Now, don't argue, Louise, but go to bed. Give me a wrap, and leave me alone.}

Louise was only too glad to obey. She took off her mistress's gorgeous ball-dress, and wrapped her up in a soft billowy gown.

\enquote{Does your ladyship wish for anything else?} she asked, when that was done.

\enquote{No, nothing more. Put out the lights as you go out.}

\enquote{Yes, my lady. Good-night, my lady.}

\enquote{Good-night, Louise.}

When the maid was gone, Marguerite drew aside the curtains and threw open the windows. The garden and the river beyond were flooded with rosy light. Far away to the east, the rays of the rising sun had changed the rose into vivid gold. The lawn was deserted now, and Marguerite looked down upon the terrace where she had stood a few moments ago trying vainly to win back a man's love, which once had been so wholly hers.

It was strange that through all her troubles, all her anxiety for Armand, she was mostly conscious at the present moment of a keen and bitter heartache.

Her very limbs seemed to ache with longing for the love of a man who had spurned her, who had resisted her tenderness, remained cold to her appeals, and had not responded to the glow of passion, which had caused her to feel and hope that those happy olden days in Paris were not all dead and forgotten.

How strange it all was! She loved him still. And now that she looked back upon the last few months of misunderstandings and of loneliness, she realised that she had never ceased to love him; that deep down in her heart she had always vaguely felt that his foolish inanities, his empty laugh, his lazy nonchalance were nothing but a mask; that the real man, strong, passionate, wilful, was there still---the man she had loved, whose intensity had fascinated her, whose personality attracted her, since she always felt that behind his apparently slow wits there was a certain something, which he kept hidden from all the world, and most especially from her.

A woman's heart is such a complex problem---the owner thereof is often most incompetent to find the solution of this puzzle.

Did Marguerite Blakeney, \enquote{the cleverest woman in Europe,} really love a fool? Was it love that she had felt for him a year ago when she married him? Was it love she felt for him now that she realised that he still loved her, but that he would not become her slave, her passionate, ardent lover once again? Nay! Marguerite herself could not have told that. Not at this moment at any rate; perhaps her pride had sealed her mind against a better understanding of her own heart. But this she did know---that she meant to capture that obstinate heart back again. That she would conquer once more... and then, that she would never lose him... She would keep him, keep his love, deserve it, and cherish it; for this much was certain, that there was no longer any happiness possible for her without that one man's love.

Thus the most contradictory thoughts and emotions rushed madly through her mind. Absorbed in them, she had allowed time to slip by; perhaps, tired out with long excitement, she had actually closed her eyes and sunk into a troubled sleep, wherein quickly fleeting dreams seemed but the continuation of her anxious  thoughts---when suddenly she was roused, from dream or meditation, by the noise of footsteps outside her door.

Nervously she jumped up and listened; the house itself was as still as ever; the footsteps had retreated. Through her wide-open windows the brilliant rays of the morning sun were flooding her room with light. She looked up at the clock; it was half-past six---too early for any of the household to be already astir.

She certainly must have dropped asleep, quite unconsciously. The noise of the footsteps, also of hushed, subdued voices had awakened her---what could they be?

Gently, on tip-toe, she crossed the room and opened the door to listen; not a sound---that peculiar stillness of the early morning when sleep with all mankind is at its heaviest. But the noise had made her nervous, and when, suddenly, at her feet, on the very doorstep, she saw something white lying there---a letter evidently---she hardly dared touch it. It seemed so ghostlike. It certainly was not there when she came upstairs; had Louise dropped it? or was some tantalising spook at play, showing her fairy letters where none existed?

At last she stooped to pick it up, and, amazed, puzzled beyond measure, she saw that the letter was addressed to herself in her husband's large, businesslike-looking hand. What could he have to say to her, in the middle of the night, which could not be put off until the morning?

She tore open the envelope and read:---
\blockquote{
A most unforeseen circumstance forces me to leave for the North immediately, so I beg your ladyship's pardon if I do not avail myself of the honour of bidding you good-bye. My business may keep me employed for about a week, so I shall not have the privilege of being present at your ladyship's water-party on Wednesday.
{\indent I remain your ladyship's most humble and most obedient servant,}\\
\begin{flushright}
\textsc{Percy Blakeney.}
\end{flushright}
}
Marguerite must suddenly have been imbued with her husband's slowness of intellect, for she had perforce to read the few simple lines over and over again, before she could fully grasp their meaning.

She stood on the landing, turning over and over in her hand this curt and mysterious epistle, her mind a blank, her nerves strained with agitation and a presentiment she could not very well have explained.

Sir Percy owned considerable property in the North, certainly, and he had often before gone there alone and stayed away a week at a time; but it seemed so very strange that circumstances should have arisen between five and six o'clock in the morning that compelled him to start in this extreme hurry.

Vainly she tried to shake off an unaccustomed feeling of nervousness: she was trembling from head to foot. A wild, unconquerable desire seized her to see her husband again, at once, if only he had not already started.

Forgetting the fact that she was only very lightly clad in a morning wrap, and that her hair lay loosely about her shoulders, she flew down the stairs, right through the hall towards the front door.

It was as usual barred and bolted, for the indoor servants were not yet up; but her keen ears had detected the sound of voices and the pawing of a horse's hoof against the flag-stones.

With nervous, trembling fingers Marguerite undid the bolts one by one, bruising her hands, hurting her nails, for the locks were heavy and stiff. But she did not care; her whole frame shook with anxiety at the very thought that she might be too late; that he might have gone without her seeing him and bidding him \enquote{God-speed!}

At last, she had turned the key and thrown open the door. Her ears had not deceived her. A groom was standing close by holding a couple of horses; one of these was Sultan, Sir Percy's favourite and swiftest horse, saddled ready for a journey.

The next moment Sir Percy himself appeared round the further corner of the house and came quickly towards the horses. He had changed his gorgeous ball costume, but was as usual irreproachably and richly apparelled in a suit of fine cloth, with lace jabot and ruffles, high top-boots, and riding breeches.

Marguerite went forward a few steps. He looked up and saw her. A slight frown appeared between his eyes.

\enquote{You are going?} she said quickly and feverishly. \enquote{Whither?}

\enquote{As I have had the honour of informing your ladyship, urgent, most unexpected business calls me to the North this morning,} he said, in his usual cold, drawly manner.

\enquote{But... your guests to-morrow...}

\enquote{I have prayed your ladyship to offer my humble excuses to His Royal Highness. You are such a perfect hostess, I do not think that I shall be missed.}

\enquote{But surely you might have waited for your journey... until after our water-party...} she said, still speaking quickly and nervously. \enquote{Surely the business is not so urgent... and you said nothing about it---just now.}

\enquote{My business, as I had the honour to tell you, Madame, is as unexpected as it is urgent... May I therefore crave your permission to go... Can I do aught for you in town?... on my way back?}

\enquote{No... no... thanks... nothing... But you will be back soon?}

\enquote{Very soon.}

\enquote{Before the end of the week?}

\enquote{I cannot say.}

He was evidently trying to get away, whilst she was straining every nerve to keep him back for a moment or two.

\enquote{Percy,} she said, \enquote{will you not tell me why you go to-day? Surely I, as your wife, have the right to know. You have \textit{not} been called away to the North. I know it. There were no letters, no couriers from there before we left for the opera last night, and nothing was waiting for you when we returned from the ball... You are \textit{not} going to the North, I feel convinced... There is some mystery... and...}

\enquote{Nay, there is no mystery, Madame,} he replied, with a slight tone of impatience. \enquote{My business has to do with Armand... there! Now, have I your leave to depart?}

\enquote{With Armand?... But you will run no danger?}

\enquote{Danger? I?... Nay, Madame, your solicitude does me honour. As you say, I have some influence; my intention is to exert it before it be too late.}

\enquote{Will you allow me to thank you at least?}

\enquote{Nay, Madame,} he said coldly, \enquote{there is no need for that. My life is at your service, and I am already more than repaid.}

\enquote{And mine will be at yours, Sir Percy, if you will but accept it, in exchange for what you do for Armand,} she said, as, impulsively, she stretched out both her hands to him. \enquote{There! I will not detain you... my thoughts go with you... Farewell!...}

How lovely she looked in this morning sunlight, with her ardent hair streaming around her shoulders. He bowed very low and kissed her hand; she felt the burning kiss and her heart thrilled with joy and hope.

\enquote{You will come back?} she said tenderly.

\enquote{Very soon!} he replied, looking longingly into her blue eyes.

\enquote{And... you will remember?...} she asked as her eyes, in response to his look, gave him an infinity of promise.

\enquote{I will always remember, Madame, that you have honoured me by commanding my services.}

The words were cold and formal, but they did not chill her this time. Her woman's heart had read his, beneath the impassive mask his pride still forced him to wear.

He bowed to her again, then begged her leave to depart. She stood on one side whilst he jumped on to Sultan's back, then, as he galloped out of the gates, she waved him a final \enquote{Adieu.}

A bend in the road soon hid him from view; his confidential groom had some difficulty in keeping pace with him, for Sultan flew along in response to his master's excited mood. Marguerite, with a sigh that was almost a happy one, turned and went within. She went back to her room, for suddenly, like a tired child, she felt quite sleepy.

Her heart seemed all at once to be in complete peace, and, though it still ached with undefined longing, a vague and delicious hope soothed it as with a balm.

She felt no longer anxious about Armand. The man who had just ridden away, bent on helping her brother, inspired her with complete confidence in his strength and in his power. She marvelled at herself for having ever looked upon him as an inane fool; of course, \textit{that} was a mask worn to hide the bitter wound she had dealt to his faith and to his love. His passion would have overmastered him, and he would not let her see how much he still cared and how deeply he suffered.

But now all would be well: she would crush her own pride, humble it before him, tell him everything, trust him in everything; and those happy days would come back, when they used to wander off together in the forests of Fontainebleau, when they spoke little---for he was always a silent man---but when she felt that against that strong heart she would always find rest and happiness.

The more she thought of the events of the past night, the less fear had she of Chauvelin and his schemes. He had failed to discover the identity of the Scarlet Pimpernel, of that she felt sure. Both Lord Fancourt and Chauvelin himself had assured her that no one had been in the dining-room at one o'clock except the Frenchman himself and Percy---Yes!---Percy! she might have asked him, had she thought of it! Anyway, she had no fears that the unknown and brave hero would fall in Chauvelin's trap; his death at any rate would not be at her door.

Armand certainly was still in danger, but Percy had pledged his word that Armand would be safe, and somehow, as Marguerite had seen him riding away, the possibility that he could fail in whatever he undertook never even remotely crossed her mind. When Armand was safely over in England she would not allow him to go back to France.

She felt almost happy now, and, drawing the curtains closely together again to shut out the piercing sun, she went to bed at last, laid her head upon the pillow, and, like a wearied child, soon fell into a peaceful and dreamless sleep.