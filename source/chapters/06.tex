%!TeX root=../scarlettop.tex

\chapter{An Exquisite of `92}
\lettrine[lines=4]{S}{ir} Percy Blakeney, as the chronicles of the time inform us, was in this year of grace 1792, still a year or two on the right side of thirty. Tall, above the average, even for an Englishman, broad-shouldered and massively built, he would have been called unusually good-looking, but for a certain lazy expression in his deep-set blue eyes, and that perpetual inane laugh which seemed to disfigure his strong, clearly-cut mouth.

It was nearly a year ago now that Sir Percy Blakeney, Bart.\footnote{Baronet, the lowest inherited rank.}, one of the richest men in England, leader of all the fashions, and intimate friend of the Prince of Wales, had astonished fashionable society in London and Bath by bringing home, from one of his journeys abroad, a beautiful, fascinating, clever, French wife. He, the sleepiest, dullest, most British Britisher that had ever set a pretty woman yawning, had secured a brilliant matrimonial prize for which, as all chroniclers aver, there had been many competitors.

Marguerite St~Just had first made her \textit{début} in artistic Parisian circles, at the very moment when the greatest social upheaval the world has ever known was taking place within its very walls. Scarcely eighteen, lavishly gifted with beauty and talent, chaperoned only by a young and devoted brother, she had soon gathered round her, in her charming apartment in the Rue Richelieu, a coterie which was as brilliant as it was exclusive---exclusive, that is to say, only from one point of view: Marguerite St~Just was from principle and by conviction a republican---equality of birth was her motto---inequality of fortune was in her eyes a mere untoward accident, but the only inequality she admitted was that of talent. \enquote{Money and titles may be hereditary,} she would say, \enquote{but brains are not,} and thus her charming salon was reserved for originality and intellect, for brilliance and wit, for clever men and talented women, and the entrance into it was soon looked upon in the world of intellect---which even in those days and in those troublous times found its pivot in Paris---as the seal to an artistic career.

Clever men, distinguished men, and even men of exalted station formed a perpetual and brilliant court round the fascinating young actress of the Comédie Française, and she glided through republican, revolutionary, bloodthirsty Paris like a shining comet with a trail behind her of all that was most distinguished, most interesting, in intellectual Europe.

Then the climax came. Some smiled indulgently and called it an artistic eccentricity, others looked upon it as a wise provision, in view of the many events which were crowding thick and fast in Paris just then, but to all, the real motive of that climax remained a puzzle and a mystery. Anyway, Marguerite St~Just married Sir Percy Blakeney one fine day, just like that, without any warning to her friends, without a \textit{soirée de contrat\footnote{\enquote{contract evening}}} or \textit{dîner de fiançailles\footnote{\enquote{engagement dinner}}} or other appurtenances of a fashionable French wedding.

How that stupid, dull Englishman ever came to be admitted within the intellectual circle which revolved round \enquote{the cleverest woman in Europe,} as her friends unanimously called her, no one ventured to guess---a golden key is said to open every door, asserted the more malignantly inclined.

Enough, she married him, and \enquote{the cleverest woman in Europe} had linked her fate to that \enquote{demmed idiot} Blakeney, and not even her most intimate friends could assign to this strange step any other motive than that of supreme eccentricity. Those friends who knew, laughed to scorn the idea that Marguerite St~Just had married a fool for the sake of the worldly advantages with which he might endow her. They knew, as a matter of fact, that Marguerite St~Just cared nothing about money, and still less about a title; moreover, there were at least half a dozen other men in the cosmopolitan world equally well-born, if not so wealthy as Blakeney, who would have been only too happy to give Marguerite St~Just any position she might choose to covet.

As for Sir Percy himself, he was universally voted to be totally unqualified for the onerous post he had taken upon himself. His chief qualifications for it seemed to consist in his blind adoration for her, his great wealth, and the high favour in which he stood at the English court; but London society thought that, taking into consideration his own intellectual limitations, it would have been wiser on his part had he bestowed those worldly advantages upon a less brilliant and witty wife.

Although lately he had been so prominent a figure in fashionable English society, he had spent most of his early life abroad. His father, the late Sir Algernon Blakeney, had had the terrible misfortune of seeing an idolized young wife become hopelessly insane after two years of happy married life. Percy had just been born when the late Lady Blakeney fell a prey to the terrible malady which in those days was looked upon as hopelessly incurable and nothing short of a curse of God upon the entire family. Sir Algernon took his afflicted young wife abroad, and there presumably Percy was educated, and grew up between an imbecile mother and a distracted father, until he attained his majority. The death of his parents following close upon one another left him a free man, and as Sir Algernon had led a forcibly simple and retired life, the large Blakeney fortune had increased tenfold.

Sir Percy Blakeney had travelled a great deal abroad, before he brought home his beautiful, young, French wife. The fashionable circles of the time were ready to receive them both with open arms. Sir Percy was rich, his wife was accomplished, the Prince of Wales took a very great liking to them both. Within six months they were the acknowledged leaders of fashion and of style. Sir Percy's coats were the talk of the town, his inanities were quoted, his foolish laugh copied by the gilded youth at Almack's or the Mall. Everyone knew that he was hopelessly stupid, but then that was scarcely to be wondered at, seeing that all the Blakeneys, for generations, had been notoriously dull, and that his mother had died an imbecile.

Thus society accepted him, petted him, made much of him, since his horses were the finest in the country, his \textit{fêtes} and wines the most sought after. As for his marriage with \enquote{the cleverest woman in Europe,} well! the inevitable came with sure and rapid footsteps. No one pitied him, since his fate was of his own making. There were plenty of young ladies in England, of high birth and good looks, who would have been quite willing to help him to spend the Blakeney fortune, whilst smiling indulgently at his inanities and his good-humoured foolishness. Moreover, Sir Percy got no pity, because he seemed to require none---he seemed very proud of his clever wife, and to care little that she took no pains to disguise that good-natured contempt which she evidently felt for him, and that she even amused herself by sharpening her ready wits at his expense.

But then Blakeney was really too stupid to notice the ridicule with which his clever wife covered him, and if his matrimonial relations with the fascinating Parisienne had not turned out all that his hopes and his dog-like devotion for her had pictured, society could never do more than vaguely guess at it.

In his beautiful house at Richmond he played second fiddle to his clever wife with imperturbable \textit{bonhomie}; he lavished jewels and luxuries of all kinds upon her, which she took with inimitable grace, dispensing the hospitality of his superb mansion with the same graciousness with which she had welcomed the intellectual coterie of Paris.

Physically, Sir Percy Blakeney was undeniably  handsome---always excepting the lazy, bored look which was habitual to him. He was always irreproachably dressed, and wore the exaggerated \enquote{Incroyable} fashions, which had just crept across from Paris to England, with the perfect good taste innate in an English gentleman. On this special afternoon in September, in spite of the long journey by coach, in spite of rain and mud, his coat set irreproachably across his fine shoulders, his hands looked almost femininely white, as they emerged through billowy frills of finest Mechlin lace: the extravagantly short-waisted satin coat, wide-lapelled waistcoat, and tight-fitting striped breeches, set off his massive figure to perfection, and in repose one might have admired so fine a specimen of English manhood, until the foppish ways, the affected movements, the perpetual inane laugh, brought one's admiration of Sir Percy Blakeney to an abrupt close.

He had lolled into the old-fashioned inn parlour, shaking the wet off his fine overcoat; then putting up a gold-rimmed eye-glass to his lazy blue eye, he surveyed the company, upon whom an embarrassed silence had suddenly fallen.

\enquote{How do, Tony? How do, Ffoulkes?} he said, recognising the two young men and shaking them by the hand. \enquote{Zounds, my dear fellow,} he added, smothering a slight yawn, \enquote{did you ever see such a beastly day? Demmed climate this.}

With a quaint little laugh, half of embarrassment and half of sarcasm, Marguerite had turned towards her husband, and was surveying him from head to foot, with an amused little twinkle in her merry blue eyes.

\enquote{La!} said Sir Percy, after a moment or two's silence, as no one offered any comment, \enquote{how sheepish you all look... What's up?}

\enquote{Oh, nothing, Sir Percy,} replied Marguerite, with a certain amount of gaiety, which, however, sounded somewhat forced, \enquote{nothing to disturb your equanimity---only an insult to your wife.}

The laugh which accompanied this remark was evidently intended to reassure Sir Percy as to the gravity of the incident. It apparently succeeded in that, for, echoing the laugh, he rejoined placidly---

\enquote{La, m'dear! you don't say so. Begad! who was the bold man who dared to tackle you---eh?}

Lord Tony tried to interpose, but had no time to do so, for the young Vicomte had already quickly stepped forward.

\enquote{Monsieur,} he said, prefixing his little speech with an elaborate bow, and speaking in broken English, \enquote{my mother, the Comtesse de Tournay de Basserive, has offenced Madame, who, I see, is your wife. I cannot ask your pardon for my mother; what she does is right in my eyes. But I am ready to offer you the usual reparation between men of honour.}

The young man drew up his slim stature to its full height and looked very enthusiastic, very proud, and very hot as he gazed at six foot odd of gorgeousness, as represented by Sir Percy Blakeney, Bart.

\enquote{Lud, Sir Andrew,} said Marguerite, with one of her merry infectious laughs, \enquote{look on that pretty picture---the English turkey and the French bantam.}

The simile was quite perfect, and the English turkey looked down with complete bewilderment upon the dainty little French bantam, which hovered quite threateningly around him.

\enquote{La! sir,} said Sir Percy at last, putting up his eye-glass and surveying the young Frenchman with undisguised wonderment, \enquote{where, in the cuckoo's name, did you learn to speak English?}

\enquote{Monsieur!} protested the Vicomte, somewhat abashed at the way his warlike attitude had been taken by the ponderous-looking Englishman.

\enquote{I protest `tis marvellous!} continued Sir Percy, imperturbably, \enquote{demmed marvellous! Don't you think so, Tony---eh? I vow I can't speak the French lingo like that. What?}

\enquote{Nay, I'll vouch for that!} rejoined Marguerite. \enquote{Sir Percy has a British accent you could cut with a knife.}

\enquote{Monsieur,} interposed the Vicomte earnestly, and in still more broken English, \enquote{I fear you have not understand. I offer you the only posseeble reparation among gentlemen.}

\enquote{What the devil is that?} asked Sir Percy, blandly.

\enquote{My sword, Monsieur,} replied the Vicomte, who, though still bewildered, was beginning to lose his temper.

\enquote{You are a sportsman, Lord Tony,} said Marguerite, merrily; \enquote{ten to one on the little bantam.}

But Sir Percy was staring sleepily at the Vicomte for a moment or two, through his partly closed heavy lids, then he smothered another yawn, stretched his long limbs, and turned leisurely away.

\enquote{Lud love you, sir,} he muttered good-humouredly. \enquote{Demmit, young man, what's the good of your sword to me?}

What the Vicomte thought and felt at that moment, when that long-limbed Englishman treated him with such marked insolence, might fill volumes of sound reflections... What he said resolved itself into a single articulate word, for all the others were choked in his throat by his surging wrath---

\enquote{A duel, Monsieur,} he stammered.

Once more Blakeney turned, and from his high altitude looked down on the choleric little man before him; but not even for a second did he seem to lose his own imperturbable good-humour. He laughed his own pleasant and inane laugh, and burying his slender, long hands into the capacious pockets of his overcoat, he said leisurely---

\enquote{A duel? La! is that what he meant? Odd's fish! you are a bloodthirsty young ruffian. Do you want to make a hole in a law-abiding man?... As for me, sir, I never fight duels,} he added, as he placidly sat down and stretched his long, lazy legs out before him. \enquote{Demmed uncomfortable things, duels, ain't they, Tony?}

Now the Vicomte had no doubt vaguely heard that in England the fashion of duelling amongst gentlemen had been suppressed by the law with a very stern hand; still to him, a Frenchman, whose notions of bravery and honour were based upon a code that had centuries of tradition to back it, the spectacle of a gentleman actually refusing to fight a duel was little short of an enormity. In his mind he vaguely pondered whether he should strike that long-legged Englishman in the face and call him a coward, or whether such conduct in a lady's presence might be deemed ungentlemanly, when Marguerite happily interposed.

\enquote{I pray you, Lord Tony,} she said in that gentle, sweet, musical voice of hers, \enquote{I pray you play the peacemaker. The child is bursting with rage, and,} she added with a \textit{soupçon} of dry sarcasm, \enquote{might do Sir Percy an injury.} She laughed a mocking little laugh, which, however, did not in the least disturb her husband's placid equanimity. \enquote{The British turkey has had the day,} she said. \enquote{Sir Percy would provoke all the saints in the calendar and keep his temper the while.}

But already Blakeney, good-humoured as ever, had joined in the laugh against himself.

\enquote{Demmed smart that now, wasn't it?} he said, turning pleasantly to the Vicomte. \enquote{Clever woman my wife, sir... You will find \textit{that} out if you live long enough in England.}

\enquote{Sir Percy is in the right, Vicomte,} here interposed Lord Antony, laying a friendly hand on the young Frenchman's shoulder. \enquote{It would hardly be fitting that you should commence your career in England by provoking him to a duel.}

For a moment longer the Vicomte hesitated, then with a slight shrug of the shoulders directed against the extraordinary code of honour prevailing in this fog-ridden island, he said with becoming dignity,---

\enquote{Ah, well! if Monsieur is satisfied, I have no griefs. You, mi'lor’, are our protector. If I have done wrong, I withdraw myself.}

\enquote{Aye, do!} rejoined Blakeney, with a long sigh of satisfaction, \enquote{withdraw yourself over there. Demmed excitable little puppy,} he added under his breath. \enquote{Faith, Ffoulkes, if that's a specimen of the goods you and your friends bring over from France, my advice to you is, drop `em `mid Channel, my friend, or I shall have to see old Pitt about it, get him to clap on a prohibitive tariff, and put you in the stocks an you smuggle.}

\enquote{La, Sir Percy, your chivalry misguides you,} said Marguerite, coquettishly, \enquote{you forget that you yourself have imported one bundle of goods from France.}

Blakeney slowly rose to his feet, and, making a deep and elaborate bow before his wife, he said with consummate gallantry,---

\enquote{I had the pick of the market, Madame, and my taste is unerring.}

\enquote{More so than your chivalry, I fear,} she retorted sarcastically.

\enquote{Odd's life, m'dear! be reasonable! Do you think I am going to allow my body to be made a pincushion of, by every little frog-eater who don't like the shape of your nose?}

\enquote{Lud, Sir Percy!} laughed Lady Blakeney as she bobbed him a quaint and pretty curtsey, \enquote{you need not be afraid! `Tis not the \textit{men} who dislike the shape of my nose.}

\enquote{Afraid be demmed! Do you impugn my bravery, Madame? I don't patronise the ring for nothing, do I, Tony? I've put up the fists with Red Sam before now, and---and he didn't get it all his own way either---}

\enquote{S'faith, Sir Percy,} said Marguerite, with a long and merry laugh, that went echoing along the old oak rafters of the parlour, \enquote{I would I had seen you then... ha! ha! ha! ha!---you must have looked a pretty picture... and... and to be afraid of a little French boy... ha! ha!... ha! ha!}

\enquote{Ha! ha! ha! he! he! he!} echoed Sir Percy, good-humouredly. \enquote{La, Madame, you honour me! Zooks!  Ffoulkes, mark ye that! I have made my wife laugh!---The cleverest woman in Europe!... Odd's fish, we must have a bowl on that!} and he tapped vigorously on the table near him. \enquote{Hey! Jelly! Quick, man! Here, Jelly!}

Harmony was once more restored. Mr Jellyband, with a mighty effort, recovered himself from the many emotions he had experienced within the last half hour.

\enquote{A bowl of punch, Jelly, hot and strong, eh?} said Sir Percy. \enquote{The wits that have just made a clever woman laugh must be whetted! Ha! ha! ha! Hasten, my good Jelly!}

\enquote{Nay, there is no time, Sir Percy,} interposed Marguerite. \enquote{The skipper will be here directly and my brother must get on board, or the \textit{Day Dream} will miss the tide.}

\enquote{Time, m'dear? There is plenty of time for any gentleman to get drunk and get on board before the turn of the tide.}

\enquote{I think, your ladyship,} said Jellyband, respectfully, \enquote{that the young gentleman is coming along now with Sir Percy's skipper.}

\enquote{That's right,} said Blakeney, \enquote{then Armand can join us in the merry bowl. Think you, Tony,} he added, turning towards the Vicomte, \enquote{that that jackanapes of yours will join us in a glass? Tell him that we drink in token of reconciliation.}

\enquote{In fact you are all such merry company,} said Marguerite, \enquote{that I trust you will forgive me if I bid my brother good-bye in another room.}

It would have been bad form to protest. Both Lord Antony and Sir Andrew felt that Lady Blakeney could not altogether be in tune with them at that moment. Her love for her brother, Armand St~Just, was deep and touching in the extreme. He had just spent a few weeks with her in her English home, and was going back to serve his country, at a moment when death was the usual reward for the most enduring devotion.

Sir Percy also made no attempt to detain his wife. With that perfect, somewhat affected gallantry which characterised his every movement, he opened the coffee-room door for her, and made her the most approved and elaborate bow, which the fashion of the time dictated, as she sailed out of the room without bestowing on him more than a passing, slightly contemptuous glance. Only Sir Andrew Ffoulkes, whose every thought since he had met Suzanne de Tournay seemed keener, more gentle, more innately sympathetic, noted the curious look of intense longing, of deep and hopeless passion, with which the inane and flippant Sir Percy followed the retreating figure of his brilliant wife.