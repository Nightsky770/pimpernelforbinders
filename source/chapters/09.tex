%!TeX root=../scarlettop.tex

\chapter{The Outrage}
\lettrine[lines=4]{A}{} beautiful starlit night had followed on the day of incessant rain: a cool, balmy, late summer's night, essentially English in its suggestion of moisture and scent of wet earth and dripping leaves.

The magnificent coach, drawn by four of the finest thoroughbreds in England, had driven off along the London road, with Sir Percy Blakeney on the box, holding the reins in his slender feminine hands, and beside him Lady Blakeney wrapped in costly furs. A fifty-mile drive on a starlit summer's night! Marguerite had hailed the notion of it with delight... Sir Percy was an enthusiastic whip; his four thoroughbreds, which had been sent down to Dover a couple of days before, were just sufficiently fresh and restive to add zest to the expedition, and Marguerite revelled in anticipation of the few hours of solitude, with the soft night breeze fanning her cheeks, her thoughts wandering, whither away? She knew from old experience that Sir Percy would speak little, if at all: he had often driven her on his beautiful coach for hours at night, from point to point, without making more than one or two casual remarks upon the weather or the state of the roads. He was very fond of driving by night, and she had very quickly adopted his fancy: as she sat next to him hour after hour, admiring the dexterous, certain way in which he handled the reins, she often wondered what went on in that slow-going head of his. He never told her, and she had never cared to ask.

At \enquote{The Fisherman's Rest} Mr Jellyband was going the round, putting out the lights. His bar customers had all gone, but upstairs in the snug little bedrooms, Mr Jellyband had quite a few important guests: the Comtesse de Tournay, with Suzanne, and the Vicomte, and there were two more bedrooms ready for Sir Andrew Ffoulkes and Lord Antony Dewhurst, if the two young men should elect to honour the ancient hostelry and stay the night.

For the moment these two young gallants were comfortably installed in the coffee-room, before the huge log-fire, which, in spite of the mildness of the evening, had been allowed to burn merrily.

\enquote{I say, Jelly, has everyone gone?} asked Lord Tony, as the worthy landlord still busied himself clearing away glasses and mugs.

\enquote{Everyone, as you see, my lord.}

\enquote{And all your servants gone to bed?}

\enquote{All except the boy on duty in the bar, and,} added Mr Jellyband with a laugh, \enquote{I expect he'll be asleep afore long, the rascal.}

\enquote{Then we can talk here undisturbed for half an hour?}

\enquote{At your service, my lord... I'll leave your candles on the dresser... and your rooms are quite ready... I sleep at the top of the house myself, but if your lordship'll only call loudly enough, I daresay I shall hear.}

\enquote{All right, Jelly... and... I say, put the lamp out---the fire'll give us all the light we need---and we don't want to attract the passer-by.}

\enquote{All ri’, my lord.}

Mr Jellyband did as he was bid---he turned out the quaint old lamp that hung from the raftered ceiling and blew out all the candles.

\enquote{Let's have a bottle of wine, Jelly,} suggested Sir Andrew.

\enquote{All ri’, sir!}

Jellyband went off to fetch the wine. The room now was quite dark, save for the circle of ruddy and fitful light formed by the brightly blazing logs in the hearth.

\enquote{Is that all, gentlemen?} asked Jellyband, as he returned with a bottle of wine and a couple of glasses, which he placed on the table.

\enquote{That'll do nicely, thanks, Jelly!} said Lord Tony.

\enquote{Good-night, my lord! Good-night, sir!}

\enquote{Good-night, Jelly!}

The two young men listened, whilst the heavy tread of Mr Jellyband was heard echoing along the passage and staircase. Presently even that sound died out, and the whole of \enquote{The Fisherman's Rest} seemed wrapt in sleep, save the two young men drinking in silence beside the hearth.

For a while no sound was heard, even in the coffee-room, save the ticking of the old grandfather's clock and the crackling of the burning wood.

\enquote{All right again this time, Ffoulkes?} asked Lord Antony at last.

Sir Andrew had been dreaming evidently, gazing into the fire, and seeing therein, no doubt, a pretty, piquant face, with large brown eyes and a wealth of dark curls round a childish forehead.

\enquote{Yes!} he said, still musing, \enquote{all right!}

\enquote{No hitch?}

\enquote{None.}

Lord Antony laughed pleasantly as he poured himself out another glass of wine.

\enquote{I need not ask, I suppose, whether you found the journey pleasant this time?}

\enquote{No, friend, you need not ask,} replied Sir Andrew, gaily. \enquote{It was all right.}

\enquote{Then here's to her very good health,} said jovial Lord Tony. \enquote{She's a bonnie lass, though she \textit{is} a French one. And here's to your courtship---may it flourish and prosper exceedingly.}

He drained his glass to the last drop, then joined his friend beside the hearth.

\enquote{Well! you'll be doing the journey next, Tony, I expect,} said Sir Andrew, rousing himself from his meditations, \enquote{you and Hastings, certainly; and I hope you may have as pleasant a task as I had, and as charming a travelling companion. You have no idea, Tony...}

\enquote{No! I haven't,} interrupted his friend pleasantly, \enquote{but I'll take your word for it. And now,} he added, whilst a sudden earnestness crept over his jovial young face, \enquote{how about business?}

The two young men drew their chairs closer together, and instinctively, though they were alone, their voices sank to a whisper.

\enquote{I saw the Scarlet Pimpernel alone, for a few moments in Calais,} said Sir Andrew, \enquote{a day or two ago. He crossed over to England two days before we did. He had escorted the party all the way from Paris, dressed---you'll never credit it!---as an old market woman, and driving---until they were safely out of the city---the covered cart, under which the Comtesse de Tournay, Mlle. Suzanne, and the Vicomte lay concealed among the turnips and cabbages. They, themselves, of course, never suspected who their driver was. He drove them right through a line of soldiery and a yelling mob, who were screaming, \enquote{À bas les aristos!} But the market cart got through along with some others, and the Scarlet Pimpernel, in shawl, petticoat and hood, yelled \enquote{À bas les aristos!} louder than anybody. Faith!} added the young man, as his eyes glowed with enthusiasm for the beloved leader, \enquote{that man's a marvel! His cheek is preposterous, I vow!---and that's what carries him through.}

Lord Antony, whose vocabulary was more limited than that of his friend, could only find an oath or two with which to show his admiration for his leader.

\enquote{He wants you and Hastings to meet him at Calais,} said Sir Andrew, more quietly, \enquote{on the 2\textsuperscript{nd} of next month. Let me see! that will be next Wednesday.}

\enquote{Yes.}

\enquote{It is, of course, the case of the Comte de Tournay, this time; a dangerous task, for the Comte, whose escape from his château, after he had been declared a \enquote{suspect} by the Committee of Public Safety, was a masterpiece of the Scarlet Pimpernel's ingenuity, is now under sentence of death. It will be rare sport to get \textit{him} out of France, and you will have a narrow escape, if you get through at all. St~Just has actually gone to meet him---of course, no one suspects St~Just as yet; but after that... to get them both out of the country! I'faith, `twill be a tough job, and tax even the ingenuity of our chief. I hope I may yet have orders to be of the party.}

\enquote{Have you any special instructions for me?}

\enquote{Yes! rather more precise ones than usual. It appears that the Republican Government have sent an accredited agent over to England, a man named Chauvelin, who is said to be terribly bitter against our league, and determined to discover the identity of our leader, so that he may have him kidnapped, the next time he attempts to set foot in France. This Chauvelin has brought a whole army of spies with him, and until the chief has sampled the lot, he thinks we should meet as seldom as possible on the business of the league, and on no account should talk to each other in public places for a time. When he wants to speak to us, he will contrive to let us know.}

The two young men were both bending over the fire, for the blaze had died down, and only a red glow from the dying embers cast a lurid light on a narrow semicircle in front of the hearth. The rest of the room lay buried in complete gloom; Sir Andrew had taken a pocket-book from his pocket, and drawn therefrom a paper, which he unfolded, and together they tried to read it by the dim red firelight. So intent were they upon this, so wrapt up in the cause, the business they had so much at heart, so precious was this document which came from the very hand of their adored leader, that they had eyes and ears only for that. They lost count of the sounds around them, of the dropping of crisp ash from the grate, of the monotonous ticking of the clock, of the soft, almost imperceptible rustle of something on the floor close beside them. A figure had emerged from under one of the benches; with snake-like, noiseless movements it crept closer and closer to the two young men, not breathing, only gliding along the floor, in the inky blackness of the room.

\enquote{You are to read these instructions and commit them to memory,} said Sir Andrew, \enquote{then destroy them.}

He was about to replace the letter-case into his pocket, when a tiny slip of paper fluttered from it and fell on to the floor. Lord Antony stooped and picked it up.

\enquote{What's that?} he asked.

\enquote{I don't know,} replied Sir Andrew.

\enquote{It dropped out of your pocket just now. It certainly did not seem to be with the other paper.}

\enquote{Strange!---I wonder when it got there? It is from the chief,} he added, glancing at the paper.

Both stooped to try and decipher this last tiny scrap of paper on which a few words had been hastily scrawled, when suddenly a slight noise attracted their attention, which seemed to come from the passage beyond.

\enquote{What's that?} said both instinctively. Lord Antony crossed the room towards the door, which he threw open quickly and suddenly; at that very moment he received a stunning blow between the eyes, which threw him back violently into the room. Simultaneously the crouching, snake-like figure in the gloom had jumped up and hurled itself from behind upon the unsuspecting Sir Andrew, felling him to the ground.

All this occurred within the short space of two or three seconds, and before either Lord Antony or Sir Andrew had time or chance to utter a cry or to make the faintest struggle. They were each seized by two men, a muffler was quickly tied round the mouth of each, and they were pinioned to one another back to back, their arms, hands, and legs securely fastened.

One man had in the meanwhile quietly shut the door; he wore a mask and now stood motionless while the others completed their work.

\enquote{All safe, citoyen!} said one of the men, as he took a final survey of the bonds which secured the two young men.

\enquote{Good!} replied the man at the door; \enquote{now search their pockets and give me all the papers you find.}

This was promptly and quietly done. The masked man having taken possession of all the papers, listened for a moment or two if there were any sound within \enquote{The Fisherman's Rest.} Evidently satisfied that this dastardly outrage had remained unheard, he once more opened the door and pointed peremptorily down the passage. The four men lifted Sir Andrew and Lord Antony from the ground, and as quietly, as noiselessly as they had come, they bore the two pinioned young gallants out of the inn and along the Dover Road into the gloom beyond.

In the coffee-room the masked leader of this daring attempt was quickly glancing through the stolen papers.

\enquote{Not a bad day's work on the whole,} he muttered, as he quietly took off his mask, and his pale, fox-like eyes glittered in the red glow of the fire. \enquote{Not a bad day's work.}

He opened one or two more letters from Sir Andrew Ffoulkes’ pocket-book, noted the tiny scrap of paper which the two young men had only just had time to read; but one letter specially, signed Armand St~Just, seemed to give him strange satisfaction.

\enquote{Armand St~Just a traitor after all,} he murmured. \enquote{Now, fair Marguerite Blakeney,} he added viciously between his clenched teeth, \enquote{I think that you will help me to find the Scarlet Pimpernel.}